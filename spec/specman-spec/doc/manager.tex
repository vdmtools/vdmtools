% WHAT

%    The main file of the documentation of the specification of 
%    the specification manager for the VDM-SL Toolbox
% FILE
%    $Source: /home/vdmtools/cvsroot/toolbox/spec/specman-spec/doc/manager.tex,v $
% VERSION
%    $Revision: 1.10 $
% DATE
%    $Date: 1997/08/18 13:47:14 $
% FORMAT
%    $State: Exp $
% PROJECT
%    Toolbox
% STATUS
%    Under development.
% AUTHOR
%    $Author: jeppe $
% COPYRIGHT
%    (C) 1995 IFAD, Denmark
 
\documentclass[11pt]{article}
\usepackage{vpp}
%%\usepackage[dvips]{color}
\usepackage{verbatimfiles,epsfig}
\usepackage{makeidx}
%%\usepackage{alltt}
\usepackage{toolbox} 
%%\usepackage{ifad} 

%\definecolor{covered}{rgb}{0,0,0}      %black
%%\definecolor{not_covered}{gray}{0.5}   %gray for previewing
%\definecolor{not_covered}{gray}{0.6}   %gray for printing
%%\definecolor{not_covered}{rgb}{1,0,0}  %read


\newcommand{\InstVarDef}[1]{{\bf #1}}
\newcommand{\TypeDef}[1]{{\bf #1}}
\newcommand{\TypeOcc}[1]{{\it #1}}
\newcommand{\FuncDef}[1]{{\bf #1}}
\newcommand{\FuncOcc}[1]{#1}
\newcommand{\MethodDef}[1]{{\bf #1}}
\newcommand{\MethodOcc}[1]{#1}
\newcommand{\ClassDef}[1]{{\sf #1}}
\newcommand{\ClassOcc}[1]{#1}

\def\insertfig#1#2#3#4{ % Filename, width, caption, label
\begin{figure}
\begin{center}
\epsfig{file=#1,width=#2,angle=-90}
\end{center}
\caption{#3} \label{#4}
\end{figure} 
}

%% \newcommand{\usecase}[11]{
%%   \begin{description}
%%   \item[Use Case no. ] 
%%   \item[Introduction] 
%%   \item[Type] 
%%   \item[Relations] 
%%   \item[Initialization] 
%%   \item[Actors] 
%%   \item[Preconditions] 
%%   \item[Description] 
%%   \item[Exeptions] 
%%   \item[Postconditions] 
%%   \end{description}     
%% }

\newcommand{\TBW}{TO BE WRITTEN}
\newcommand{\vdmpp}{{\small VDM}$^{++}$\/}
\newcommand{\vdmsl}{\small VDM-SL\/}
\newcommand{\VDM}{\small VDM\/}
\newcommand{\specman} {Specification Manager}

\makeindex

\begin{document}

\special{!userdict begin /bop-hook{gsave 220 30 translate
0 rotate /Times-Roman findfont 21 scalefont setfont
0 0 moveto (CONFIDENTIAL) show grestore}def end}

\docdef{Specification of the VDM-SL/VDM++ Toolbox Specification Manager}
{Henrik Voss,\\
 Jeppe Nyl\o{}kke J\o{}rgensen \\
  The Institute of Applied Computer Science}
{\today}
{IFAD-VDM-28}
{Report}
{Under Development}
{Confidential}
{\copyright IFAD}
{\item[V1.0] First version.
 \item[V1.1] The version used in release 3.0 of the VDM-SL Toolbox.
 \item[V1.2] Requirements added.
 \item[V1.3] Updated to include specification manager for VDM++ as
   well.
 \item[V1.4] Pretty printed version.
 \item[\mbox{}] \mbox{}}
{\mbox{}}

\pagenumbering{roman}
\tableofcontents
\newpage
\renewcommand{\thepage}{\arabic{page}}
\setcounter{page}{1}

\parskip12pt
\parindent0pt
     
\section{Introduction}


This document describes the design and specification of the \vdmsl{}
and \vdmpp{} Toolbox Specification Manager, written in \vdmpp{}.
\TBW{}.

\section{System Requirements}
\label{sec:req}

\subsection{General Description}
\label{sec:reqtext}

The motivation for the \specman{} is basically the introduction of the
GUI (Graphical User Interface) to the IFAD \vdmsl{} Toolbox. With the
GUI, two different interfaces are available for the toolbox, namely
the GUI and the traditional ascii interface. The main purposes of the
specification manager can be stated as:
\begin{itemize}
\item Keep track of the status for each module in the current project,
\item Decide legal actions for each module depending on the status of
  the module, 
\item Take care of error handling,
\item Form the interface to the internals of the toolbox,
\item Save the project in its current status, making it possible to
  quit the session and later restart at the same point.
\end{itemize}

The \specman{} should be viewed as an internal part of the toolbox
that is not directly related to the end user. Instead, it is related
to the developer of the toolbox. The functionality of the toolbox
wanted by the developer should be accessed via the \specman{}. 

It is not possible to foresee all possible requirements from the
developer to the internals of the toolbox. Instead, the \specman{}
should offer a suitable interface to the internals of the toolbox,
eventually including a basic set of functions (parsing, type checking,
code generation, pretty printing). It should also be possible to
define different interfaces depending on the developers need (\vdmsl{}
Toolbox, \vdmpp{} Toolbox, the Specification Animator). This could be
stated as calling conventions or calling structure to the internals of
the toolbox. 

\subsection{Definitions} 
\label{sec:def}
Core definitions common to \vdmsl{} and \vdmpp{}:
\begin{description}
\item[Project:] Consists of one or more files containing
  specification. 
\item[File:] For \vdmsl{} a file can contain one or more modules in
  case of a structured specification. In case of a flat specification,
  a file contains one or more definitions blocks. For \vdmpp{} a file
  contain one or more classes. Syntax checking and pretty printing are
  performed at files.
\item[Status:] Status for files concerns syntax checking and pretty
  printing and status for modules/classes concerns type checking and code
  generation. Status can be \texttt{None}(Action has not been
  performed), \texttt{OK}(Action has been performed successfully) or
  \texttt{Error}(Action has been performed with error).
\end{description}

Definitions specific to \vdmsl{}:
\begin{description}
\item[Module:] Contains the \vdmsl{} specification. In case of a flat
  specification, this specification is turned into a single module
  called ``DefaultMod''. Type checking and code generation is
  performed on modules.
\item[Session:] Describes the type of specification as either
  \textit{flat} or \textit{structured}. Flat and structured
  specifications cannot be parsed in the same session. 
\end{description}

Definitions specific to \vdmpp{}:
\begin{description}
\item[Class] Contains the \vdmpp{} specification. Type checking and
  code generation is performed on classes.
\end{description}

\subsection{Static Requirements}
\label{sec:statreq}

The information that the \specman{} must contain and operate on is
described as the \textit{state} of the \specman{}. This section
describes the requirements to this state. 

The core state should contain:
\begin{enumerate}
\item The project name,
\item The names of files and modules in current project,
\item Status for each file and module/class,
\item AST for each module/class,
\item Tag information for the code generator,
\item Error and warning messages,
\item Options,
\item Dependency information (for \vdmsl{} this is imported modules
  and for \vdmpp{} this is super and subclasses and uses and used by
  classes).
\end{enumerate}

For the \vdmsl{} Toolbox furthermore:
\begin{enumerate}
\item Session type
\end{enumerate}

For the \vdmpp{} Toolbox furthermore:
\begin{enumerate}
\item State for the type checker.
\end{enumerate}

\subsection{Functional Requirements Stated as Use cases}
\label{sec:requse}
 

\paragraph{Actors}

Two actors to the use cases were identified:
\begin{description}
\item[API:] The Application Program Interface, i.e. the user of
  \specman{} from the interface. 
\item[Toolbox:] The internal toolbox functionalities.
\end{description}


  \begin{description}
  {\large \item[Use Case no. 1:] Handling commands to the Toolbox}
\item[Introduction] \hfill\par This use case handles all possible
  commands that can be given to the toolbox. This includes actions on
  files and modules/classes, i.e. parsing and pretty printing of files
  and type checking, code generation and
  ``processing''\footnote{Syntax and type checking of a module/class and
    modules/classes it depends on} of modules/classes.
  \item[Type] Concrete
  \item[Relations] Extends
  \item[Initialisation] \hfill\par Activated by API
  \item[Actors] \hfill\par API
  \item[Preconditions] \hfill\par \specman{} must be instantiated. 
  \item[Description] \hfill\par All issued commands are passed to the
    Toolbox, i.e. both known and unknown commands.
    Furthermore, the following commands should be handled:
    \begin{itemize}
    \item Syntax check and pretty printing of a sequence of files.
    \item Type checking and code generation of a sequence of
      modules/classes.
    \item ``Processing'' a sequence of files. 
    \end{itemize}
    
    Commands not covered by the above (e.g. commands dedicated to
    special purposes) must be described in an \texttt{extends} use
    case. 

    It does not affect the funcional behaviour of the \specman{}
    whether the issued command is correct or not, or whether the
    command results in errors or not. Under all circumstances is the
    command passed to the Toolbox, and the result is always
    simply returned to the caller.
  \item[Exeptions] \hfill\par
    None.
  \item[Postconditions] \hfill\par None.
  \end{description}

  \begin{description}
  {\large \item[Use Case no. 2:] Extracting information from the \specman{} }
\item[Introduction] \hfill\par This information concerns allowed
  actions on files and modules/classes, which modules/classes a file contains,
  extracting ASTs etc.
  \item[Type] Concrete 
  \item[Relations] None
  \item[Initialisation] \hfill\par Activated by API or Toolbox
  \item[Actors] \hfill\par API, Toolbox
  \item[Preconditions] \hfill\par None
  \item[Description] \hfill\par 
    An actor can extract:
    \begin{enumerate}
    \item Allowed actions on files and modules/classes \textit{(Exception:
        File/module/class for action is not defined)}, 
    \item Compare session type (\vdmsl{} Toolbox only),
    \item Status for a file or module\textit{(Exception: File/module/class
        for status not defined)},
    \item Get file name from file id and vice versa
      \textit{(Exception: File id/name not defined)},defined files and
      modules, correspondence between files and modules/classes
      \textit{(Exception: Name not defined)},
    \item ASTs,
    \item Current project name\textit{(Exception: Project name has not
        been set)},
    \item Next, previous, first and last error/warning
      \textit{(Exception: The requested error/warning is not defined)}
      for the following commands: Syntax check, type check, code
      generation and pretty printing.
    \item the type checker state for the \vdmpp{} Toolbox.
    \end{enumerate}
  \item[Exeptions] \hfill\par \textit{File/module/class for action is not defined:}
    Give message to log. \hfill\par
    \textit{File/module/class for status is not defined:} Return nil, \hfill\par
    \textit{File id/name not defined:} Return 0/nil, \hfill\par
    \textit{Name not defined:} Return an empty set, \hfill\par
    \textit{Project name has not been set:} Return nil, \hfill\par
    \textit{The requested error/warning is not defined:} Ignore.
  \item[Postconditions] \hfill\par None
  \end{description}

\begin{description}   
{\large \item[Use Case no. 3:]  Updating the state of the \specman{}.}
\item[Introduction]  \hfill\par When the Toolbox receives a command, this
  command possibly will update the state of the \specman{}. This use
  case describes these commands. 
\item[Type] Concrete          
\item[Relations] Extends
\item[Initialization]  \hfill\par Activated by Toolbox.
\item[Actors]  \hfill\par Toolbox       
\item[Preconditions]  \hfill\par The API has issued the initial
  command and the \specman{} has passed the command to the 
  Toolbox. 
\item[Description]  \hfill\par 
  The state for the \specman{} must be updated with regards to:
  \begin{enumerate}
  \item Adding/removing of files,
  \item Status for files and modules/classes,
  \item ASTs and tag information for code generator,
  \item Errors and warnings,
  \item Options.
  \item The type checker state (only for the \vdmpp{} Toolbox).
  \end{enumerate}
\item[Exeptions]   \hfill\par  
\item[Postconditions]  \hfill\par Updated state for \specman{}.
\end{description}      

  \begin{description}
  {\large \item[Use Case no. 4:] Project handling (New, Open, Save,
    Save As)}
  \item[Introduction]  \hfill\par This use case handles changing of project.
  \item[Type] Concrete
  \item[Relations] None
  \item[Initialization]  \hfill\par Activated by API
  \item[Actors]  \hfill\par API
  \item[Preconditions]  \hfill\par None.
  \item[Description] \hfill\par The actor should be able to explicit
    set the name of current project, create a new project, save
    current project and open a named project. Furthermore should the
    actor be able to detect if there is unsaved changes in the current
    project. 
    \textit{(Exception: Project cannot be saved or loaded)}.  
  \item[Exeptions]  \hfill\par
    \textit{Project cannot be saved or loaded:} Give message to log 
  \item[Postconditions]  \hfill\par None.
  \end{description}

  \begin{description}
  {\large \item[Use Case no. 5:] Configuring the current project}
  \item[Introduction]  \hfill\par This use case handles adding and removing of
    files to the current project
  \item[Type] Concrete
  \item[Relations] None
  \item[Initialization]  \hfill\par
    Activated by API
  \item[Actors]  \hfill\par
    API
  \item[Preconditions]  \hfill\par None.
  \item[Description] \hfill\par Files should be added
    \textit{(Exception: File exists)} or removed \textit{(Exception: File
      does not exist)} from the current project.
  \item[Exeptions]  \hfill\par
    \textit{File exists:} Nothing should be done on the state. \hfill\par
    \textit{File does not exist:} Nothing should be done on the state. 
  \item[Postconditions]  \hfill\par None.
  \end{description}

\begin{description}   
{\large \item[Use Case no. 6:] Commands for the Specification Animator to the
  \vdmsl{} Toolbox }
\item[Introduction]  \hfill\par This use case extends use case no. 1
\item[Type] Concrete          
\item[Relations] Extends use case no. 1     
\item[Initialization] \hfill\par
\item[Actors]  \hfill\par API      
\item[Preconditions] \hfill\par 
\item[Description] \hfill\par   
  Commands needed by the Specification Animator that should be passed
  via the \specman{} to the \vdmsl{} Toolbox. These commands include: 
  \begin{enumerate}
  \item \vdmsl{} expressions parsed to abstract syntax trees,
  \item Semantic value converted to string,
  \item Parsing and evaluation of a \vdmsl{} value to a semantic
    value,
  \item Extract defined functions and operations into a string,
  \item Extract defined extended explicit operations and functions
    into a map of ASTs, 
  \item Comparing types.
  \end{enumerate}
\item[Exeptions]   \hfill\par   None.
\item[Postconditions] \hfill\par None.
\end{description}     

\begin{description}
{\large \item[Use Case no. 7:] Dependecy information for the \vdmpp{} Toolbox}
\item[Introduction]  \hfill\par This use case deals with the
  dependency information (super and subclasses, uses and used by). 
\item[Type] Concrete
\item[Relations] Extends use case no. 3
\item[Initialization]  \hfill\par Activated by the \vdmpp{} Toolbox.
\item[Actors]  \hfill\par Toolbox
\item[Preconditions]  \hfill\par 
\item[Description] \hfill\par Toolbox can insert dependency
  information for each class in the \specman{}. \hfill\par 
  Toolbox can extract either
  dependency or inheritance information for a named class
  \textit{(Exeption: Class does not exist)}.
\item[Exeptions]  \hfill\par \textit{Class does not exist:} Return
  empty information string.
\item[Postconditions]  \hfill\par Possibly updated state for
  dependency informatin. 
\end{description}     

\subsection{Document Requirements}
\label{sec:docreq}

\begin{description}
\item[Developers Manual to \specman{}] This manual must describe how
  the \specman {} is to be used by the developer. This includes the
  creation, how to get access to the toolbox internals, how to add
  extra functionalites. 
\end{description}

\insertfig{usecase.eps}{8cm}{Use case diagram for \specman{}}{fig:usecase}

%%% \section{System Context Diagram}
%%% \label{sec:context}
%%% 
%%% \insertfig{syscontext.eps}{3cm}{System Context Diagram}{}
%%%


\section{Class Structure}

This section describes the overall {\bf design} of the Specification
Manager and the structure of its specification. Furthermore the
specification {\bf style} used is described in Section
\ref{sec:style}.

\subsection{Design}
The overall ideas in the design of the \specman{} is the following:
\begin{itemize}
\item All calls to the internals of the Toolbox must go via the
  \specman{}. The BaseTools class is designed for this purpose. 
\item The \specman{} is to be used for both an ascii and graphical
  user interface. The Interface and Errors classes are abstract in the
  specification in order to make them independant of the kind of
  interface. 
\item The full functionality of the Toolbox can be accessed from the
  ToolMediator class. A restricted access to the Toolbox should be
  provided by the ToolKit class, forming an initial version API.
\item The Repository contains the status information for the current
  project in the Toolbox and is updated from the Toolbox. 
\item The Options class contains the options for the Toolbox. 
\end{itemize}

The design can be seen in figure \ref{fig:speccontext}.

\begin{figure}
\begin{center}
\epsfig{file=contextoverview.eps,width=13cm}
\end{center}
\caption{Context diagram and desing overview for \specman} \label{fig:speccontext}
\end{figure} 

%%\insertfig{toolkit.eps}{5cm}{Toolkit}{}
%% \insertfig{context.eps}{5cm}{Context for the \specman{}.}{fig:context}
%% \insertfig{topology.eps}{5cm}{Topology}{}

\subsection{Specification Style}
\label{sec:style}

The specification is described as a combination of imperative and
functional parts. That is, methods and functions are used in
combination.

Each class is described in a separate section by means of:
\begin{itemize}
\item An introduction to the class.
\item An OMT class diagram giving the interface to the class. This
  diagram will usually have been automatically generated by the
  reverse \vdmpp{} to OMT generator.
\item The \vdmpp{} specification of the class.
\end{itemize}


The following name convention are used in the specification:
\begin{itemize}
\item All class, type, function, value and method names starts with a
  capital letter.
\item All local variable names start with a lower case letter. An
  exception is when a variable clashes a key word. In this case the
  variable will start with an upper case letter.
\item For variables of type sequence, maps, sets, records, tuples and
  object reference the name might have a postfix being a underscore
  concatenated with the letter:
  \begin{description}
  \item[sequence: ] l
  \item[map: ] m
  \item[set: ] s
  \item[record: ] r
  \item[tuple: ] t
  \item[object reference: ] o
  \end{description}

  Underscores do not appear in variable names otherwise.

\end{itemize}

%%%%%%%%%%%
%%
%% The actual specification
%%
%%%%%%%%%%%

\subsection{Toolkit}
\label{sec:tkit}

The classes in this section forms the coupling between the user, the
internals of the \specman{} and the functionalities of the Toolbox.
This is achived using the \textit{Mediator} behaviour pattern as
described in \cite{DESIGNPAT95}. The Toolkit class is the public
interface to the methods in the class ToolMediator. The
\textit{intension} behind this is that not all methods in the
ToolMediator class should be made public. Only methods that the user
should be able to access should be defined in the Toolkit class. In
\textit{reality} this is implemented in another way in order to reduce
the number of methods. For each class that the ToolMediator refers to
there is a method returning a reference to the actual class. All
functionalities for the class can then be accessed by using this
reference. This reduces the number of methods in the classes Toolkit
and ToolMediator significantly, but on the other hand this makes all
methods public. 

\subsubsection{Class ToolKit}
\insertfig{toolstotalview.eps}{12cm}{The ToolKit and ToolMediator
  classes}{fig:toolstotal}

The Toolkit class is the fundamental class in the Specification
Manager in the sense that this class is the one that instantiates the
rest of the \specman{}. This class forms the interface to the
ToolMediator class. The Toolkit class must contain at least one
initialization method in order to set up object references to the
abstract classes as defined by the user (see section
\ref{sec:using}) and instantiate the proper BaseTools class
(\ref{sec:newcall}).  

\input{toolkit.vpp.tex}

\subsubsection{Class ToolMediator}
The ToolMediator class is the central class in the \specman{}. This
class contains references to the different ToolColleague subclasses
and makes it possible for each class to access methods in other
classes. The ToolMediator class must contain initialisation methods
corresponding to initialisation methods in class Toolkit.  


%% NOTE: Comments on call back to interface

\input{mediator.vpp.tex}

\subsubsection{Class ToolColleague}
\insertfig{toolcollinherit.eps}{9cm}{ToolColleague inheritance and
  associations for ToolMediator.}{fig:toolcoll}

This class is superclass for all classes that need an object
reference to the ToolMediator class as part of the instance variable. 

\input{colleague.vpp.tex}


%%%%%%%%%%%

\subsection{Repository Interface}
\label{sec:repos}

The repository subsystem of the \specman{} contains information about
status for files and modules/classes and the corresponding AST's.  The
Repository class forms the interface to all repository related classes
and instantiates the classes of the subsystem. The associations
between the Repository class and the subsystem classes can be seen in
figure \ref{fig:repass}. 

\subsubsection{Class Repository}

The primary role for the Repository class is to delegate the different
repository calls to the prober subsystem class and in this way form a
uniform interface to the caller independant of the subsystem
structure.

\insertfig{repositoryasso.eps}{17cm}{Associations for Repository class}{fig:repass}
\insertfig{repositoryinterface.eps}{9cm}{Interface to Repository class}{fig:repinter}
\input{repository.vpp.tex}

\subsubsection{Class CGRepository}
\input{cgrepos.vpp.tex}

%%%%%%%%%%%

\subsection{Database for Repository}

The database for the repository is the class that contains the file
and module/class status information as part of the object state. 

\subsubsection{Class RepDatabase}

The core data in the RepDatabase class is the two maps for file status
and module/class status. These maps contain object references to the
FileStatus and VDMUnitElem classes as can be seen from figure \ref{fig:repdb}

%% \insertfig{repdatabaseview.eps}{9cm}{Associations for RepDatabase class}{}
%% \insertfig{repdbclass.eps}{9cm}{Interface to RepDatabase class}{}
\insertfig{repdatabasedetailview.eps}{12cm}{Associations and Interface
  for database related classes.}{fig:repdb}
%%\input{repdb.vpp.tex}
\input{repdb.vpp.tex}
\subsubsection{Class VDMUnitElem}
\input{unit.vpp.tex}

\subsubsection{Class UnitStatus}
\input{unitstatus.vpp.tex}

\subsubsection{Class AST}
\input{ast.vpp.tex}

\subsubsection{Class FileStatus}
\input{filestatus.vpp.tex}

%%%%%%%%%%%

\subsection{Status Info for Files and Modules/Classes}

The StatusInfo class extracts status and allowed actions for files and
modules/classes. 

\subsubsection{Class StatusInfo}
\insertfig{statusinfointerface.eps}{3.5cm}{Interface to StatusInfo class}{fig:status}
\input{statusinfo.vpp.tex}

%%%%%%%%%%%

\subsection{Project Handling}

Project handling concerns creation of a new project and opening an
existing project. Furthermore must it be possible to load and save the
state of a project. 

\subsubsection{Class UpdateProject}
\insertfig{updateprojinterface.eps}{10cm}{Interface to UpdateProject
  and StateStore classes.}{fig:update}
\input{updateproj.vpp.tex}

\subsubsection{Class StateStore}
\input{statestore.vpp.tex}

%%%%%%%%%%%

\subsection{Session Handling}

The session handling classes are a \textit{state} pattern
(\cite{DESIGNPAT95}) with the UpdateSes as super class. This class
contains methods common to all subclasses. Only the behaviour of the
syntax checker depends on the type of session, i.e. whether it is a
flat or structured specification. 

\subsubsection{Class UpdateSes}
\insertfig{updatesesinterface.eps}{10cm}{Interface and inheritance for
  UpdateSes classes.}{fig:sess} 
\input{updateses.vpp.tex}

\subsubsection{Class NoneSes}
\input{noneses.vpp.tex}

\subsubsection{Class StructSes}
\input{structses.vpp.tex}

\subsubsection{Class FlatSes}
\input{flatses.vpp.tex}

%%%%%%%%%%%


\subsection{Dependency Information}
\label{sec:dep}

The dependency information is defined in the \vdmsl{} module $DEP$,
the ``Dependency Abstract Syntax''. The $Dependency$ is an abstract
class interface to this module. 

\subsubsection{Class Dependency}
\insertfig{dependinterface.eps}{5cm}{Interface for Dependency class}{fig:dep}
\input{depend.vpp.tex}


%%%%%%%%%%%

\subsection{BaseTools}

The BaseTools and subclasses are the \specman{} interface to the
internals of the \vdmsl{} and \vdmpp{} toolboxes. The BaseTools class
defines the fundamental functionalites needed by the ascii and gui
interface to the toolboxes. Enhanced functionalities for the
Specification Animator is defined in the SABaseTools class. 

\subsubsection{Class BaseTools}
\insertfig{basetoolsinterface.eps}{8cm}{Interface and inheritance for
 BaseTools  classes.}{fig:base} 
\input{basetools.vpp.tex}


%\subsubsection{Class SABaseTools}
%\input{sabasetools.vpp.tex}



%%%%%%%%%%%

\subsection{Interface}

The Interface class is an abstract class. It defines how to handle log
and error messages and what to do when a call back is received.

\subsubsection{Class Interface}
\insertfig{interfaceinterface.eps}{2.5cm}{Interface for the Interface  class.}{fig:inter} 
\input{interface.vpp.tex}


%%%%%%%%%%%


\subsection{Errors}

The Errors class is an abstract class for errors arising from the
internal tools in the toolbox (i.e. syntax checker, type checker
etc.). It has an association to an ErrorState class, which is a
\textit{state} pattern. The ErrorState subclasses defines the
different ways of showing errors.

\subsubsection{Class Errors}
\insertfig{errorsinterface.eps}{11cm}{Interface and inheritance for
  Errors and ErrorState classes.}{fig:errors}
\input{errors.vpp.tex}

\subsubsection{Class ErrorState}
\input{errstate.vpp.tex}

\subsubsection{Class BatchErr}
\input{batcherr.vpp.tex} 

\subsubsection{Class PromptErr}
\input{prompterr.vpp.tex}

\subsubsection{Class ScriptErr}
\input{scripterr.vpp.tex}

\subsubsection{Class NonPrintErr}
\input{nonprinterr.vpp.tex}
\insertfig{errorstate.eps}{7cm}{States for Errors class}{fig:state}
 
%%%%%%%%%%%

\subsection{Options}

This class is not yet used. 

\subsubsection{Class Options}
\input{options.vpp.tex}

%%%%%%%%%%%

%%%%%%%%%%%


\subsection{Global Types}

In order to share types between classes a global superclass
ProjectTypes is defined. Classes that need types from this class must
be subclasses of ProjectTypes. In practise this means that all classes
in the \specman{} are subclasses of ProjectTypes. 

Some types are implemented different from the specification. This is
types that are outside the specification, e.g. the AST. The
\texttt{AstVal} is specified as a record containing a token but in the
implementation the token is exchanged with the real AST record type.
This can be done because the code generator only generates the first
level of the type, i.e. a Record.

\subsubsection{Class ProjectTypes}
\input{ptypes.vpp.tex}

\section{Using the Specification Manager in the Toolbox development}
\label{sec:using}

\subsection{Setting up the Specification Manager}
\label{sec:setup}

In order to make an instance of the Specification Manager, you must:
\begin{enumerate}
\item Define the abstract classes Errors and Interface.
\item Instantiate a ToolKit class.
\item Create an object reference to an Interface class.
\item Instanciate two Errors classes and create two object references
  to them.
\item Create proper object reference to the initial state for each
  Errors instance and call the Errors \texttt{InitState} method.
\item Initialize the ToolKit instance with the Interface and Errors
  object references.
\end{enumerate}

In the main program for the ascii interface, \texttt{vdmde.cc}, the
instantiation part looks like:
\begin{verbatim}
  
  // Set up the ToolKit
  //  (we use the default settings of vdm_log and vdm_err)
  toolkit = new vdm_ToolKit ();
  ObjectRef TK (toolkit);
  ObjectRef inf (new AsciiInterface ());

  AsciiErrors * errs = new AsciiErrors();
  ObjectRef ERRS (errs);

  AsciiErrors * exprerrs = new AsciiErrors();
  ObjectRef ExprERRS (exprerrs);

  //  The Errors class must be initiated with a state.
  //  This is PromptErr in case of Ascii interface,
  //  and BatchErr in case of GUI or batch version.
    
  ObjectRef batch_state (new vdm_BatchErr());
  errs->vdm_InitState(batch_state);
  
  exprerrs->vdm_InitState(batch_state);
  toolkit->vdm_Init (inf,ERRS, ExprERRS);
\end{verbatim}

\subsection{Abstract classes}

The \textbf{Interface} and \textbf{Errors} classes are abstract
classes in the specification and it is the users responsibility to
define the concrete subclasses as needed in the implementation. This
way the \specman{} can be adapted to the different needs from e.g. an
ascii interface or a graphical interface.  Only the \textit{delegated}
methods needs to be defined.

\subsubsection{Call back}

The \specman{} generates call back calls to the Interface when parts
of the Interface must be changed or updated. E.g. when a file is added
or removed. The user must define which actions the Interface will
take. 

\subsection{Calling the \specman}
\label{sec:call}

The \specman{} is called either from the Toolbox or from the
Interface. Calls from the Interface to the Toolbox should go through
the \specman{}. In order to access the type definitions in class
ProjectTypes auxiliary functions is defined in
\texttt{projectval.h}. For each type \texttt{<type>} in class
ProjectTypes there should be a \texttt{mk\_<type>}-function and a
\texttt{Extract<type>}-function. If necessary there is also a
\texttt{is\_<type>}-function. From \texttt{projectval.h}:

\begin{verbatim}
  Record mk_ToolCommand (String command);
  bool is_ToolCommand(Record toolcom); 
  String ExtractToolCommand (Record ToolCommand);
\end{verbatim}

\subsection{Adding new calls to Toolbox}
\label{sec:newcall}
 
Calls from the Interface to the Toolbox go via the BaseTools classes.
These methods are defined as userimplinary methods and must be hand
implemented. The convention is that the BaseTools methods just pass
the call to a corresponding method in \texttt{tools.cc}. If needed, a
new subclass to BaseTools can be defined. It must then be assured that
the Toolkit class is initialised with this class (see also section
\ref{sec:tkit}).
  
\appendix


\section{Implementation of userimplinary methods}
\label{sec:impuserimpl}

\subsection{Class BaseTools}
\label{sec:basetoolsuserimpl}
\small
\verbatimfile{../../../code/specman/code/BaseTools_userimpl.cc}


%\subsection{Class SABaseTools}
%\label{sec:sauserimpl}
%\verbatimfile{../../../code/specman/code/SABaseTools_userimpl.cc}


\subsection{Class Dependency}
\label{sec:depuserimpl}

\verbatimfile{../../../code/specman/code/Dependency_userimpl.cc}


\subsection{Class StateStore}
\label{sec:storeuserimpl}
\verbatimfile{../../../code/specman/code/StateStore_userimpl.cc}


\subsection{Class CGRepository}
\label{sec:cguserimpl}
\verbatimfile{../../../code/specman/code/CGRepository_userimpl.cc}


\subsection{Class Options}
\label{sec:optuserimpl}
userimplinary methods is not yet implemented.

\subsection{Class ToolMediator}
\label{sec:meduserimpl}
userimplinary methods is not yet implemented.

\newpage

\bibliographystyle{newalpha}
%\bibliography{/home/peter/bib/dan}
\bibliography{../../../doc/bib/ifad.bib}
\newpage
\addcontentsline{toc}{section}{Index}
\printindex

 
\end{document}
