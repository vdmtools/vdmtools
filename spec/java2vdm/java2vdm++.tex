\documentclass[a4paper,dvips]{article}
\usepackage[dvips]{color}
\usepackage{vdmsl-2e}
\usepackage{longtable}
\usepackage{makeidx}
\usepackage{ifad}
\usepackage{colortbl}
\usepackage{multicol}
\usepackage{alltt}

\newenvironment{formalparameters}{
\vspace{2ex}\hrule\vspace{1ex}
\noindent\textit{Formal Parameters}\\[2ex]
\begin{tabular}{|>{\ttfamily}l|p{7cm}|}\hline}{%
\hline\hline\end{tabular}\vspace{1ex}}
 
\newcommand{\methodresult}[2]{%
\vspace{2ex}\hrule\vspace{1ex}
\noindent\textit{The Results}\\[2ex]
\begin{tabular}{|>{\ttfamily}l|p{7cm}|}\hline
#1 & #2 \\
\hline\hline\end{tabular}\vspace{1ex}}


\newcommand{\TBW}{\fbox{To be written}}
\newcommand{\TBD}{\fbox{To be discussed}}

\newcommand{\VDMTools}{\textsf{\textbf{VDMTools\raisebox{1ex}{\Pisymbol{psy}{226}}}}} 


% VDMgray environment
\newcolumntype{G}{>{\columncolor[gray]{0.8}}p{.98\columnwidth}}
\newenvironment{VDMgray}%
{\small\begin{alltt}\begin{tabular}{G}}%
{\end{tabular}\end{alltt}\normalsize} 

\usepackage{vdmsl-2e}
\usepackage{vpp}
\newcommand{\vdmstyle}[1]{\texttt{#1}}

\definecolor{covered}{rgb}{0,0,0}      %black
%\definecolor{not_covered}{gray}{0.5}   %gray for previewing
%\definecolor{not_covered}{gray}{0.6}   %gray for printing
\definecolor{not_covered}{rgb}{1,0,0}  %read

%\documentstyle[vdmsl_v1.1.31,alltt,makeidx,ifad,a4]{article}

\newcommand{\VDM}{VDM++}

\newcommand{\StateDef}[1]{{\bf #1}}
\newcommand{\TypeDef}[1]{{\bf #1}}
\newcommand{\TypeOcc}[1]{{\it #1}}
\newcommand{\FuncDef}[1]{{\bf #1}}
\newcommand{\FuncOcc}[1]{#1}
\newcommand{\ModDef}[1]{{\tiny #1}}


\newcommand{\MCL}{Meta-IV Class Library}
\newcommand{\NYI}{Notice, that this function/operation is not fully
  specified in order to capture a possible integration with another
  library than \MCL{}.}
\newcommand{\nfs}{{\em not fully specified\/}}

\makeindex

\begin{document}

\special{!userdict begin /bop-hook{gsave 220 30 translate
0 rotate /Times-Roman findfont 21 scalefont setfont
0 0 moveto (CONFIDENTIAL) show grestore}def end}

\docdef{Specification of the Java to VDM++ Translator}
         {IFAD VDM Tool Group \\
          IFAD A/S}
         {\today,}
         {IFAD-Legacy-1}
         {Report}
         {Under Development}
         {Confidential}
         {}
         {\copyright IFAD}
         {\item[V1.0] First version.}
         {}

\renewcommand{\thepage}{\roman{page}}

\tableofcontents
\newpage
\renewcommand{\thepage}{\arabic{page}}
\setcounter{page}{1}

\parskip12pt
\parindent0pt


\section{Introduction}

This document contains the specification of the Java to VDM++
translator. The Java language is described in \cite{Gosling&00}. In
essense it will be the reverse of the VDM++ to Java translator
\cite{CGJavaManPP}. 


\section{Overall Structure}

This section describes the {\bf design} of the specification, the {\bf
strategy} used in the Java to VDM++ translator. In addition to these
VDM modules there must a a module with a backend for VDM++ (converting
the abstract syntax three for VDM++ to concrete syntax inside RTF? files).

\subsection{Design}

The specification is divided into several modules:

\begin{description} 
\item[J2V: ] (J2V is an abbreviation of Java to VDM++ ).
  Providing functions/operations for translating top-level Java to VDM++
  definitions.

\item[J2VCLASS:] This module describes the Java to VDM++ translation of
the class structure including inheritance.

\item[J2VEXPR: ] (EXPR is an abbreviation of EXPRession). Providing
  functions/operations  for translating Java expressions corresponding
  to VDM++ statements/expressions.

\item[J2VSTMT: ] (STMT is an abbreviation of STateMenT). Providing
  functions/operations for translating Java statements corresponding
  to VDM++ statements.

\item[J2VTYPE:] Providing functions/operations for type specifiers from
  Java to type representations at the VDM++ level.

\item[J2VOP:] (OP is an abbreviation for OPerations). Provides
  functions/operations for translating Java operation definitions to
  equivalent VDM++ definitions.

\item[J2VENV:] (ENV is an abbreviation for ENVironment). Provides
  functions/operations for setting and looking up information about
  the environment in which a construct is being translated in the
  context of.�This includes things such as imports from packages.

\item[J2VNS:] (NS is an abbreviation for Name Spacing). Provides
  functions/operations for dealing with name space differences between
  Java and VDM++.

\item[CPP:] (CPP is an abbreviation of C Plus Plus). Describing the
  abstract syntax of Java and C++.

\item[CI:] (CI is an abbreviation fo Context Information). It is used
  to be able to refer back to errors in a concrete syntax such that
  the user can understand error messages.

\item[REP: ] Type Representations Module. Provides the type
  definitions for the semantic domains which are used internally in
  the static semantics to model the types from the abstract syntax and
  the derived types.

\item[J2VBVDM: ] (J2VBVDM is an abbreviation of Building VDM++). The construction
  of part of an abstract tree of VDM++ is described in this module. The
  main idea behind this module is divide the specification of the Java to
  VDM++ translator into parts of the same nature. This module also provides
  functions for naming temporary variables and auxiliary functions.

\item[AS: ] (AS is an abbreviation of Abstract Syntax). Describing the
  abstract syntax of the VDM-SL/VDM++.

\item[J2VAUX:] (AUXIL is an abbreviation of AUXILiary). Provides auxiliary
  functions and operations to the rest of the specification.

\end{description}

\section{Issues}

There are a number of issues where we have major design decisions to
make. Below we discuss the alternatives for each of the issues. This
information is to serve as basis for making decisions for each of
them.

\begin{description}
\item[static:] Java (and C++) has a static modifier which is currently not
present in VDM++. We can either introduce this at the VDM++ level or
we must treat all declarations made static in Java specially in the
translation process. Thus, in the translation we would have to make a
new instance of the given class before invoking a static constract.

Alternatively, the introduction of static at the VDM++ level would
require changes to the different Toolbox components.

\noindent{\textbf{Estimate for static class members in VDM++}}

\begin{tabular}{|l|l|p{5cm}|}\hline
Component    & Estimate & Comment \\ \hline
Parser       & 0.25d    & Add keyword (call syntax already exists -
                          field select expression)\\
Type checker & 0.5d     & No access to non-static class members, no
                          self expr, no object expressions.\\
Interpreter  & 3d       & Modify lookup mechanism, object initialization\\
CGs          & 0.25d    & Straightforward - generate extra keyword\\
Pretty printers
             & 0.25d    & \\
Rose Mapper  & 0.25d    & \\ 
Test coverage 
             & 0.25d    & \\
Documentation
             & 1d       & Must update language manual, Rose mapper
                          manual, CG manuals, examples. \\ \hline
\textbf{Total}
             & \textbf{5.75d} &\\ \hline
\end{tabular}

\item[method overloading:]
Java supports method (but not operator) overloading. This must be
resolved in the translation either by offering primitive method
overloading in VDM++, or by defining some name mangling scheme.

\noindent\textbf{Estimate for Method Overloading in VDM++}

\begin{tabular}{|l|l|p{5cm}|}\hline
Component    & Estimate & Comment \\ \hline
Parser       & 0.25d    & Currently signals problem if the same name
                          is used in several places.\\
Type checker & 1d       & Check a method exists for each call made;
                          check overloaded method signatures are
                          distinguishable. \\
Interpreter  & 4d       & Modify lookup mechanism: must take argument
                          types and then try to match against a method
                          name; possible performance cost due to
                          frequent type lookup.\\
Rose Mapper  & 0.5d     & Check that mapper doesn't assume uniqueness
                          of method names.\\ 
Documentation
             & 1d       & Must update language manual, CG
                          manuals, examples. \\ \hline
\textbf{Total}
             & \textbf{6.75d} &\\ \hline
\end{tabular}
\item[constructor/destructor:] In Java (and C++) constructors can be
defined for each class. In VDM++ this cannot be done in general (only
initialisation of instance variables are possible and nothing can be
transferred in the construction of a new instance of a class as
parameters. Destructors in Java are called finalisers and it must be
clarified how to deal with these.

\noindent\textbf{Estimate for Constructors in VDM++} assuming method
overloading is already in place.

\begin{tabular}{|l|l|p{5cm}|}\hline
Component    & Estimate & Comment \\ \hline
Parser       & 1d       & Update new expressions, special signature (no
                          return type) for constructors. \\
Type checker & 1d       & Check class exists; check a corresponding
                          constructor exists for new expressions.\\
Interpreter  & 2d       & Straightforward if method overloading is
                          already implemented.\\
CGs          & 0.25d    & Straightforward - accept arguments in new
                          expressions.\\
Pretty printers
             & 0.25d    & Straightforward - accept arguments in new
                          expressions.\\
Rose Mapper  & 1d       & Ensure that constructors are correctly
                          mapped (care needed since signature is
                          different).\\ 
Documentation
             & 1d       & Must update language manual, Rose mapper
                          manual, examples. \\ \hline
\textbf{Total}
             & \textbf{6.5d} &\\ \hline
\end{tabular}


\item[Packages:] In Java packages have semantics whereas it is a
purely syntactic part of VDM++ used in the management of projects and
their organisation into directories.

\noindent\textbf{Estimate for Packages in VDM++} 

\begin{tabular}{|l|l|p{5cm}|}\hline
Component    & Estimate & Comment \\ \hline
Parser       & 3d       & Devise format for package declaration and
                          import, update parser, add new modifier.\\ 
Type checker & 2d       & Add appropriate data structures for
                          packages; extend well-formedness functions
                          for names to search imported packages.\\
Interpreter  & 2d       & Modify look up mechanism; how much can be
                          statically decided?\\
CGs          & 2d       & Name spaces in VDM++?\\
Pretty printers
             & 1d       & \\
Rose Mapper  & 2d       & Use Rose packages?\\ 
Documentation
             & 2d       & \\ \hline
\textbf{Total}
             & \textbf{14d} &\\ \hline
\end{tabular}


\item[Exception handling:] In Java the exception handling mechanism is
more present from a type point of view than in VDM++. This is not
necessarily a problem but it needs to be clarified.

\item[Java Comments:] What do we do about Java comments? The current
position is that we can probably support some kind of comments made
before each definition (optionally naturally) with a special syntax
such that these comments can be carried over to the corresponding VDM
file before the VDM definition of the same construct. Currently this
is not supported by the Java parser which simply ignorres all
comments. It would also need to be taken into account in the VDM
backend translating the VDM++ AS representation to concrete syntax so
a kind of transfer format for this information must be defined. For
arbitrary comments in the middle of definitions in Java it is going to
be virtually impossible to position in an appropriate position at the
VDM++ level source files.

\item[Position information:] In the abstract syntax representation for
Java (the CPP module) position information is covered inside the
context information. This is neccesary in order to be able to report
errors to users at the right positions whenever something is wrong or
a limitation has been reached. However, presumably this information is
going to be irrelevant when we transfer to the VDM++ level. At least
it must be claried if it is realistic to keep track of where a
construct generated to VDM++ originally came from in the Java source
context.

\end{description}

\section{The VDM-SL Specification}

\include{j2v.vdm}

\include{j2vclass.vdm}

\include{j2vop.vdm}

\include{j2vtype.vdm}

\include{j2vstmt.vdm}

\include{j2vexpr.vdm}

\include{j2venv.vdm}

\include{j2vaux.vdm}

\include{j2vns.vdm}

\include{j2vbvdm.vdm}

\appendix
\bibliographystyle{iptes}
\bibliography{/home/peter/bib/dan}

\newpage
\addcontentsline{toc}{section}{Index}
\printindex

\end{document}



