% WHAT
%   The main file of the document describing the specification of 
%   the VDM++ part of the Code Generator Front-End.
%
% FILE
%    $Source: /home/vdmtools/cvsroot/toolbox/spec/cg-spec/vdm++cg.tex,v $
% VERSION
%    $Revision: 1.2 $
% DATE
%    $Date: 1999/05/03 08:43:47 $
% FORMAT
%    $State: Exp $
% PROJECT
%    Afrodite/IDERS
% STATUS
%    Under development.
% AUTHOR
%    $Author: jojo $
% COPYRIGHT
%    (C) 1993 IFAD, Denmark

\documentclass[a4paper,dvips]{article}
\usepackage[dvips]{color}
\usepackage{vdmsl-2e}
\usepackage{longtable}
\usepackage{alltt}
\usepackage{makeidx}
\usepackage{ifad}

\definecolor{covered}{rgb}{0,0,0}      %black
%\definecolor{not_covered}{gray}{0.5}   %gray for previewing
\definecolor{not_covered}{gray}{0.6}   %gray for printing
%\definecolor{not_covered}{rgb}{1,0,0}  %read


\newcommand{\StateDef}[1]{{\bf #1}}
\newcommand{\TypeDef}[1]{{\bf #1}}
\newcommand{\TypeOcc}[1]{{\it #1}}
\newcommand{\FuncDef}[1]{{\bf #1}}
\newcommand{\FuncOcc}[1]{#1}
\newcommand{\ModDef}[1]{{\tiny #1}}

\newcommand{\MCL}{Meta-IV Class Library }
\newcommand{\vinput}[1]{
\begin{verbatim}
\input #1
\end{verbatim}
}

\newcommand{\VDM}{VDM++}
\newcommand{\nfs}{{\em not fully specified\/}}
\newcommand{\TBW}{To be written}
\newcommand{\NYI}{Notice, that this function/operation is not fully
  specified in order to capture a possible integration with another
  library than \MCL.}

\makeindex
\parindent 0pt
\parskip 0pt

\begin{document}


\special{!userdict begin /bop-hook{gsave 220 30 translate
0 rotate /Times-Roman findfont 21 scalefont setfont
0 0 moveto (CONFIDENTIAL) show grestore}def end}

\docdef{Specification of the VDM++ Code Generator}
         {IFAD VDM Tool Group \\
          The Institute of Applied Computer Science}
         {\today,}
         {IFAD-VDM-29}
         {Report}
         {Under Development}
         {Confidential}
         {}
         {\copyright IFAD}
         {\item[V1.0] First version.}
         {}


\tableofcontents
\newpage


\section{Introduction}

This document contains the specification
of the VDM++ part of the code generator front-end. In addition
it describes the overall structure of the specification and
the used strategies. 

\section{Overall Structure}

The specification is divided into the following modules:

\begin{description}
\item[AS: ] (AS is an abbreviation of Abstract Syntax). Describing the
  abstract syntax of VDM++.

\item[CI: ] (CI is an abbreviation of Context Information). 
On every node in the Abstract Syntax Tree (AST) it is possible to
attach context information, as for instance position information and
type information.

\item[CPP:] (CPP is an abbreviation of C Plus Plus). Describing the
  abstract syntax of C++ and Java.

\item[CGMAIN:] (CGMAIN is an abbreviation of Code Generator MAIN). This module is
  to be considered as the main module, it generates code corresponding
  a whole specification.

\item[CLASS:] This module provides functions generating code
  corresponding to VDM++ class definitions.

\item[FVD:] This module provides functions generating code
  corresponding to function and value definitions inside VDM++ class
  definitions.

\item[VD: ] (VD is an abbreviation of Value Definitions). Providing
  functions generating code for value definitions.

\item[FD: ] (FD is an abbreviation of Function Definitions ).
  Providing functions generating code for function and operation
  definitions.

\item[TD: ] (TD is an abbreviation of Type Definitions). Providing
  functions generating code for invariants on type definitions.

\item[TPGEN: ] (TPGEN is an abbreviation of TyPe Generation). Providing
  functions generating code for type definitions.

\item[EXPR: ] (EXPR is an abbreviation of EXPRession). Providing
  functions generating code corresponding to expressions (specified
  for cases expressions, literals and name expressions).

\item[STMT: ] (STMT is an abbreviation of STateMenT). Providing
  functions generating code corresponding to statements (not
  specified yet ).

\item[PM: ] (PM is an abbreviation of Pattern Matching). Providing
  functions generating code of pattern match. 
  
\item[BC: ] (BC is an abbreviation of Building C++). The construction
  of part of an abstract tree of C++ is described in this module. The
  main idea behind this module is divide the specification of the code
  generator into parts of the same nature. This module also provides
  functions for naming temporary variables and auxiliary functions.

\item[DS: ] (DS is an abbreviation of Data Structures). Providing
  functions generating code corresponding to functions on VDM data
  structures, for example providing a function which generates code
  corresponding to a {\bf union} operation between two sets. That is,
  the basic data type implementation is described in this module.

\item[CGAUX:] (CGAUX is an abbreviation of CodeGenerator AUXiliary). Provides auxiliary
  functions and operations to the rest of the specification.

\item[REP: ] (Type Representations Module). Provides the type
  definitions for the semantic domains which are used internally in
  the static semantics to model the types from the abstract syntax and
  the derived types.

\item[POS: ] (Position Information Module).  Provides the representation of the position information that 
has been added to the abstract syntax, as defined in the AS module.

\item[TI:] This module provides operations keeping track of type
  information collected by the type checker. This type information is
  used to determine methods class membership and dependencies between
  classes.

\end{description}

%%%Specification.

%%module AS
\section{Module AS - VDM++ Abstract Syntax}
\input{common_as.vdmpp.tex}
\newpage

%%module CI
\input{contextinfo.vdm.tex}
\newpage

%%module CPP
\input{mod_cppast.vdm.tex}
\newpage

%%module CGMAIN
\input{mod_cgmain.vdmpp.tex}
\newpage

%%module CLASS
\input{mod_class.vdm.tex}
\newpage

%%module FVD
\input{mod_fvd.vdm.tex}
\newpage

%%module VD
\input{mod_valdef.vdmpp.tex}
\newpage

%%module FD
\input{mod_fctdef.vdmpp.tex}
\newpage

%%module TD
\input{mod_typedef.vdmpp.tex}
\newpage

%%module TPGEN
\input{mod_tpgen.vdmpp.tex}
\newpage

%module EXPR
\input{mod_expr.vdmpp.tex}
\newpage

%module STMT
\input{mod_stmt.vdmpp.tex}
\newpage

%module PM
\input{mod_patmat.vdm.tex}
\newpage

%module BC
\input{mod_bcppast.vdmpp.tex}
\newpage

%module DS
\input{mod_vdm_ds.vdmpp.tex}
\newpage

%module CGAUX
\input{mod_cgaux.vdmpp.tex}
\newpage

%module REP
\input{rep.vdmpp.tex}
\newpage

%module POS
\input{pos.vdmpp.tex}
\newpage

%module TI
\input{mod_ti.vdm.tex}
\newpage

\end{document}

%%module MD
%%\input{mod_md.vdm.tex}



\appendix

%\section{The Definition of the Abstract Syntax of VDM}
%\label{AS}
%\input{mod_as.vdm.tex}

%\label{AS}
%\input{rep.vdmpp.tex}

%\section{The Definition of the Abstract Syntax of C++}
%\input{mod_cppast.vdm.tex}





%%%module TI
%\input{mod_ti.vdm.tex}


%\newpage
%\addcontentsline{toc}{section}{Index}
%\printindex

\end{document}
