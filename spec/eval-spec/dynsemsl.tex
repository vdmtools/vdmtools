\documentclass[a4paper,dvips]{article}
\usepackage[dvips]{color}
\usepackage{float}
\usepackage{epsf}
\usepackage{vdmsl-2e}
\usepackage{longtable}
\usepackage{alltt}
\usepackage{makeidx}
\usepackage{ifad}
\usepackage{graphics}

\definecolor{covered}{rgb}{0,0,0}
\definecolor{not-covered}{rgb}{1,0,0}
\definecolor{not_covered}{rgb}{1,0,0}
%\definecolor{not_covered}{gray}{0.6}

\definecolor{covered}{rgb}{0,0,0}
\definecolor{not_covered}{gray}{0.6}


\floatstyle{plain}
\restylefloat{figure}

\newcommand{\StateDef}[1]{{\bf #1}}
\newcommand{\TypeDef}[1]{{\bf #1}}
\newcommand{\TypeOcc}[1]{{\it #1}}
\newcommand{\FuncDef}[1]{{\bf #1}}
\newcommand{\FuncOcc}[1]{{#1}}
\newcommand{\ModDef}[1]{{\tiny #1}}

\newcommand{\LAM}{{\restoreLambda$\lambda$}}

\newcommand{\id}[1]{\hbox{\it #1}}
\newcommand{\keyw}[1]{\hbox{\sf #1}}
\newcommand{\auth}[1]{{\sc #1}}

\newcommand{\emptychapter}[1]{%
 \chapter*{#1}
 \markboth{#1}{#1}
 \addcontentsline{toc}{chapter}{\protect\numberline{}{#1}}}



\newcommand{\evalstackone}[1]{%
\begin{tabular}{|l|}%
\mbox{} \\ \hline%
#1\\ \hline%
\end{tabular}}

\newcommand{\evalstacktwo}[2]{%
\begin{tabular}{|l|}%
\mbox{} \\ \hline%
\texttt{#1}\\ \hline%
\texttt{#2}\\ \hline%
\end{tabular}}

\newcommand{\fctstacktwo}[4]{%
\begin{tabular}{|l|l|}%
\mbox{} & \mbox{} \\ \hline%
\texttt{#1} & \texttt{#2}\\ \hline%
\texttt{#3} & \texttt{#4}\\ \hline%
\end{tabular}}



\newcommand{\RM}[1]{{\bf Remark:} {\em #1}}


\hyphenation{ex-cep-tion}
\hyphenation{ex-cep-tions}
\hyphenation{hand-ling}
\hyphenation{hand-ler}
\hyphenation{hand-lers}
\hyphenation{mo-du-le}
\hyphenation{mo-du-les}

\newenvironment{outdatedDocumentation}{

***** THE DOCUMENTATION BELOW IS OUTDATED *******

}
{

***** THE DOCUMENTATION ABOVE IS OUTDATED *******

}

\newenvironment{updatedSection}[2]{

******** The following has been updated on #1 by #2 ********

}
{

******** End of updated documentation ****************

}


\floatstyle{plain}
\restylefloat{figure}

\def\insertfig#1#2#3#4{ % Filename, epsfxsize, caption,  label
\begin{figure}[H]
\begin{center}
\leavevmode
\epsfxsize=#2
\epsffile{#1}
\end{center}
\caption{#3} #4 
\end{figure}
}

\newcommand{\evalstacktwo}[2]{%
\begin{tabular}{|l|}%
\mbox{} \\ \hline%
\texttt{#1}\\ \hline%
\texttt{#2}\\ \hline%
\end{tabular}}

\newcommand{\fctstacktwo}[4]{%
\begin{tabular}{|l|l|}%
\mbox{} & \mbox{} \\ \hline%
\texttt{#1} & \texttt{#2}\\ \hline%
\texttt{#3} & \texttt{#4}\\ \hline%
\end{tabular}}

\newcommand{\StateDef}[1]{{\bf #1}}
\newcommand{\TypeDef}[1]{{\bf #1}}
\newcommand{\TypeOcc}[1]{{\it #1}}
\newcommand{\FuncDef}[1]{{\bf #1}}
\newcommand{\FuncOcc}[1]{{#1}}
\newcommand{\ModDef}[1]{{\tiny #1}}

\newcommand{\LAM}{{\restoreLambda$\lambda$}}

\newcommand{\id}[1]{\hbox{\it #1}}
\newcommand{\keyw}[1]{\hbox{\sf #1}}
\newcommand{\auth}[1]{{\sc #1}}


\newcommand{\RM}[1]{{\bf Remark:} {\em #1}}


\newcommand{\emptychapter}[1]{%
 \chapter*{#1}
 \markboth{#1}{#1}
 \addcontentsline{toc}{chapter}{\protect\numberline{}{#1}}}

\hyphenation{ex-cep-tion}
\hyphenation{ex-cep-tions}
\hyphenation{hand-ling}
\hyphenation{hand-ler}
\hyphenation{hand-lers}
\hyphenation{mo-du-le}
\hyphenation{mo-du-les}
\newenvironment{outdatedDocumentation}{

***** THE DOCUMENTATION BELOW IS OUTDATED *******

}
{

***** THE DOCUMENTATION ABOVE IS OUTDATED *******

}

\newenvironment{updatedSection}[2]{

******** The following has been updated on #1 by #2 ********

}
{

******** End of updated documentation ****************

}


%\definecolor{covered}{rgb}{0,0,0}
\definecolor{not_covered}{gray}{0.6}


\floatstyle{plain}
\restylefloat{figure}

\newcommand{\StateDef}[1]{{\bf #1}}
\newcommand{\TypeDef}[1]{{\bf #1}}
\newcommand{\TypeOcc}[1]{{\it #1}}
\newcommand{\FuncDef}[1]{{\bf #1}}
\newcommand{\FuncOcc}[1]{{#1}}
\newcommand{\ModDef}[1]{{\tiny #1}}

\newcommand{\LAM}{{\restoreLambda$\lambda$}}

\newcommand{\id}[1]{\hbox{\it #1}}
\newcommand{\keyw}[1]{\hbox{\sf #1}}
\newcommand{\auth}[1]{{\sc #1}}

\newcommand{\emptychapter}[1]{%
 \chapter*{#1}
 \markboth{#1}{#1}
 \addcontentsline{toc}{chapter}{\protect\numberline{}{#1}}}



\newcommand{\evalstackone}[1]{%
\begin{tabular}{|l|}%
\mbox{} \\ \hline%
#1\\ \hline%
\end{tabular}}

\newcommand{\evalstacktwo}[2]{%
\begin{tabular}{|l|}%
\mbox{} \\ \hline%
\texttt{#1}\\ \hline%
\texttt{#2}\\ \hline%
\end{tabular}}

\newcommand{\fctstacktwo}[4]{%
\begin{tabular}{|l|l|}%
\mbox{} & \mbox{} \\ \hline%
\texttt{#1} & \texttt{#2}\\ \hline%
\texttt{#3} & \texttt{#4}\\ \hline%
\end{tabular}}



\newcommand{\RM}[1]{{\bf Remark:} {\em #1}}


\hyphenation{ex-cep-tion}
\hyphenation{ex-cep-tions}
\hyphenation{hand-ling}
\hyphenation{hand-ler}
\hyphenation{hand-lers}
\hyphenation{mo-du-le}
\hyphenation{mo-du-les}

\newenvironment{outdatedDocumentation}{

***** THE DOCUMENTATION BELOW IS OUTDATED *******

}
{

***** THE DOCUMENTATION ABOVE IS OUTDATED *******

}

\newenvironment{updatedSection}[2]{

******** The following has been updated on #1 by #2 ********

}
{

******** End of updated documentation ****************

}

\makeindex

\begin{document}
\bibliographystyle{newalpha}

\special{!userdict begin /bop-hook{gsave 220 30 translate
0 rotate /Times-Roman findfont 21 scalefont setfont
0 0 moveto (CONFIDENTIAL) show grestore}def end}

\docdef{The Dynamic Semantics of VDM-SL}
{ VDMTools Group}
{\today}
{VDM-SL 3.8.7}
{Report}
{Under Development}
{Confidential}
{}
{\copyright Kyushu University}
{\item[\mbox{}] \mbox{}}
{}


\pagenumbering{roman}
\addtocounter{tocdepth}{1}
\tableofcontents
\newpage
\pagenumbering{arabic}

\parskip12pt
\parindent0pt

%\section{Introduction}

\noindent \tbw.

\begin{enumerate}
\item Introducing VDM++ (move from start of Section~2?).
\item Explain the typical properties of distributed embedded
software development, use \cite{Lee05}. Typically SFM are used but
they are limited when complex algorithms come into play (for example
higher levels of control and error handling). Also introduce commonly
accepted RTE notions such as jitter, clock drift.
\item Aim of modeling is to increase the insight into the system
under construction at minimal cost.
\item Make minimal changes to both VDM++ syntax, static and
dynamic semantics. Keep the application specification as much
as possible free from (hardware) architecture specific knowledge.
\item Concrete ASCII syntax for case study in VDM++ and mathematical
syntax for semantics in VDM-SL.
\item Make clear to the reader the distinction between models at
the VDM++ level (application models?) and at the meta-level
(semantic models?). Align all sections accordingly.
\end{enumerate}

\subsection{Contribution of this paper}

\noindent \tbw.

\begin{enumerate}
\item Enabling (bridging the gap between) formal software engineering
and hardware / networking disciplines.
\item Properly deal with the notion of deployment on the 
computation and communication level.
\item Improvement of VDM++ technology to better support distributed
and embedded real-time systems. Ability to describe a whole new class
of systems.
\end{enumerate}

\subsection{Related work}

\noindent \tbw.

\begin{enumerate}
\item Certainly we have to reflect on TrueTime, Ptolemy and Giotto
and show why we are different.
\item Check the dan.bib file for more related work, just like
\cite{Huijsman&93} and \cite{Garlan&90b} -- product families.
\item Discussion notion of deployment in UML and AADL.
\end{enumerate}
                 % introduction

\chapter{Introduction}

The debugger for VDMTools is made using a stack machine approach. This
document contains the formal VDM-SL description of this tool
component. In order to better understand how to use this document to
discover how some feature is treated we start of by providing some
reading guidance.

Chapter~\ref{sec:comp} starts of with the general compilation of
constructs. This is followed by Section~\ref{sec:cexpr} to
Section~\ref{sec:cpat} which describes how expressions, statements and
patterns are compiled respectively. These sections are the place where
the main explanatory description of the strategy chosen for the
stack-machine based VDM interpreters. In order to fully understand the
explanation one naturally also need to understand the way the stack
machine itself is working and the instructions it is able to deal
with. The stack-machine itself is described in
Section~\ref{sec:stack}. The is followed by Section~\ref{sec:instr}
which describes how the different instructions are executed.

Section~\ref{ch:semvalues} contains the definitions for the semantics
values.

Finally Appendix~\ref{sec:as} is the common abstract syntax used for
all VDMTools for VDM.

\input{compile.vdmsl.tex}
\input{cexpr.vdmsl.tex}
\input{cstmt.vdmsl.tex}
\input{cpat.vdmsl.tex}

\input{stack-eval.vdmsl.tex}
\input{instructionsTP.vdmsl.tex}
\input{instructions.vdmsl.tex}

\input{eval_sem.vdmsl.tex}
\input{eval_global.vdmsl.tex}

\input{eval_state.vdmsl.tex}
\input{eval_mod.vdm.tex}
\input{eval_def.vdmsl.tex}
\input{eval_expr.vdmsl.tex}
\input{eval_pat.vdmsl.tex}
\input{eval_aux.vdmsl.tex}
\input{eval_free.vdmsl.tex}
\input{eval_old.vdm.tex}
\input{rterr.vdmsl.tex}
\input{debug.vdmsl.tex}
\input{eval_settings.vdmsl.tex}

\appendix
\input{common_as.vdmsl.tex}
\input{contextinfo.vdmsl.tex}

% end of evaluate.tex


\typeout{Bibliography}
\restoreLambda
\bibliography{string,dynsem}
\useLambda
\printindex
\addcontentsline{toc}{section}{Index}
\end{document}










