\documentclass[a4paper,10pt]{jsarticle}
%\documentclass[a4paper,10pt]{jsbook}

\usepackage[dvipdfm]{graphicx, color}
%\usepackage{folha}
\graphicspath{{image/}}

\usepackage{color}
\usepackage{array}
\usepackage{longtable}
\usepackage{alltt}
\usepackage{graphics}
\usepackage{vpp-nms}
%\usepackage{vpp}
\usepackage{makeidx}
\makeindex

\usepackage{colortbl}

\usepackage[dvipdfm,bookmarks=true,bookmarksnumbered=true,colorlinks,plainpages=true]{hyperref}

%\AtBeginDvi{\special{pdf:tounicode 90ms-RKSJ-UCS2}}
\AtBeginDvi{\special{pdf:tounicode EUC-UCS2}}


\definecolor{covered}{rgb}{0,0,0}      %black
\definecolor{not-covered}{rgb}{1,0,0}  %red

\setcounter{secnumdepth}{6}
\makeatletter
\renewcommand{\paragraph}{\@startsection{paragraph}{4}{\z@}%
  {1.5\Cvs \@plus.5\Cdp \@minus.2\Cdp}%
  {.5\Cvs \@plus.3\Cdp}%
  {\reset@font\normalsize\bfseries}}
\makeatother

\renewcommand{\sf}{\sffamily \color{blue}}

\newcommand{\syou}{\texttt{<}}
\newcommand{\dai}{\texttt{>}}

%\title{�z�e��}
\author{
�����L\\
�i���jCSK�V�X�e���Y\\
�ʐM�O���[�v\\
VDM�S��\\
}
\date{2006�N10��5��}

%\pagestyle{empty}
\usepackage{fancyhdr}
\usepackage{lastpage} 
  \pagestyle{fancy} 
   \let\origtitle\title 
  \renewcommand{\title}[1]{\lfoot{#1}\origtitle{#1}}

  \rfoot{\today}
  \rhead{[\ \scshape\oldstylenums{\thepage}\ / %
      \scshape\oldstylenums{\pageref{LastPage}}\ ]}
  \cfoot{}


\begin{document}

% the title page
\title{VDM++関数型ライブラリのドキュメント雛形}
\author{佐原 伸\\\\
(株)CSK\\
}
%\institute{\pgldk \and \chessnl}
\date{\mbox{}}
\maketitle

%\TaoReport{ガードコマンド・モデル}{\today}{タオベアーズ}{佐原伸}
%\setlength{\baselineskip}{12pt plus .1pt}
%\tolerance 10000
\tableofcontents
%\thispagestyle{empty} 

%\begin{abstract}
\setlength{\baselineskip}{12pt plus .1pt}
�Q�l����\cite{DJ2006}�t�^E�̃z�e�������̌��|���̗��̌��ׂ��C�����A���{�ꉻ���A���s�”\�d�l�Ƃ����B

\end{abstract}
%\vspace{-1cm}

\tableofcontents
\newpage

\section{���̊T�v}
�z�e���̋q�́A�`�F�b�N�C�����ɁA�t�����g�ŁA�����̌��ƂȂ�J�[�h���󂯎��B

�J�[�h��2�‚̌��������A���̌��͂��̌�ς�邱�Ƃ͂Ȃ��B

�t�����g�́A����܂łɔ��s�����J�[�h�ƁA�g�p�ς݂̃J�[�h���L�^���Ă���B

�q���A�J�[�h���g���ĕ����ɓ���ƁA�O�̋q�̃J�[�h�ł͕����ɓ���Ȃ��Ȃ�B

�q�́A�ēx�`�F�b�N�C�����邱�ƂŁA�V���Ȍ��ƂȂ�J�[�h���󂯎�邱�Ƃ��ł��邪�A�V�����J�[�h���g���ƁA�Â��J�[�h�ł͕����ɓ���Ȃ��Ȃ�B
\section {はじめに}
本ドキュメントは、VDM++関数型ライブラリ・ドキュメントの雛形であり、
まだ、すべてのVDMモジュールを記述しているわけではない。

\section {関数型ライブラリのドキュメント}
\include{AllT.vpp}
\include{FBusinessTable.vpp}
\include{FBusinessTableT.vpp}
\include{FCalendar.vpp}	
%\include{FCalendarT.vpp}
\include{FCharacter.vpp}
\include{FCharT.vpp	}
\include{FFunction.vpp}	
\include{FFunctionT.vpp}
\include{FHashtable.vpp}
\include{FHashtableT.vpp}	
\include{FInteger.vpp}
\include{FIntegerT.vpp}	
\include{FJapaneseCalendar.vpp}
\include{FJapaneseCalendarT.vpp}
\include{FMap.vpp}
\include{FMapT.vpp}	
\include{FNumber.vpp}
\include{FNumberT.vpp}	
%% LaTeX 2e Document.
% 
% $Id: Queue.tex,v 1.2 2006/01/10 10:46:35 vdmtools Exp $
% 

%%%%%%%%%%%%%%%%%%%%%%%%%%%%%%%%%%%%%%%%
% PDF compatibility code. 
\makeatletter
\newif\ifpdflatex@
\ifx\pdftexversion\@undefined
\pdflatex@false
%\message{Not using pdf}
\else
\pdflatex@true
%\message{Using pdf}
\fi

\newcommand{\latexorpdf}[2]{
  \ifpdflatex@ #2
  \else #1
  \fi
}

\newcommand{\pformat}{a4paper}

\makeatother
%%%%%%%%%%%%%%%%%%%%%%%%%%%%%%%%%%%%%%%%

%\latexorpdf{
%\documentclass[\pformat,12pt]{article}
%}{
%\documentclass[\pformat,pdftex,12pt]{article}
%}
\documentclass[]{jarticle}

\usepackage[dvips]{color}
\usepackage{array}
\usepackage{longtable}
\usepackage{alltt}
\usepackage{graphics}
\usepackage{vpp}
\usepackage{makeidx}
\makeindex

\definecolor{covered}{rgb}{0,0,0}      %black
%\definecolor{not-covered}{gray}{0.5}   %gray for previewing
%\definecolor{not-covered}{gray}{0.6}   %gray for printing
\definecolor{not-covered}{rgb}{1,0,0}  %red

\newcommand{\InstVarDef}[1]{{\bf #1}}
\newcommand{\TypeDef}[1]{{\bf #1}}
\newcommand{\TypeOcc}[1]{{\it #1}}
\newcommand{\FuncDef}[1]{{\bf #1}}
\newcommand{\FuncOcc}[1]{#1}
\newcommand{\MethodDef}[1]{{\bf #1}}
\newcommand{\MethodOcc}[1]{#1}
\newcommand{\ClassDef}[1]{{\sf #1}}
\newcommand{\ClassOcc}[1]{#1}
\newcommand{\ModDef}[1]{{\sf #1}}
\newcommand{\ModOcc}[1]{#1}

%\nolinenumbering
%\setindent{outer}{\parindent}
%\setindent{inner}{0.0em}

\title{FQueue���C�u�����[}
\author{
�����L
���{�t�B�b�c�������\\
���Z�p������\\
TEL : 03-3623-4683\\
shin.sahara@jfits.co.jp\\
}
%\date{2004�N2��25��}

\begin{document}
\setlength{\baselineskip}{12pt plus .1pt}
\tolerance 10000
\maketitle

\begin{abstract}
\setlength{\baselineskip}{12pt plus .1pt}
�҂��s��Ɋւ��֐���񋟂��郂�W���[���ł���B
\end{abstract}
%\vspace{-1cm}

\include{test/Queue.vpp}

\include{test/QueueT.vpp}

%\newpage
%\addcontentsline{toc}{section}{Index}
%\printindex

\end{document}

\include{FQueueT.vpp}
\include{FProduct.vpp}	
\include{FProductT.vpp}	
\include{FReal.vpp}
\include{FRealT.vpp}
\include{FSet.vpp}
\include{FSetT.vpp}	
\include{FString.vpp}
\include{FStringT.vpp}	
\include{FSequence.vpp}
\include{FSequenceT.vpp}
\include{FTestDriver.vpp}
\include{FTestLogger.vpp}

%\begin{thebibliography}{9}
\section{参考文献等}
VDM++\cite{CSK2007PP}は、
1970年代中頃にIBMウィーン研究所で開発されたVDM-SL\cite{CSK2007SL}を拡張し、
さらにオブジェクト指向拡張したオープンソース
\footnote{使用に際しては、(株)CSKシステムズとの契約締結が必要になる。}の形式仕様記述言語である。
\bibliographystyle{jplain}
%\bibliography{/Users/sahara/svnw/sahara}
\bibliography{/Users/sahara/bib/saharaUTF8}
%\bibliography{/Users/ssahara/svnwork/sahara}

%\end{thebibliography}

%\newpage
%\addcontentsline{toc}{section}{Index}
\printindex

\end{document}
