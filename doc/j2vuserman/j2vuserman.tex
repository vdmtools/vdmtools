% LaTeX 2e Document.
% 
% $Id: j2vuserman.tex,v 1.25 2006/04/19 10:56:48 vdmtools Exp $
% 

%%%%%%%%%%%%%%%%%%%%%%%%%%%%%%%%%%%%%%%%
% PDF compatibility code. 
\makeatletter
\newif\ifpdflatex@
\ifx\pdftexversion\@undefined
\pdflatex@false
%\message{Not using pdf}
\else
\pdflatex@true
%\message{Using pdf}
\fi

\newcommand{\latexorpdf}[2]{
  \ifpdflatex@ #2
  \else #1
  \fi
}

#ifdef A4Format
\newcommand{\pformat}{a4paper}
#endif A4Format
#ifdef LetterFormat
\newcommand{\pformat}{letterpaper}
#endif LetterFormat

\makeatother
%%%%%%%%%%%%%%%%%%%%%%%%%%%%%%%%%%%%%%%%

\latexorpdf{
\documentclass[\pformat,12pt]{article}
}{
\documentclass[\pformat,pdftex,12pt]{article}
}

\usepackage[dvipdfmx]{graphicx, color}
\usepackage[dvipdfm,bookmarks=true,bookmarksnumbered=true,colorlinks,plainpages=true]{hyperref}

\usepackage{toolbox}
\usepackage{vdmsl-2e}
\usepackage{alltt}
%\usepackage{graphics}
\usepackage{makeidx}
%\usepackage{palatino}
\usepackage{ifthen}
\usepackage{verbatim}

\usepackage{vpp}

\graphicspath{{figures/}}
\def\seename{$\Rightarrow$}

% Ueki change start
%\usepackage[dvipdfm,bookmarks=true,bookmarksnumbered=true,colorlinks,plainpages=true]{hyperref}
% Ueki change end

% Ueki delete start
%\latexorpdf{
%\usepackage[plainpages=true,colorlinks,linkcolor=black,citecolor=black,pagecolor=black, urlcolor=black]{hyperref}
%}{
%\usepackage[plainpages=true,colorlinks]{hyperref}
%}
% Ueki delete end

\makeindex

\def\vdmsl{{\small VDM-SL}}
\def\vdmpp{{\small VDM}++}
\newcommand{\vdmslpp}{VDM++}
\newcommand{\vdmslppEm}{VDM++}
\newcommand{\ToolboxName}{VDM++ Toolbox}
\newcommand{\Toolbox}{Toolbox}
\newcommand{\toolbox}{Toolbox}
\newcommand{\vdmde}{vppde}
\newcommand{\vdmgde}{vppgde}
\newcommand{\vdmhome}{vpphome}
\newcommand{\vdmdeNineteen}{vppde-19}
\newcommand{\vdmdeNineteenEl}{vppde-19.el}
\DeclareRobustCommand{\VdmSlPp}{VDM++-\VdmSl}
\newcommand{\vdmext}{vpp}
\newcommand{\vdmtoolsver}{v9.0.6}
\newcommand{\cg}{\vdmslpp\ to C++ Code Generator}

\newcommand{\JAVA}{Java}
\newcommand{\tjTov}{the \JAVA\ to VDM++ translator}
\newcommand{\TjTov}{The \JAVA\ to VDM++ Translator}
\newcommand{\VDM}{VDM++}
\newcommand{\jTov}{\JAVA\ to VDM++ Translator}



\newcommand{\subsubsubsection}[1]{\paragraph{#1}\mbox{}\\}


% The use of VDMSL/VDMPP ifdef's have basicly been exchanged with the
% use of LaTeX ifthenelse's.  For this two LaTeX boolean value VDMSL and
% VDMpp have been defined  (Lowercase p's are used to avoid conflict with
% the VDMPP environment variable.  The typical use are:
%   \ifthenelse{\boolean{VDMsl}}{vdmsl-text}{vdmpp-text}
%   \ifthenelse{\boolean{VDMsl}}{vdmsl-text}{}
%   \ifthenelse{\boolean{VDMpp}}{vdmpp-text}{}
% The advantage of this as opposed to ifdef's is that within a general
% paragraph specific VDM-SL and VDM++ parts can be distinguished without
% problematic empty lines.
% 
% The values are initialised such that exactly one of the values is true
% and the other is false.  This should hopefully avoid strange behaviour
% due to possible preprossing errors.  The default case is VDM-SL.
\newboolean{VDMsl}
\setboolean{VDMsl}{true}
\newboolean{VDMpp}
\setboolean{VDMpp}{false}
\setboolean{VDMpp}{true}
\setboolean{VDMsl}{false}

\newcommand{\AfterInit}[1]{}
%Initialisation must be performed before
%  {\tt #1} can be called.}

% This macro can be used in `description' lists where
% the item given to `meti' is put on its own line,
% thereby giving proper (nicer) identation to the
% explanation.
\newcommand{\meti}[1]{\item[#1]\mbox{}\\}

\newcommand{\Index}[1]{#1\index{#1}}

\newcommand{\Lit}[1]{`#1\Quote}
\newcommand{\Rule}[2]{
  \begin{quote}\begin{tabbing}
    #1\ \ \= = \ \ \= #2  ; %    Adds production rule to index
  \end{tabbing}\end{quote}
  }
\newcommand{\SeqPt}[1]{\{\ #1\ \}}
\newcommand{\lfeed}{\\ \> \>}
\newcommand{\OptPt}[1]{[\ #1\ ]}
\newcommand{\dsepl}{\ $|$\ }
\newcommand{\dsep}{\\ \> $|$ \>}
\newcommand{\Lop}[1]{\Lit{\kw{#1}}}
\newcommand{\Sig}[1]{\Lit{{\tt #1}}}
\newcommand{\blankline}{\vspace{\baselineskip}}
\newcommand{\Brack}[1]{(\ #1\ )}
\newcommand{\nmk}{\footnotemark}
\newcommand{\ntext}[1]{\footnotetext{{\bf Note: } #1}}

\usepackage[]{color}
\usepackage{longtable}
\usepackage{float}
\definecolor{covered}{rgb}{0,0,0}     %black
\definecolor{not_covered}{gray}{0.5}  %gray

%\restylefloat{figure}
\setcounter{topnumber}{3}
\def\topfraction{1.0}
\setcounter{bottomnumber}{3}
\def\bottomfraction{1.0}
\setcounter{totalnumber}{3}
\def\textfraction{.1}


\parindent0mm

\newlength{\keywwidth}

\newcommand{\xfigpicture}[4]{
\begin{figure}[hbt]
\setlength{\unitlength}{1mm}
\begin{center}
\mbox{
\begin{picture}(#1,#2)
\put(0,0){\special{psfile=#3 hscale=70 vscale=55}}
\end{picture} }
\end{center}
\caption{#4}
\end{figure}
}

%\newcommand{\qq}{\marginpar{\bf ???}}
\newcommand{\aaa}{\tt }
\newcommand{\cmd}{\tt }
\newcommand{\guicmd}[1]{{\sf #1}}
\newcommand{\keyw}[1]{{\sf #1}}
%\newcommand{\id}[1]{%
%  \settowidth{\keywwidth}{\tt #1}%
%  \protect\makebox[\keywwidth][l]{{\it #1}}}
%\nolinenumbering

\begin{document}
\vdmtoolsmanualcsk{The Java to \vdmslpp\ User Manual}
       {\vdmtoolsver}
       {2016}
       {\vdmslpp}
       {1.0}
 

\section{Introduction} \label{sec:introduction}

This manual gives an introduction to the Java to \vdmslpp\ feature
of \VDMTools. This feature can be used for reverse
engineering existing legacy Java applications to VDM++. At the VDM++
level different kinds of analysis may then be conducted and new
features specified and forward engineered.

The Java to \vdmslpp\ translator is an add-on feature to the
\ToolboxName{}. This manual is an extension to the {\em User Manual
  for the VDM++ Toolbox} \cite{UserManPP-CSK}. In general it is
intended that the standard \texttt{javac} should always be invoked on
a collection of Java files 
before they are included in a project with \VDMTools. If Java code
which cannot be accepted by \texttt{javac} is provided \VDMTools\ may
not behave correctly.

This manual starts by explaining how to include Java classes in a
project file in the \ToolboxName{}. This is followed by an overview of the
options for the Java to VDM++ translator, which include the ability to
apply a set of transformations to the generated \vdmslpp\ which
convert certain parts of it to equivalent but more abstract
forms. More details of this feature can be found in Section~\ref{xforms}.

Section~\ref{limitations} describes the different limitations for \tjTov{}.
This includes all the different situations which should be
avoided in order to automatically produce an equivalent VDM++ model
for a collection of Java source files.

Section~\ref{sec:translation}  gives specific details of the
translation process and also describes some of the design decisions
made when developing the \jTov{}, including the name conventions
used. This section should be studied intensively before using \tjTov\ 
professionally.

Finally, Section~\ref{javaapi} describes the subset of \JAVA\ API
which is available at the \VDM\ level. 

\section{Including Java Classes in a Project}

Before your Java source files can be automatically translated to VDM++
using the translator described in this manual you need to include all
the files you would like to have translated into a \VDMTools\
project. 

To do this, first start up the \ToolboxName{} (details of how to do
this can be found in the general User Manual for the VDM++
Toolbox \cite{UserManPP-CSK}), then press the 
\raisebox{-0.4mm}{\includegraphics[width=0.03\textwidth]{plus}}  
(\guicmd{Add Files}) button on the (\guicmd{Project Operations})
toolbar (or if you prefer you can select the action \guicmd{Add File
  to Project} on the \guicmd{Project} menu). The dialog box shown
in Figure~\ref{fig:addJavaFiles} will then appear. 

\begin{figure}[tbh]
\begin{center}
\mbox{}
\resizebox{11cm}{!}{\includegraphics{addJavaFiles}}
\caption{Adding Files to a Project\label{fig:addJavaFiles}}
\end{center}
\end{figure}

As an example, select the five {\tt .java} files in the 
{\tt \vdmhome/java2vdm/examples/sort} directory from the
\Toolbox\ distribution by holding down the {\cmd Ctrl} key and
clicking the left-hand mouse button on each of the 
files in turn. Then press the ``Open'' button. The files will then
be included in the project and will appear in the \guicmd{Project View}
of the \guicmd{Manager} in the main \Toolbox\ window as
shown in Figure~\ref{fig:addedJavaFiles}. You can also add a single
file to a project by double clicking the left-hand mouse button on it
(but note that this also closes the dialog box so it is not an
efficient way of adding a number of files), and you can also mark a
list of files at the same time by selecting the first and last files
in the list (in either order), holding down the {\cmd Shift} key while
making the second selection. 

\begin{figure}[tbh]
\begin{center}
\resizebox{8cm}{!}{\includegraphics{addedJavaFiles}}
\caption{Manager after Addition of Files}\label{fig:addedJavaFiles}
\end{center}
\end{figure}

It is important here to note that it is also necessary to include some
Java API skeletons in the project. These define functionality
equivalent to the standard Java API classes except that in general
they only include the signatures of methods because this information
is all that is required in order to perform the necessary checks on
your Java project files. These files are located in the {\tt
  \vdmhome/java2vdm/javaapi/java} directory, and the particular files
required for this application are shown in
Figure~\ref{fig:javaLibFiles}. Add these to the 
project in the same way.

\begin{figure}[tbh]
\begin{center}
\resizebox{8cm}{!}{\includegraphics{javaLibFiles}}
\caption{The Java API Skeletons Needed}\label{fig:javaLibFiles}
\end{center}
\end{figure}

Next you need to syntax check all your Java files, including the Java
API skeleton files. 

To do this, select all the files in the \guicmd{Project View} 
of the \guicmd{Manager}, then press the 
\raisebox{-0.7mm}{\includegraphics[width=0.03\textwidth]{syntaxcheck}}  
(\guicmd{Syntax Check}) button on the (\guicmd{Class Operations})
toolbar. (Selecting the containing level folders and
applying the syntax check operation to those  has the same effect --
this applies the syntax check operation to each of the  files in the folders.) 
Notice that at this point the \guicmd{Log Window} opens automatically
(if it is not already open) and displays a message informing you of
the success or failure of the check for each file. In addition, if
syntax errors are discovered the \guicmd{Error List} is
also au\-to\-matically invoked and the \guicmd{Source Window} is
automatically opened. See the general User Manual for the VDM++
Toolbox \cite{UserManPP-CSK} for information about the \guicmd{Error List}
and the \guicmd{Source Window} and about using the \guicmd{External
  Editor} to correct errors.

The next step is to ``type check'' your files, that is to check that
they only use the subset of Java 
that can be translated to VDM++ (see Section~\ref{limitations} for a
description of the current limitations). However, this is not
necessary for the Java API skeleton files because these do not need to
be translated and syntax checking gives enough context information to
allow you to check and translate the Java files from your own
application successfully. So just select the folder containing the
application files and invoke the type checker by pressing the 
\raisebox{-1.0mm}{\includegraphics[width=0.03\textwidth]{typecheck}}
(\guicmd{Type Check}) button on the (\guicmd{Class Operations})
toolbar. 

Note that after Java files have been successfully syntax checked the
names of the classes defined in those files are listed in the
\guicmd{Java View} in the \guicmd{Class View} of the
\guicmd{Manager}. You can select individual classes here to which you
want to apply Toolbox operations (i.e.\ instead of applying the
operations to all the classes in a file as is done in the
\guicmd{Project View}). You can also see the current status of each of
the individual Java classes in the project. 

Figure~\ref{fig:javaView} shows the current state of the \guicmd{Java
  View}. The symbol
\raisebox{-0.7mm}{\includegraphics[width=0.03\textwidth]{syntaxcheckdone}}
in the \guicmd{Syntax} column next to each class indicates that it has
  been syntax checked successfully, and the symbol
\raisebox{-1.0mm}{\includegraphics[width=0.03\textwidth]{typecheckdone}}
  in the \guicmd{Type} column next to the classes belonging to the
  example application indicate that these have also been successfully
  type checked. In case the syntax or type check was unsuccessful, the
  symbols 
\raisebox{-0.7mm}{\includegraphics[width=0.03\textwidth]{syntaxcheckerror}}
  respectively
\raisebox{-1.0mm}{\includegraphics[width=0.03\textwidth]{typecheckerror}}
are shown in the appropriate columns instead. In addition, if a source
  file is edited the symbol
\raisebox{-0.7mm}{\includegraphics[width=0.03\textwidth]{syntaxcheckmodified}}
  (this is the symbol 
\raisebox{-0.7mm}{\includegraphics[width=0.03\textwidth]{syntaxcheckdone}}
with a red triangle superimposed)
is shown in the \guicmd{Syntax} column to indicate that there is an
  inconsistency between the version of the file 
currently in the \Toolbox\ and the version on the file system. The
file must be syntax checked (and type checked) again before proceeding.

\begin{figure}[tbh]
\begin{center}
\resizebox{8cm}{!}{\includegraphics{javaView}}
\caption{The Java View}\label{fig:javaView}
\end{center}
\end{figure}

Once your files/classes have passed the syntax and type checks, they
are ready to be translated to VDM++. Select the five classes belonging
to the example application (you do not need to translate the Java API
skeleton classes but you will later need to load VDM++ equivalents of
these which are also supplied with the Toolbox), then press the
\raisebox{-0.8mm}{\includegraphics[width=0.03\textwidth]{java2vdm}}
button to translate them to VDM++. The symbol 
\raisebox{-1.0mm}{\includegraphics[width=0.03\textwidth]{java2vdmdone}}
appears in the \guicmd{VDM} column of the \guicmd{Java View} next to
each of the selected classes to indicate that it was translated
successfully~(see Figure~\ref{fig:translationDone}). In case of failure,
the symbol  
\raisebox{-1.0mm}{\includegraphics[width=0.03\textwidth]{java2vdmerror}}
is shown instead.

\begin{figure}[tbh]
\begin{center}
\resizebox{8cm}{!}{\includegraphics{translationDone}}
\caption{The Java View after Translation}\label{fig:translationDone}
\end{center}
\end{figure}

This generates five {\tt .vpp} files, one for each class. These can be
seen by returning to the \guicmd{Project View}~(see
Figure~\ref{fig:vdmFiles}). 

\begin{figure}[tbh]
\begin{center}
\resizebox{8cm}{!}{\includegraphics{vdmFiles}}
\caption{The Generated VDM++ Files}\label{fig:vdmFiles}
\end{center}
\end{figure}

You have now finished with the Java files and we recommend that you
remove them from the project -- select them all and press the 
\raisebox{-0.8mm}{\includegraphics[width=0.03\textwidth]{minus}}
(\guicmd{Remove Files}) button on the \guicmd{Project Operations}
toolbar.

The final step is to add the VDM++ files which correspond to the Java
API skeletons used with the Java files. These can be found in the {\tt
  \vdmhome/java2vdm/javaapi/vpp} directory, but since a lot of files are
required we recommend that to save time you add the contents of all
the appropriate subdirectories rather than the individual files. These
subdirectories are shown in Figure~\ref{fig:vdmApiFiles}. 

\begin{figure}[tbh]
\begin{center}
\resizebox{8cm}{!}{\includegraphics{vdmApiFiles}}
\caption{The VDM++ API Files}\label{fig:vdmApiFiles}
\end{center}
\end{figure}

In fact these files also contain only skeletons of the appropriate
classes in general, the main functionality being provided through the
dynamic link facility of the Toolbox (see \cite{DLMan-CSK} for
details). Thus the bodies of the methods in the skeleton classes
appear in general as \textsf{is not yet specified} and the
functionality is linked in as executable C++ code. In order to access
this code (so that, for example, you can use the interpreter to debug
your specification), you need to make sure the Toolbox knows where to
find it. In fact it is contained in the file {\tt
  \vdmhome/bin/j2vdll.so}, so you should either add this full 
path name to your global path or include the directory {\tt
  \vdmhome/java2vdm/javaapi} in the list of directories referenced by the
environment variable {\tt VDM\_DYNLIB}. (Alternatively you can copy
the file {\tt \vdmhome/bin/j2vdll.so} to the current directory.)

You now have a set of VDM++ files which correspond to your original
Java application and you can interact with these as with any other
VDM++ project. See the User Manual for the VDM++ Toolbox
\cite{UserManPP-CSK} for details.


\section{Options to the Translation}

The Java to VDM++ translator has two options which can be set in the 
\guicmd{Java to VDM++} panel of the \guicmd{Project Options} window, 
which is displayed by pressing the
\raisebox{-1.0mm}{\includegraphics[width=0.03\textwidth]{projectoptions}} 
(\guicmd{Project Options}) button on the (\guicmd{Project Operations})
toolbar. This is shown in Figure~\ref{fig:j2voptions}.

\begin{figure}[tbh]
\begin{center}
\resizebox{12.5cm}{!}{\includegraphics{j2vOptions}}
\caption{Setting Options for the Java to VDM++ Translator\label{fig:j2voptions}}
\end{center}
\end{figure}

The possible options are as follows:
\begin{description}
\item[Generate stubs only:]
If this option is selected it means that the translator will only
produce VDM++ code for class members and method signatures and will
use the ``is not yet specified'' construct for the VDM++ operation
bodies; 
% \item[Automatic renaming:]
% If this option is selected all occurrences of identifiers in Java
% which have the same names as VDM++ keywords will be automatically
% renamed by adding a suffix of two underscore symbols; 
% \item[Generate accessor functions:] This option is not yet implemented
% in the Java to VDM++ translator.
\item[Apply VDM++ transformations] If this option is selected the
  VDM++ which is generated by the translator is also passed through a
  series of transformations which convert certain parts of the
  specification to equivalent but more abstract forms. This feature is
  described in more detail in Section~\ref{xforms}. 
\end{description}

\section{Limitations}\label{limitations}

This section explains the current limitations on the Java to VDM++
translator and also includes recommendations on how your Java
source files should be modified before using the translator.

\subsection{Scope Differences between Java and VDM++}

The scope rules for Java and VDM++ are not identical. Thus, there are
a number of limitations for the Java to VDM++ translator which are
related to this difference. In general it is worth noting here that
the default modifier in Java is package. In VDM++ the default modifier
is private and no semantic package structure is present. Thus, when
the Java to VDM++ translator translates all default modifiers to
default modifiers in VDM++ there is a semantic difference. This gives
less visibility than in Java, but this reflects the differences in
scoping for the two languages. A number of limitations related
to scope issues are presented below.

\subsubsection{Class Names}

Class names in Java are considered as being local to packages, which
means that it is possible in a Java program to have two classes with
the same name provided they belong to different packages. In VDM++,
however,  the notion of packages is purely syntactic: class
names are effectively globally visible and there cannot be two classes
with the same name in the same project, even if they belong to
different packages. Thus, if two Java classes have the same name, one
of these must be renamed before translation to VDM++.

\subsubsection{Name Conflicts}

In Java there are less restrictions on the overloading of names than
in VDM++ so that, for example, in Java it is possible to have an instance
variable and a method with the same name in the same class  whereas in
VDM++ this causes an error.  

\begin{small}
\begin{verbatim}
class B
{
  public int b() { return 0 };
}
class A extends B
{ 
  b: int:=0;
}
\end{verbatim}
\end{small}

\begin{small}
\begin{verbatim}
class B

operations
  public b : ()  ==> int
  b() == ( return 0);

end B

class A is subclass of B

instance variables
  b: int:=0;        -- Error: "b" is multiple defined in super classes

end  A
\end{verbatim}
\end{small}

In general, the restriction in VDM++ is that there should never be two
different constructs with the same name visible in the same scope and
the Java should generally respect this. Of course the visibility of
constructs can be determined by the access modifiers so that it is
possible for a construct in a subclass to have the same name as a
private construct in a superclass as in the following example:

\begin{small}
\begin{verbatim}
class B
{
  private int b() { return 0 };
}
class A extends B
{ 
  b: int:=0;
}
\end{verbatim}
\end{small}

\subsubsection{The scope of a class member}

In Java, the fact that the default modifier is package means that
subclasses can reference declarations from superclasses if no explicit
modifiers are included as in the following example where
\texttt{super.i} references the declaration of i in class B: 

\begin{small}
\begin{verbatim}
class B
{
  int i=1;
}

class A extends B {
  int j=0;

  void a() { j = super.i; }
}
\end{verbatim}
\end{small}

In VDM++, however, declarations are by default assumed to be private,
so that the direct analogue of the above in VDM++:

\begin{small}
\begin{verbatim}
class B
instance variables 
  i : int  := 1;
  
end  B

class A is subclass of B
instance variables 
  j : int:=0;

operations

a: () ==> ()
a() == 
  j:= B`i;  -- Error: Access violation for "B`i"

end  A
\end{verbatim}
\end{small}

would yield a compile-time error stating that the instance variable B`i is
not in scope in class A. This can be fixed easily by adding a
protected or public modifier to the declaration of i in class B and
the translator currently issues a warning and asks the user to specify
these access modifiers explicitly\footnote{The translation could be
  modified to automatically generate public or protected modifiers by
  default in such situations.}.

\subsubsection{Access to instance variables of a subclass}

In Java selection of class attributes is determined at compile time,
whereas it is determined at run-time in VDM++. This means that there
is a slight difference in semantics between Java sources and
translated VDM++ descriptions in a few cases. This can be an issue when
class attributes are redefined by a subclass and instances of both the
superclass and the subclass are intermixed as in the following
example in which an instance of the subclass is assigned to a variable
representing the superclass:

\begin{small}
\begin{verbatim}
class S {int i=0;}
class T extends S {int i=1;}
class A {
  void a() {
    T t=new T();
    JavaLangSystem.out.println(t.i);  
    S s=new S();
    JavaLangSystem.out.println(s.i);  
    s=t;
    JavaLangSystem.out.println(s.i);  
  }
}
\end{verbatim}
\end{small}

The result of evaluating the above will be that the sequence of
numbers 1, 0, 0 will be printed out. Note that the the scope rules of
Java cause the last number to be 0 since the value of s is fixed at
compile time.

Compare this with the following VDM++ specification of the ``same''
example: 

\begin{small}
\begin{verbatim}
class S
instance variables
  public i:int:=0
end  S

class T is subclass of S
instance variables
  public i:int:=1
end  T

class A
operations
public Test: () ==> seq of int
Test() == (
  dcl t:T:=new T(),
      s:S:=new S(),
      res: seq of int:=[];
      res:=res^[t.i];
      res:=res^[s.i];
      s:=t;
      res:=res^[s.i];
  return res
)

end A
\end{verbatim}
\end{small}

Here the result of calling the Test operation is a sequence of 1, 0, 1
because in VDM++ the third value is determined at run-time rather
than at compile time as done in Java. Care should therefore be taken
to avoid such situations in the Java code\footnote{A future version
  of the type checker (applied statically to the Java source code)
  will detect this kind of problem and produce a warning, but this is 
not yet implemented.}.

\subsection{Restrictions on Statements with Side Effects}

In Java expressions can have side effects. For example, the expression
\texttt{i++} returns the value of \texttt{i} and as a side-effect also
increments \texttt{i} by 1. Such an expression therefore effectively
corresponds to a sequence of VDM++ statements. However, the syntax of
VDM++ does not allow sequences of statements to occur at arbitrary
positions within a specification, which imposes certain constraints on
the Java to VDM++ translator. We present these below.

\subsubsection{Initialization of class or instance variables}

In Java it is possible to use expressions which have side effects on
the right-hand side of initialisation expressions in a class
declaration as in the following example:

\begin{small}
\begin{verbatim}
class A {
int i=1;
int j=i++;
static int k=1;
static int l=k++;
}
\end{verbatim}
\end{small}

In the translator, however, we require that only expressions without
side effects can be used in initialisation expressions\footnote{In
  some special cases, including this example in fact, it would
  actually be possible to make a translation so it may be possible to
  relax this restriction in the future.}.

\subsubsection{Conditional Expressions on the Left-hand side of Assignments}

In Java it is possible to use a conditional expression on the
left-hand side of an assignment statement and this is supported by the
Java to VDM++ translator. However, in Java the alternatives in the
conditional statement can involve expressions with side effects as in
the following example:

\begin{small}
\begin{verbatim}
(i==0 ? a[i++] : b)[0] = ...
\end{verbatim}
\end{small}

These are not supported by the translator, which requires that it must
be possible to translate the alternatives to VDM++ expressions not
sequences of VDM++ statements\footnote{Again in some special cases it
  would be possible to make a translation so it may be possible to
  relax this restriction in the future.}.

\subsubsection{JavaLangObject Member Access}

JavaLangObject member access expressions which involve expressions with side
effects, for example:

\begin{small}
\begin{verbatim}
... a[i++].b[i++] ...
\end{verbatim}
\end{small}

also cannot in general be translated to VDM++ and are not currently
accepted by the translator in any form\footnote{Again simple forms
  could be translated so this restriction could possibly be relaxed in
  the future.}.

\subsection{Language Construct Differences}

In a number of areas the syntactic and semantic differences
between Java and VDM++ cause problems with translation. We discuss
these areas below.

\subsubsection{Java vs.\ VDM++ Keywords}

Some of the keywords in VDM++ are not keywords in Java (e.g.\ len,
value, is\_bool) which means they can be used as the names of classes,
instance variables, functions, etc.\ in Java. These should be renamed
before translation.

\subsubsection{Assignment to Function Parameters}

In Java, it is possible to assign a value to a parameter of a function
within the body of the same function, as, for example, in:

\begin{small}
\begin{verbatim}
int f(int i) { ...; i=1; ...; }
\end{verbatim}
\end{small}

In VDM++, this is not possible so functions which include such
assignments cannot be translated. 

\subsubsection{Using Assignment to new Instances as an Expression}

In Java the assignment statement

\begin{small}
\begin{verbatim}
a1 = new A();
\end{verbatim}
\end{small}

returns the value a1 which represents a new object of class A and this
assignment statement can therefore be used to interact directly with
that object as in the following example:

\begin{small}
\begin{verbatim}
(a1 = new A()).i = 1;
\end{verbatim}
\end{small}

% class A {public int i; public A a() {... return ...}}

In VDM++ this notation is not allowed (because the assignment does not
return a1 as a result so the field reference .i is not being applied
to an object). Expressions of this form therefore cannot be
translated. They should instead be written as a sequence of statements
in which the assignment is factored out, as in:

\begin{small}
\begin{verbatim}
a1 = new A();
a1.i = 1;
\end{verbatim}
\end{small}


\subsubsection{Numeric Types}

In Java, the numeric types `int', `long' and `real' are considered as
being distinguishable, so that, for example, with the following
definitions 

\begin{small}
\begin{verbatim}
class A
operations

public class A 
{
  int o(int i) {return 0;}
  int o(long i) {return 1;}
  int o(real i) {return 2;}
}
\end{verbatim}
\end{small}

the evaluation of the expression o(j) will yield 0 if j is of type
`int', 1 if j is of type `long', and 2 if j is of type `real'. In
VDM++, however, there is no type `long' (it is effectively equivalent
to `int') and the type `int' is a subtype of the type `real' so that
if j is of type `int' it is also of type `real'. In translating the
above example, therefore, the second function would override the first 
(because both `int' and `long' translate to `int') and would also
cause a conflict with the third when applied to integer arguments
(because the fact that `int' is a subtype of `real' means that the
third function can be applied to values of type `int' as well as
values of type `real'). Overloading functions in Java in such a way
that distinguishing between them relies on the distinctions between
numeric types should therefore be avoided.

\subsubsection{Type Conversions}

In Java, the type definition associated with a particular value can
cause a calculated value to change in order to conform to the declared
type, whereas in VDM++ such an inconsistency between the actual type
and the declared type would give rise to an error. For example, in the 
following Java code

\begin{small}
\begin{verbatim}
int j = 5;
int i = j/2;
\end{verbatim}
\end{small}

the value of i becomes 2 and not 2.5 because i is declared to be an
integer. 

In a similar way, Java also does explicit conversion of the actual
parameters of methods. For example, an operation `op' which is defined
to take a string as an argument can be called with a character `A' as
that parameter: the character `A' is in fact converted to the string
``A''. 

\begin{small}
\begin{verbatim}
... op(JavaLangString str) {...}
...

op('A')
\end{verbatim}
\end{small}

\TjTov\ translates the Java code precisely as written and does not
simulate such implicit type conversions, which means that the
specification generated would contain type errors (in the first
example a run-time type error would arise when trying to assign a real
value to an integer variable, whereas the second example would give a
static type error because a string should be a sequence of characters
and not just a single character). Implicit type conversions should
therefore not be used in the Java code.

\subsubsection{Inner Classes}

In Java it is possible to nest class definitions using inner classes
as in the following example: 

\begin{small}
\begin{verbatim}
class WithDeepNesting{
 boolean toBe;
 WithDeepNesting(boolean b) { toBe = b;}
 class Nested {
  boolean theQuestion;
  class DeeplyNested {
   DeeplyNested(){
    Nested.this.theQuestion
          = WithDeepNesting.this.toBe || !WithDeepNesting.this.toBe; }}}}
\end{verbatim}
\end{small}

This example also includes examples of the use of the qualified
\texttt{this}, which allows an inner class to reference definitions
belonging to one of its containing classes -- the keyword
\texttt{this} is prefixed with the name of the appropriate containing
class. 

VDM++ does not support inner classes, nor the qualified \texttt{this},
so neither of these is supported by the Java to VDM++
translator\footnote{It might be possible to support these
  at some stage in the future, for instance by converting the inner
  classes to normal classes and automatically renaming them and
  their attributes where appropriate, but this requires further
  investigation.}.


\subsubsection{Implicit use of Qualified \texttt{this}}

The \texttt{this} keyword can also be used implicitly. Semantically
the use of a class name in a field access expression is equivalent to
an implicit use of the \texttt{this} construct. For example an
expression of the form \texttt{ClassName.super.Identifier}
semantically means \texttt{((NameOfSuperClass)ClassName.this).Identifier}. This
implicit use of \texttt{this} is not supported by the
Java to VDM++ translator.

\subsubsection{Anonymous Classes}

Anonymous classes in Java have no counterpart in VDM++ and are not
supported by the Java to VDM++ translator\footnote{It would be fairly
  easy to support these if inner classes can be supported: the translator
  could simply allocate an arbitrary name to the anonymous class and
  then treat it in the same way as an inner class.}. 

\subsubsection{\texttt{label}, \texttt{break} and \texttt{continue}}

Using \texttt{label:}, \texttt{break label}, \texttt{continue
  label} and \texttt{continue} in Java effectively corresponds to
  using GOTO's -- the flow of control is interrupted and transferred
  to another point. This is not supported in VDM++ so these constructs
  cannot be translated.

The use of \texttt{break} alone (i.e.\ with no label) to leave a loop
is supported, however, though this is done using exception handling
which means that the VDM++ generated looks quite different from the
Java source code. We therefore recommend that \texttt{break} should
not be used to exit from loops.

\subsubsection{\texttt{switch} without Alternative Breaks}

In Java it is not necessary to put \texttt{break} between alternatives
in a \texttt{switch} statement, so that the following is valid:

\begin{small}
\begin{verbatim}
  oneOrTwo=0;
  switch(i) {
    case 1:
      secondAlternative=false;
    case 2:
      secondAlternative=true;
    break;
  }
\end{verbatim}
\end{small}

However, if \texttt{switch(1)} is evaluated \texttt{secondAlternative} will have
the value true -- since
there is no break between the cases the execution will continue to the
final \texttt{break}. This means that both \texttt{switch(1)} and \texttt{switch(2)}
have the same effect with respect to \texttt{secondAlternative}, namely
\texttt{secondAlternative=true} and, so the line
\texttt{secondAlternative=false} is entirely redundant and
we could achieve the same effect
as the above by writing the following instead:

\begin{small}
\begin{verbatim}
  oneOrTwo=0;

  switch(i) {
    case 1:
    case 2:
      secondAlternative=true;
    break;
  }
\end{verbatim}
\end{small}

In the translator we insist that this second form is used and we
reject switch statements in which there is some code between the case
alternatives but no \texttt{break} at the end of that code. Switch
statements in which the alternatives are separated by breaks are fully
supported, however, so that, for example, the following is allowed:

\begin{small}
\begin{verbatim}
  one=0;
  two=0;

  switch(i) {
    case 1:
      one=1;
    break;
    case 2:
      two=1;
    break;
  }
\end{verbatim}
\end{small}


\subsection{Unsupported Concepts}

In this subsection we present a few concepts where Java and VDM++ are
compatible but which have not yet been incorporated into the Java to
VDM++ translator. 

\subsubsection{Concurrency}

The concurrency concepts in Java with the wait and notify mechanisms
are similar to the notions for concurrency found in VDM++. However,
currently the Java to VDM++ translator does not provide support
for any of the concurrency features of Java including the
synchronized statement.

\subsubsection{Unicode Characters}

In Java it is possible to write arbitrary Unicode identifiers. This is
not yet supported by the Java parser built into \VDMTools.

\newpage
\section{Details of the \JAVA\ to \vdmslpp\ Translation}\label{sec:translation}

This section explains how individual elements of a \JAVA\ program are
translated to \vdmslpp.

\subsection{Built-in Types}\label{types}

The various kinds of integers in Java are all translated to the type
`int' in VDM++, and the various kinds of real numbers are all
translated to the type `real'. Characters and booleans are translated
to the types `char' and `bool' respectively. These transformations are
summarised in the table in Figure~\ref{fig:typexfs}.

\begin{figure}[htbp]
  \begin{center}
\begin{longtable}{|l|l|}
\hline
  \JAVA\   & \VDM\ \\ \hline \hline
  byte   & int \\ \hline
  short  & int \\ \hline
  int    & int \\ \hline
  long   & int \\ \hline
  float  & real \\ \hline
  double & real \\ \hline
  char  & char \\ \hline
  boolean  & bool \\ \hline
\end{longtable}    
    \caption{Transformations of Built-in Types}
    \label{fig:typexfs}
  \end{center}
\end{figure}

\subsection{Literals}\label{literals}

Literal values belonging to the boolean, character and numeric types
all have exact counterparts in VDM++ so are translated
verbatim. JavaLangString literals are translated to equivalent values
belonging to the VDM++ class JavaLangString since the JavaLangString class in
Java belongs to the API~(see Section~\ref{names}). Some
examples of the translation of literals are given in
the table in Figure~\ref{fig:litxfs}.  

\begin{figure}[htbp]
  \begin{center}
\begin{longtable}{|l|l|l|}
\hline
Type &  \JAVA\   & \VDM\ \\ \hline \hline
  boolean   & true, false  & true, false   \\ \hline
  char      & 'A', 'B', ...    & 'A', 'B', ...  \\ \hline
  long/int/short/byte  & 255, -12, ...  & 255, -12, ...    \\ \hline
  float/double  & 1.1, 1e-5, ...  & 1.1, 1e-5, ...  \\ \hline
  JavaLangString    & "abc..."  & new JavaLangString("abc...") \\ \hline
\end{longtable}    
    \caption{Transformations of Literals}
    \label{fig:litxfs}
  \end{center}
\end{figure}



\subsection{Names}\label{names}

Names in \JAVA\ are translated to the same names in \vdmslpp\ with the
following two exceptions:

\begin{enumerate}
\item where the name coincides with a \VDM\ keyword two underscores
  are appended to the name in \VDM. Thus, for example, a Java method
  named `bool' would become a method named `bool\_\_' in VDM++. 
\item where the name represents a class which is part of the \JAVA\
  API, the VDM++ name is prefixed by the (Java) package name in which
  the class is defined. Thus, for example, the Java class JavaLangObject
  becomes the class JavaLangObject in VDM++. 
\end{enumerate}

The reason for the second property is that when we translate a Java
class to VDM++ we add some extra functionality, for instance to
simulate the \texttt{null} value (see Section~\ref{null}). If we then
want to translate our VDM++ back to Java we must generate Java code
which corresponds to this added functionality because it is not
included in the Java API classes, and this would give us two Java
classes with the same name, the one already in the Java API and the
one generated by translating the extended version of this from VDM++
back to Java.  

\subsection{Arrays}\label{arrays}

An array in Java is translated to a map in VDM++, and accessing and
modifying values in an array are written in terms of map application
and map override. Some examples illustrating this are given in
the table in Figure~\ref{fig:arrayxfs}. 
\newpage
\begin{figure}[htbp]
  \begin{center}
\begin{longtable}{|l|l|}
\hline
  \JAVA\   & \VDM\ \\ \hline \hline
  \texttt{Type a[]}   & \texttt{map int to Type} \\ \hline
  \texttt{... = a[0]}   & \texttt{... := a(0)} \\ \hline
  \texttt{a[0]=...}   & \texttt{a:=a++\{0|->...\}} \\ \hline
\end{longtable}    
    \caption{Transformations of Arrays}
    \label{fig:arrayxfs}
  \end{center}
\end{figure}

Note that an array can be used as an instance of
\texttt{java.lang.JavaLangObject} in Java but if this is done the
VDM++ generated by the translator will contain a type error due to
mismatching of types. Arrays should therefore not be used in this way
in Java. 

\subsection{Classes}\label{class}

A \JAVA\ class is translated to a \VDM\ class of the same name, and
the various elements of the class are translated as explained in the
following subsections.

\subsubsection{Methods and Members}\label{methods}

Methods become operations while members become instance variables,
with static methods and members becoming static operations and 
instance variables. Methods whose result type is \texttt{void}
become operations with no result in VDM++. This is illustrated in
the following example:  

\begin{small}
\begin{verbatim}
class A                            class A is subclass of JavaLangObject
{                                  instance variables
  int i = 0;                         i: int := 0;
  static int s = 1;                  static s : int := 1;
                                   operations
  int method(int)                    method : int ==> int
    { return i; }                      ( return i );
  static int smethod(int)            static smethod : int ==> int
    { return i; }                      ( return i );
  void op()                          op : () ==> ()
    {...}                              (  ...  );
  static int op1 (int j)             static op1 : int ==> int
    {...}                              (  ...  );
}                                    end A
\end{verbatim}
\end{small}

Note that in \JAVA\ a variable which is not explicitly initialised has
a value by default whereas in \VDM\ it is undefined. This can lead to
run-time errors in VDM++ so care should be taken to ensure that
instance variables in Java are initialised correctly where appropriate.


\subsubsection{Inheritance}\label{inheritance}

In \JAVA{}, a class can inherit from other classes by extension or by
implementation. VDM++ only supports a single form of inheritance,
however, namely subclassing. Both forms of inheritance in Java are 
therefore translated to subclassing in VDM++. Thus, for example, the
inheritance clause \texttt{class A extends B implements C,D} becomes
\texttt{class A is subclass of B, C, D} in VDM++.

Java classes which have no explicit inheritance clause are assumed to
inherit implicitly from java.lang.JavaLangObject. The translator makes this
dependency explicit, so that in \VDM\ such a class explicitly inherits
from JavaLangObject, the VDM++ counterpart of
\texttt{java.lang.JavaLangObject}. The example in Section~\ref{methods}
illustrates this.

Note that the fact that both extension and implementation in Java are
translated to subclasses in VDM++ can cause a problem if the VDM++ is
translated naively back to Java since VDM++ subclasses are by default
translated to extension in Java. The manual for the VDM++ to Java Coce
Generator~\cite{CGJavaManPP-CSK} explains how to avoid this problem. 

\subsubsection{Modifiers of a Class}

Modifiers of a \JAVA\ class (e.g.\ abstract, final) have no
counterpart in VDM++ and are all ignored by the translator. 

\subsubsection{Access modifiers}

Access modifiers are translated directly from Java to VDM++ as shown
in the table in Figure~\ref{fig:accessxfs}. Note, however, that in Java the absence
of a modifier indicates that the construct can be accessed by any
class in the same package, whereas in VDM++ a construct with no access
modifier is assumed to be private. Omitting access modifiers in Java
can thus give rise to access violation errors when type checking the
translated VDM++. 

\begin{figure}[htbp]
  \begin{center}
\begin{longtable}{|l|l|}
\hline
  \JAVA\   & \VDM\ \\ \hline \hline
  no modifier         & no modifier        \\ \hline
  \texttt{public}     & \texttt{public}    \\ \hline
  \texttt{protected}  & \texttt{protected} \\ \hline
  \texttt{private}  & \texttt{private} \\ \hline
\end{longtable}    
    \caption{Transformations of Access Modifiers}
    \label{fig:accessxfs}
  \end{center}
\end{figure}


\subsubsection{Static initialisers}

Static initialisers are supported by the translator although they have
no direct counterpart in VDM++. To implement these, the Java code for
all static initialisers in a class is collected together and
translated into the body of a static operation called
\texttt{j2v\_staticInitializer}. This operation takes no inputs and
its result is the special VDM++ quote value \texttt{<VOID>}. In
addition, a static instance variable called \texttt{dummy} is added to
the VDM++ class, the type of which is this same quote value
\texttt{<VOID>}, and the operation \texttt{j2v\_staticInitializer} is
defined as the initialisation of this variable. Thus, when the VDM++
class is initialised, the operation \texttt{j2v\_staticInitializer} is
invoked and simulates the effects of the original static initialisers
in the Java class. 

The following example illustrates this.

\begin{small}
\begin{verbatim}
                      instance variables
static int i=1;         static i : int  := 1;
static int j=2;         static j : int  := 2;
static int k=3;         static k : int  := 3;

static {
  i=k+j;
}

int l;                  l : int ;
                        
static {                static dummy : <VOID> := j2v_staticInitializer()
  j=i+k;              
}                     operations
                        static j2v_staticInitializer : () ==> <VOID>
                          j2v_staticInitializer() ==
                          ( i := k+j;
                            ( j := i+k
                            ) ;
                            return  <VOID>
                          ) ;
\end{verbatim}
\end{small}



\subsubsection{Getting information about a class}\label{getclass}

In \JAVA\ every class inherits from \texttt{java.lang.JavaLangObject}.
This includes the method \texttt{getClass} which returns an instance
of \texttt{java.lang.Class}, and this in turn provides functionality
for obtaining various information about the class, e.g.\ the class
name, whether the class represents an interface type or a primitive
Java type, etc. 

The translator simulates part of this functionality using the class
\texttt{JavaLangClass} which is defined as follows:

\begin{small}
\begin{verbatim}

class JavaLangClass ...
types
  CLASS ::
    name : seq of char
    cori : <CLASS> | <INTERFACE>
    isPrim : bool

instance variables
  private val : JavaLangClass`CLASS;

operations 

public  JavaLangClass :  seq1 of char * 
             (<CLASS> | <INTERFACE>) * bool ==> JavaLangClass
JavaLangClass(name, cori, prim) == 
( val.name := name;
  val.isPrim := prim;
  val.cori := cori;
);

public  getName : () ==> JavaLangString
getName() == 
  return new JavaLangString(val.name);

public  isArray : () ==> bool 
isArray() == 
  return false;

public  toString : () ==> JavaLangString
toString() == 
( dcl
    str: seq of char :=
      if isInterface()
      then "interface "
      else 
        if isPrimitive()
        then ""
        else "class ";
  str:=str^getName().toSeqOfChar();
 return new JavaLang(str)
);

public  isInterface : () ==> bool 
isInterface() == 
  return val.cori=<INTERFACE>;

public  isPrimitive : () ==> bool 
isPrimitive() == 
  return val.isPrim;

...

end JavaLangClass

\end{verbatim}
\end{small}

A constant \texttt{CLASS}, whose value is an appropriate instance of
this class, together with a (public) method \texttt{getClass} which
returns the value of this constant, are then added to each generated
\VDM\ class. This is illustrated by the example below. 

\begin{small}
\begin{verbatim}
class A ...
values 
  CLASS : JavaLangClass = new  JavaLangClass("A", <CLASS>, false)
...

public  getClass : () ==> JavaLangClass
getClass() == 
return  CLASS;

end A
\end{verbatim}
\end{small}

\subsection{Interfaces}\label{interface}

Interfaces in Java are translated to classes in VDM++. However, only
the signatures of the methods in a Java interface are given, while in
VDM++ a method must always have a body. The translator therefore sets
the bodies of methods in Java interfaces to \texttt{is not yet
  specified} in VDM++. 

It might seem more natural to use \texttt{is subclass responsibility}
  here instead of \texttt{is not yet specified}. However, this does
  not work because of the way the value \texttt{null} is
  translated~(see Section~\ref{null}). To see this, consider the
  following Java example in which the value \texttt{null} is used in
  conjunction with objects of an interface type:

\begin{small}
\begin{verbatim}
interface IFace {...}
...
IFace iface = null;
\end{verbatim}
\end{small}

In this case the translator would translate \texttt{null} to
\texttt{new IFace(<NIL>)} as explained in 
Section~\ref{null}, which includes an instantiation of the class
\texttt{IFace}. This means that the class \texttt{IFace} cannot be
abstract (otherwise this instantiation does not make sense), and thus
that we cannot use \texttt{is subclass responsibility} to represent
the bodies of interface methods.

Recall also that translating an interface class from Java to VDM++ and
then re-translating the VDM++ back to Java can cause problems. See the
discussion at the end of Section~\ref{inheritance} for details.


\subsection{\texttt{null}}\label{null}

In Java it is possible to overload operations and invoke a particular
one of these operations with the (polymorphic) value \texttt{null} by
casting the value to the appropriate input type as in the examples
below: 

\begin{small}
\begin{verbatim}
void op(A a)  
 
void op(B b)   

op((A)null)

op((B)null)
\end{verbatim}
\end{small}

In VDM++ there is no polymorphic constant which belongs to all types
and which can be coerced to a specific class by tagging it with the
name of that class. (The closest is perhaps the value \texttt{nil},
which belongs to all optional types, though this cannot be coerced to
any one type.)

Instead, therefore, we make every translated VDM++ class a
subclass of a special class \texttt{Nullable} (by defining the class
\texttt{JavaLangObject} to be a subclass of the class
\texttt{Nullable}). The (boolean valued) instance variable
\texttt{isNil} in the class \texttt{Nullable} is then used to indicate
whether or not a particular object belonging to the class corresponds
to the Java value \texttt{null}. 

\begin{small}
\begin{verbatim}
class JavaLangObject is subclass of Nullable
...

class Nullable
instance variables
  public isNil: bool := false
operations
  public IsNil: () ==> bool
  IsNil() == return isNil
end Nullable
\end{verbatim}
\end{small}

In addition, we introduce a new special value \texttt{<NIL>}
into the VDM++ specification and generate an additional constructor
for each \VDM\ class which creates instances which represent the
\texttt{<NIL>} value of that class. This is shown in the following
example: 

\begin{small}
\begin{verbatim}
ClassName: <NIL> ==> ClassName
ClassName() ==
  isNil := true;
\end{verbatim}
\end{small}


To simulate the \texttt{null} value for arrays we use the empty map. 

The examples in the table in Figure~\ref{fig:nullxfs} illustrate how
\texttt{null} is translated in different contexts.

\begin{figure}[htbp]
  \begin{center}
\begin{longtable}{|l|p{7cm}|}
\hline
  \JAVA\   & \VDM\ \\ \hline \hline
  \texttt{ClassType o = null;}      & \texttt{dcl o: ClassType := new ClassType(<NIL>)}    \\ \hline
  \texttt{ClassType o[2] = {null};} & \texttt{dcl o: map int to ClassType}
                                      \texttt{:= \{new ClassType(<NIL>) | i in set \{0,...,1\}\}}    \\ \hline
  \texttt{type o[] = null;}         & \texttt{o := \{|->\}}    \\ \hline
  \texttt{op(null);}                & \texttt{op(new ClassType(<NIL>))}    \\ \hline
  \texttt{o == null}                & \texttt{o.IsNil()}    \\ \hline
  \texttt{o != null}                & \texttt{not o.IsNil()}    \\ \hline
\end{longtable}    
    \caption{Transformations of \texttt{null}}
    \label{fig:nullxfs}
  \end{center}
\end{figure}

It is important to note here that this treatment of \texttt{null} can
lead to endless loops at the VDM++ level. Consider the following
example in Java:

\begin{small}
\begin{verbatim}
class Database

{
    Database db = null ;
}
\end{verbatim}
\end{small}

This would be translated to the following in VDM++:

\begin{small}
\begin{verbatim}
class Database is subclass of JavaLangObject

instance variables
    db : Database = new Database(<NIL>) ;
\end{verbatim}
\end{small}

which contains an infinite loop -- each instantiation of the class
Database invokes another instantiation in order to set the value of
its instance variable. The same problem can also occur indirectly as
in the following example where the \texttt{Database} class has an
instance variable of type \texttt{Controller} and vice versa:

\begin{small}
\begin{verbatim}
JAVA:

class Controller
{
  Database database;
  Controller()
  {
          database = null;
  }
  static public Controller getController()
  {
          return new Controller();
  }
}

class Database

{
        Controller
                controller = Controller.getController();
}

VDM:

class Controller is ...
instance variables
        database: Database;
operations
        Controller: () ==> Controller
                database := new Database(<NIL>);

        static public getController: () ==> ()
        getController() ==
                return new Controller(<NIL>);
end Controller

class Database is ...
instance variables
        controller: Controller := Controller`getController();
end Database
\end{verbatim}
\end{small}

Care should therefore be taken to avoid such situations in the Java
code. 

\subsection{Expressions}\label{expressions}

In \JAVA\ it is possible to write an expression with side effects such
as \texttt{f(n) - mm[n++]}. This feature has no direct counterpart
in VDM++, and in general a sequence of VDM++ statements is required in
order to obtain something semantically equivalent. We deal with this
by introducing a \texttt{def} expression and generating local names
within this to store the values of the various components of the
expression. This is best illustrated by an example.

Consider the following fragment of an expression which has side
effects in Java:

\begin{small}
\begin{verbatim}
  ... f(n)-mm[n++] ...
\end{verbatim}
\end{small}

This expression is part of an enclosing expression, so in the VDM++ we
introduce a \texttt{def} expression and the pattern name \texttt{l\_1}
to record its value. 

The expression itself is a binary expression, so two more pattern
names \texttt{l\_2} and \texttt{l\_3} are introduced to store the
values of its left and right operands respectively and \texttt{l\_1}
is defined appropriately in terms of \texttt{l\_2} and \texttt{l\_3}. 

Note that \texttt{l\_2} is only needed because
the right operand has side effects. We can not define \texttt{l\_1}
simply as \texttt{f(n) - l\_3} because its definition is preceeded
by the statements generated for the right operand which change \texttt{n}.

The final result is as follows:

\begin{small}
\begin{verbatim}
  def l_2 = f(n);
  l_4 = n;
  l_5 = l_4
  in ( n := l_4 + 1;
    def l_3 = mm(l_5);
    l_1 = l_2 - l_3
    in ... l_1 ...
  ) ;

\end{verbatim}
\end{small}


Expressions like \texttt{new ClassType(...);} and \texttt{op();} can
also be used as statements in Java, and it is possible to write
sequences of such expressions as in the following example:

\begin{small}
\begin{verbatim}
new A();
op1();
op2();
\end{verbatim}
\end{small}

In Java, if an expression in such a sequence returns a result that
result is ignored and the execution passes to the next expression in
the sequence, but the semantics of VDM++ do not match this and instead
state that as soon as the execution encounters an expression that
returns a result the execution terminates and that result becomes the
result of the whole sequence. When translating such expressions,
therefore, we need to avoid this problem. This is done by using a
\texttt{let} statement to assign the result to a dummy (in fact
unnamed) variable and making the body of the \texttt{let} statement
the identity statement \texttt{skip}. The translation of the Java
example above is thus:

\begin{small}
\begin{verbatim}
let - = new  A()
in 
  skip ;
let - = op1()
in 
  skip ;
let - = op2()
in 
  skip ;
\end{verbatim}
\end{small}

\subsection{Statements}

\subsubsection{If statements}

The translation of \texttt{if} statements is straightforward. The
component parts are simply translated as illustrated below:

\begin{small}
\begin{verbatim}
JAVA:

if condition
then thenStmt
else elseStmt

VDM:

if translatedCondition
then translatedThenStmt
else translatedElseStmt
\end{verbatim}
\end{small}

\subsubsection{Block statement}

A block in Java becomes a block statement in VDM++. The Java local
variable declarations become \texttt{dcl} statements within the block
and the body of the block in Java becomes the body of the block
statement in VDM++. The following example illustrates this:

\begin{small}
\begin{verbatim}
JAVA:

{ int b, c; int d, e=0; ... }

VDM:

( dcl b : int ,
      c : int ,
      d : int ,
      e : int  := 0;
  ...
) ;
\end{verbatim}
\end{small}

\subsubsection{\texttt{for} statement}

A \texttt{for} statement in Java is translated to a block statement in
VDM++ in which the loop variables are introduced in \texttt{dcl}
statements and the body is a while loop. This is illustrated by the
following example: 

\begin{small}
\begin{verbatim}
JAVA:

for(int i=0, j=10; i<10; i++, j--) {
  ...
}

VDM:

( dcl   i : int  := 0,
        j : int  := 10;
  while  i < 10 do  
  ( ...
    i := i+1;
    j := j-1
  ) 
) ;
\end{verbatim}
\end{small}

\subsubsection{\texttt{while} statement}

The translation of while loops is straightforward except where the
loop termination expression has side effects in which case it is
treated in the way described in Section~\ref{expressions} both before
the loop is entered and at the end of each iteration. This is
illustrated in the following example:

\begin{small}
\begin{verbatim}
JAVA:

while(mm[i++]>0) {
  ...
}

VDM:

( dcl   l_7 : bool ;
  def l_5 = i;
  l_4 = l_5
  in ( i := l_5+1;
    def l_2 = mm(l_4);
    l_1 = l_2 > 0
    in l_7 := l_1
  ) ;
  while  l_7 do  
  ( ...
    def l_5 = i;
    l_4 = l_5
    in ( i := l_5+1;
      def l_2 = mm(l_4);
      l_1 = l_2 > 0
      in l_7 := l_1
    ) 
  ) 
) ;
\end{verbatim}
\end{small}

\subsubsection{\texttt{do while} statement}

To translate a \texttt{do while} statement of the form 

\begin{small}
\begin{verbatim}
do {
  i++; ... 
} while(mm[i]>0);
\end{verbatim}
\end{small}

we convert it to the equivalent \texttt{while} statement 

\begin{small}
\begin{verbatim}
( i := i+1; ...
) ;
while  mm(i) > 0 do  
( i := i+1; ...
) ;
\end{verbatim}
\end{small}

and translate this \texttt{while} statement.

\subsubsection{\texttt{switch} statement}

A \texttt{break} in Java is translated to an exception throw in VDM++
because it can appear anywhere inside an alternative. The following
example, in which there is a \texttt{break} in \texttt{case 1:},
illustrates this.

\begin{small}
\begin{verbatim}
JAVA:

switch(a) {
  case 0:
    alternative1
  break;
  case 1:
    if(b<5) break;
    alternative2
  break;
  default:
    alternative3
  break;
}

VDM:

trap  j2v_break with  
  cases  true  : 
    (isofclass  (J2V_Break,j2v_break)) -> 
      skip 
    
  end  
in  
  ( cases  a: 
      (0) -> 
        ( translatedAlternative1
          exit  new  J2V_Break()
        ) ,
      (1) -> 
        ( if b<5 then exit  new  J2V_Break();
          translatedAlternative2
          exit  new  J2V_Break()
        ) ,
      others  -> 
        ( translatedAlternative3
          exit  new  J2V_Break()
        ) 
    end  
  ) ;
\end{verbatim}
\end{small}


\subsubsection{\texttt{try catch} statement}

A Java \texttt{try catch} statement is translated to a combination of
an \texttt{always} statement and a \texttt{trap} statement as
illustrated in the following example:

\begin{small}
\begin{verbatim}
JAVA:

class B extends JavaLangException { ... }

try { tryBody }
catch(B b) { catchBbody }
catch(JavaLangException e) { catchEbody }
finally{ finallyBody }

VDM:

class B is subclass of JavaLangException
...
end B

always  
  ( translatedFinallyBody 
  ) 
in  
  trap j2v_exception : JavaLangException with  
    cases true  : 
      (isofclass(B, j2v_exception) and not  j2v_exception.IsNil()) -> 
        ( dcl b : B := j2v_exception;
          translatedCatchBbody 
        ) ,
      (isofclass(JavaLangException, j2v_exception) or 
                                           j2v_exception.IsNil()) -> 
        ( dcl b : JavaLangException := j2v_exception;
          translatedCatchEbody 
        ) ,
      others  -> 
        exit  j2v_exception
    end  
  in  
    translatedTryBody
\end{verbatim}
\end{small}

\newpage
\section{VDM++ Transformations}
\label{xforms}

The transformations which are currently supported apply to certain
kinds of expressions (binary expressions) and certain kinds of
statements (if statements, block statements, and while loop
statements). The specific transformations are listed in
Table~\ref{fig:xfs}, the first column of which shows the 
name of the transformation and the second the kind of expression or
statement to which it applies. A brief explanation and an example of
each transformation is given below.

\begin{figure}[htbp]
  \begin{center}
\begin{tabular}{|l|l|}
\hline 
Name & Applies To\\ \hline \hline
isMapCompLoop & Block Stmt  \\ \hline
ifTestTrue & Block Stmt  \\ \hline
ifTestFalse & Block Stmt  \\ \hline
isRedundantIfBlock & Block Stmt \\ \hline
ifToAnd & Block Stmt  \\ \hline
isRedundantIfBlockNoElse & Block Stmt \\ \hline
isRedundantDcl & Block Stmt \\ \hline
ifToCases & If Stmt \\ \hline
ifToEquiv & If Stmt \\ \hline
nestedIfsNoElses & If Stmt  \\ \hline
whileIfTestTrue &  While Loop \\ \hline
%isRenamingDefStmt & Define Stmt \\ \hline
%isRedundantDef & Define Expr \\ \hline
%isRenamingDef & Define Expr \\ \hline
orToNotEquiv & Binary Expr \\ \hline
\end{tabular}    
    \caption{List of Transformations}
    \label{fig:xfs}
  \end{center}
\end{figure}

The transformations are attempted in the order in which they appear in
the table, so in situations in which two different transformations
could in principle be applied to the same construct only the first of
these is actually applied. In addition, the transformations are only
applied once. This means that there could still be places in the VDM++
output where transformations could be applied, for example when one
transformation transforms the specification into something which
matches another transformation. 

\subsection{isMapCompLoop (Block Stmt)} 
  Replaces stepwise construction of a map using a while loop with a
  map comprehension expression. Can deal with increasing and
  decreasing loop variables, inclusion or exclusion of equality in the
  test of the while loop, and arbitrary step sizes for the loop
  variable. Also allows arbitrary specification to appear after the
  loop. 

As an example, the following specification involving a while loop

\begin{verbatim}
  public Test : () ==> map nat to [nat]
  Test () ==
    ( dcl m : map nat to [nat] := 
                  {l |-> nil | l in set {0,...,12}};
      dcl b : bool := true;
        (dcl  i : int := 0;
         while i < 12 do
           ( m := m ++ {i |-> f(i)};
             i := i + 1 )
        );
      return m);
\end{verbatim}

would transform to this specification involving a map comprehension

\begin{verbatim}
public Test : () ==> map nat to [nat]
  Test () ==
    ( dcl m : map nat to [nat] := 
                  { l |-> nil | l in set {0,...,12} },
          b : bool := true;
      ( dcl i : int := 0;
        m := m ++ { i |-> f(i) | 
                         i in set {0, ..., 12 - 1} }
      );
      return m);
\end{verbatim}


\subsection{ifTestTrue (Block Stmt)}
  Transforms a block which begins

\begin{verbatim}
  ( dcl ... , x : bool := true, ... ; 
    if x then y else z;
    ...
  )
\end{verbatim}

into one which begins

\begin{verbatim}
  ( dcl ... , x : bool := true, ... ; 
    y;
    ...
  )
\end{verbatim}

For example, the following specification

\begin{verbatim}
  public Test : () ==> nat  
  Test () ==
    def n : nat = 5 in
        ( dcl x : bool := true;
          if x then return n else return 0 );
\end{verbatim}

is transformed to

\begin{verbatim}
  public Test : () ==> nat  
  Test () ==
    def n : nat = 5 in
        ( dcl x : bool := true;
          return n );
\end{verbatim}

\subsection{ifTestFalse (Block Stmt)}

  Transforms a block which begins

\begin{verbatim}
  ( dcl ... , x : bool := false, ... ; 
    if x then y else z;
    ...
  )
\end{verbatim}

into one which begins

\begin{verbatim}
  ( dcl ... , x : bool := false, ... ; 
    z;
    ...
  )
\end{verbatim}

For example, the following specification

\begin{verbatim}
  public Test : () ==> nat  
  Test () ==
    def n : nat = 5 in
        ( dcl x : bool := false;
          if x then return n else return 0 );
\end{verbatim}

is transformed to

\begin{verbatim}
  public Test : () ==> nat  
  Test () ==
    def n : nat = 5 in
        ( dcl x : bool := false;
          return 0 );
\end{verbatim}


\subsection{isRedundantIfBlock (Block Stmt)}
  Replaces a block statement of the form 

\begin{verbatim}
  ( dcl ...,  b : bool  := false, ... ;
    if  test 
    then  
      ( b := true  
      ) 
    else  
      ( b := false  
      ) ;
    return  b
  ) ;
\end{verbatim}
  
with 

\begin{verbatim}
  return test;
\end{verbatim}

The block can be inside another block and the boolean type can be an
indirect reference as in the following example

\begin{verbatim}
  types T = bool;

  operations

    public Test : () ==> bool  
    Test() ==
      ( dcl b : seq of char;
        ( dcl b : bool := false;
          ( dcl b : T := true;
            dcl c : bool := b;
            if c
            then (b := true)
            else (b := false);
            ((return b;))
          );
          return b
        );
      );
\end{verbatim}

which transforms to

\begin{verbatim}
  types T = bool;

  operations

    public Test : () ==> bool  
    Test() ==
      ( dcl b : seq of char;
         ( dcl b : bool := false;
            ( dcl b : T := true,
                  c : bool := b;
              return c
            );
           return b 
         ); 
      );
\end{verbatim}


\subsection{isRedundantIfBlockNoElse (Block Stmt)}
  Analogous to the above transform except that it applies to the case
  when the if statement has no else clause. Thus, for example, the
  following specification

\begin{verbatim}
  public Test : () ==> bool  
  Test () ==
    ( dcl a : bool;
        ( dcl b : bool := false;
            ( dcl c : bool := false;
              if (2 > 1)
              then (b := true)
            );          
          return b
        );
    );
\end{verbatim}

is transformed to

\begin{verbatim}
  public Test : () ==> bool  
  Test () ==
    ( dcl a : bool;
        ( dcl b : bool := false;
            ( dcl c : bool := false;
              return (2 > 1)
            );
        );
    );
\end{verbatim}

\subsection{ifToAnd (Block Stmt)}
  Transforms a consecutive pair of statements of the form

\begin{verbatim}
  if a then return b;
  return false;
\end{verbatim}

to the single statement

\begin{verbatim}
  return a and b;
\end{verbatim}

Redundant blocks are also removed as in the following example

\begin{verbatim}
  public Test : () ==> bool
  Test () ==
    ( dcl x : nat := 12;
        ( if x < 3 then return x > 3);
        (((return false)))
                
    );
\end{verbatim}

which is transformed to

\begin{verbatim}
  public Test : () ==> bool
  Test () ==
    ( dcl x : nat := 12;
      return x < 3 and  x > 3 
    ); 
\end{verbatim}

\subsection{isRedundantDcl (Block Stmt)}
  Replaces a block statement of the form

\begin{verbatim}
   ( dcl x : A := y;
     return x
   );
\end{verbatim}

with the return statement

\begin{verbatim}
  return y;
\end{verbatim}

Thus, for example, the following specification

\begin{verbatim}
  types

    public T = (nat * nat)

  operations

    public Test : () ==> T
    Test () ==
      (dcl r : T := def mk_(a,2) = mk_(1,2) in mk_(a,2);
       return r);
\end{verbatim}

is transformed to

\begin{verbatim}
  types

    public T = (nat * nat);

  operations

    public Test : () ==> T
    Test() ==
      return def mk_(a,2) = mk_(1,2) in mk_(a,2);
\end{verbatim}

\subsection{ifToCases (If Stmt)}
  Transforms a multiply nested if statement of the form
\begin{verbatim}
  if x = a1
  then r1
  else if x = a2
       then r2
       else if x = r3
            then r3
            else ....
\end{verbatim}

to a cases statement of the form
\begin{verbatim}
  cases x :
    a1 -> r1,
    a2 -> r2,
    a3 -> r3,
    ...
\end{verbatim}
At the same time, redundant blocks are removed. Thus, for example, the operation
\begin{verbatim}
  public Test : () ==> nat  
  Test () ==
    ( dcl x : nat := 5;
      if x = 1 
      then return 1
      else if (x = 2) 
           then ((return 2))
           else if (((x = 3))) 
                then return (3) 
                else if x = 4 
                     then (return 4)
                     elseif (x = 5)
                     then return 5; 
      return x);
\end{verbatim}
is transformed to
\begin{verbatim}
  public Test : () ==> nat
  Test() == 
    ( dcl x : nat := 5;
      cases x :
        1 -> return 1,
        2 -> return 2,
        3 -> return 3,
        4 -> return 4,
        5 -> return 5
      end;
    return x);
\end{verbatim}


\subsection{ifToEquiv (If Stmt)}
  Transforms an if statement of the form 
\begin{verbatim}
  if a then (if b then x else y) else (if b then y else x) 
\end{verbatim}    
to the form
\begin{verbatim} 
   if a <=> b then x else y
\end{verbatim}

As an example, the following operation
\begin{verbatim}
  public Test : nat * nat ==> nat  
  Test (x, y) ==
    ( if x > 2 
      then ( if y < 1 then return 1 else return 2 )
      else ( if y < 1 then return 2 else return 1 ) 
    );
\end{verbatim}
would be transformed to
\begin{verbatim}
  public Test : nat * nat ==> nat  
  Test (x, y) ==
    ( if x > 2 <=> y < 1 
      then return 1 
      else return 2 
    );
\end{verbatim}

\subsection{nestedIfsNoElses (If Stmt)}
  Transforms a double if statement of the form
\begin{verbatim}
  if a then (if b then c)
\end{verbatim}
to a single if statement
\begin{verbatim}
  if a and b then c
\end{verbatim}

Thus, for example, the operation
\begin{verbatim}
  public Test : int * int ==> bool
  Test (x, y) ==
    ( for i in [1, 2, 3] do
        ( if x > i then if i > y then return false );
      return true;
    );
\end{verbatim}
would be transformed to
\begin{verbatim}
  public Test : int * int ==> bool
  Test (x, y) ==
    ( for i in [1, 2, 3] do
        ( if x > i and i > y then return false );
      return true;
    );
\end{verbatim}

\subsection{whileIfTestTrue (While Loop)}
  Simplifies an if statement which occurs at the beginning of a while
  loop and which has the same test as the while loop. In general, the
  transformation converts a while loop of the form
\begin{verbatim}
   while test do  
      ( if test then x else y; ... ) 
\end{verbatim}
to one of the form
\begin{verbatim}
   while test do  
      ( x; ... ) 
\end{verbatim}

For example, the following operation
\begin{verbatim}
  public Test : nat * int ==> nat
  Test (x, y) ==
    ( while x > 5 do
        ( if x > 5 then y := y + 1 else y := y - 1; 
          x := x - 1 );
      return y;
    );
\end{verbatim}
is transformed to
\begin{verbatim}
  public Test : nat * int ==> nat
  Test (x, y) ==
    ( while x > 5 do
        ( y := y + 1; 
          x := x - 1 );
      return y;
    );

\end{verbatim}

%\subsection{isRenamingDefStmt (Define Stmt)}

% \subsection{isRedundantDef (Define Expr)}
%   Transforms a define expression of the form
% \begin{verbatim}
%   def x : A = y in x
% \end{verbatim}
% in which x is a simple name, to y. Thus, for example, the operation 
% \begin{verbatim}
%   public Test : () ==> real  
%   Test () ==
%     return ( def x : real = 
%       ( 12 - 7 ) in x ) + ( def y : real = 0 in y )
% \end{verbatim}
% transforms to
% \begin{verbatim}
%   public Test : () ==> real  
%   Test () ==
%     return ( 12 - 7 ) + 0 
% \end{verbatim}

%\subsection{isRenamingDef (Define Expr)}


\subsection{orToNotEquiv (Binary Expr)}
  Transforms an or expression of the form
\begin{verbatim}
  not a and b or a and not b
\end{verbatim}
into the logically equivalent form
\begin{verbatim}
  not (a <=> b)
\end{verbatim}
At the same time, redundant brackets are removed. Thus, for example,
the operation
\begin{verbatim}
  public Test : () ==> bool
  Test () ==
    ( return (((not (let a = true in a) and (false)) or 
               (((((let a = true in a) and not (false)))))))
    );
\end{verbatim}
is transformed to
\begin{verbatim}
  public Test : () ==> bool
  Test () ==
    return (not ((let a = true in a) <=> false));
\end{verbatim}

\newpage
\section{\VDM\ model of \JAVA\ API classes}\label{javaapi}

\TjTov\ is supplied with \texttt{*-.java} files containing skeletons of a subset of 
\JAVA\ API classes. They might need to be in your project.
At least you need \texttt{java.lang.JavaLangObject} and all classes it references.

There is no need to translate them to \VDM\ because \tjTov\ is
also equipped with their \VDM\ counterparts. 
A \JAVA\ API class is represented at the \VDM\ level as a
plain \VDM\ class or a dynamically linkable class (\texttt{dlclass}).
Implementation of dlclasses is housed in \texttt{j2vdll.so}.

\JAVA\ API is only partially covered at the \VDM\ level. 
Some methods could not be translated into \VDM\
because they use \JAVA\ features that are not available in \VDM{}.
For example, the method \texttt{JavaUtilVector.copyInto} has an array
as an output parameter which is impossible in \VDM\ because an array
is a map and a map is passed by value. There are also \JAVA\ API
classes which have overloaded methods which cannot be translated
because their signatures will coincide (e.g.\
\texttt{JavaLangString.valueOf(int) and
  JavaLangString.valueOf(long)}). 

The following is the list of \VDM\ classes with their methods
which implement a subset of \JAVA\ API.
This documents gives only the list.
Actual documentation can be found in the \JAVA\ documentation.

\subsection{\texttt{java.lang}}

\subsubsection{\texttt{JavaLangArrayIndexOutOfBoundsException}}
\begin{small}
\begin{verbatim}
public  JavaLangArrayIndexOutOfBoundsException : 
          () ==> JavaLangArrayIndexOutOfBoundsException
public  JavaLangArrayIndexOutOfBoundsException : 
          int  ==> JavaLangArrayIndexOutOfBoundsException
public  JavaLangArrayIndexOutOfBoundsException : 
          <NIL> ==> JavaLangArrayIndexOutOfBoundsException
public  JavaLangArrayIndexOutOfBoundsException : 
          JavaLangString ==> JavaLangArrayIndexOutOfBoundsException
\end{verbatim}
\end{small}

\subsubsection{\texttt{JavaLangBoolean}}
\begin{small}
\begin{verbatim}
public  static TRUE : JavaLangBoolean := new JavaLangBoolean(true);
public  static FALSE : JavaLangBoolean := new JavaLangBoolean(false);
public  static TYPE : JavaLangClass := 
          new JavaLangClass("boolean", <CLASS>, true);
public  JavaLangBoolean : bool  ==> JavaLangBoolean
public  hashCode : () ==> int
public  toString : () ==> JavaLangString
public  JavaLangBoolean : <NIL> ==> JavaLangBoolean
public  booleanValue : () ==> bool
public  equals : JavaLangObject ==> bool
public  JavaLangBoolean : JavaLangString ==> JavaLangBoolean
public  static valueOf : JavaLangString ==> JavaLangBoolean
public  static getBoolean : JavaLangString ==> bool
\end{verbatim}
\end{small}

\subsubsection{\texttt{JavaLangCharacter}}
\begin{small}
\begin{verbatim}
public static MIN_RADIX : int := 2;
public static MAX_RADIX : int := 36;
public static MIN_VALUE : char:= '0';
public static MAX_VALUE : char:= 'f';
public  static getType : char  ==> int
public  static isSpace : char  ==> bool
public  static isDefined : char  ==> bool
public  static forDigit : int  * int  ==> char
public  static isLowerCase : char  ==> bool
public  static isTitleCase : char  ==> bool
public  static isUpperCase : char  ==> bool
public  static toLowerCase : char  ==> char
public  static toTitleCase : char  ==> char
public  static toUpperCase : char  ==> char
public  static isISOControl : char  ==> bool
public  static isJavaLetter : char  ==> bool
public  JavaLangCharacter : <NIL> ==> JavaLangCharacter
public  static getNumericValue : char  ==> int
public  static isJavaLetterOrDigit : char  ==> bool
public  static isJavaIdentifierPart : char  ==> bool
public  static isIdentifierIgnorable : char  ==> bool
public  static isJavaIdentifierStart : char  ==> bool
public  static isUnicodeIdentifierPart : char  ==> bool
public  static isUnicodeIdentifierStart : char  ==> bool
\end{verbatim}
\end{small}

\subsubsection{\texttt{JavaLangClass}}
\begin{small}
\begin{verbatim}
public  JavaLangClass : <NIL> ==> JavaLangClass
public  JavaLangClass :  seq1 of char * 
            (<CLASS> | <INTERFACE>) * bool ==> JavaLangClass
public  getName : () ==> JavaLangString
public  isArray : () ==> bool
public  toString : () ==> JavaLangString
public  getClasses : () ==> map  int  to  JavaLangClass
public  getSigners : () ==> map  int  to  JavaLangObject
public  isInterface : () ==> bool
public  isPrimitive : () ==> bool
public  newInstance : () ==> JavaLangObject
public  getInterfaces : () ==> map  int  to  JavaLangClass
public  static forName : JavaLangString ==> JavaLangClass
\end{verbatim}
\end{small}

\subsubsection{\texttt{JavaLangClassCastException}}
\begin{small}
\begin{verbatim}
public  JavaLangClassCastException : 
          () ==> JavaLangClassCastException
public  JavaLangClassCastException : 
          <NIL> ==> JavaLangClassCastException
public  JavaLangClassCastException : 
          JavaLangString ==> JavaLangClassCastException
\end{verbatim}
\end{small}

\subsubsection{\texttt{JavaLangClassNotFoundException}}
\begin{small}
\begin{verbatim}
public  getException : () ==> JavaLangThrowable
public  printStackTrace : () ==> ()
public  JavaLangClassNotFoundException : 
          () ==> JavaLangClassNotFoundException
public  JavaLangClassNotFoundException : 
          <NIL> ==> JavaLangClassNotFoundException
public  printStackTrace : JavaIoPrintStream ==> ()
public  printStackTrace : JavaIoPrintWriter ==> ()
public  JavaLangClassNotFoundException : 
          JavaLangString ==> JavaLangClassNotFoundException
public  JavaLangClassNotFoundException : JavaLangString * 
              JavaLangThrowable ==> JavaLangClassNotFoundException
\end{verbatim}
\end{small}

\subsubsection{\texttt{JavaLangComparable}}
\begin{small}
\begin{verbatim}
public  compareTo : JavaLangObject ==> int

\end{verbatim}
\end{small}

\subsubsection{\texttt{JavaLangConversionBufferFullException}}
\begin{small}
\begin{verbatim}
public  JavaLangConversionBufferFullException : 
          () ==> JavaLangConversionBufferFullException
public  JavaLangConversionBufferFullException : 
          <NIL> ==> JavaLangConversionBufferFullException
public  JavaLangConversionBufferFullException : 
          JavaLangString ==> JavaLangConversionBufferFullException
\end{verbatim}
\end{small}

\subsubsection{\texttt{JavaLangDouble}}
\begin{small}
\begin{verbatim}
public  static POSITIVE_INFINITY : real :=0;
public  static NEGATIVE_INFINITY : real :=0;
public  static NaN : real :=0;
public  static MAX_VALUE : real := 1.79769313486231570e+308;
public  static MIN_VALUE : real := 4.94065645841246544e-324;
public  static TYPE : JavaLangClass := 
          new JavaLangClass("double", <CLASS>, true);
public  isNaN : () ==> bool
public  static isNaN : real  ==> bool
public  JavaLangDouble : real  ==> JavaLangDouble
public  hashCode : () ==> int
public  intValue : () ==> int
public  toString : () ==> JavaLangString
public  byteValue : () ==> int
public  longValue : () ==> int
public  static toString : real  ==> JavaLangString
public  floatValue : () ==> real
public  isInfinite : () ==> bool
public  shortValue : () ==> int
public  JavaLangDouble : <NIL> ==> JavaLangDouble
public  doubleValue : () ==> real
public  static isInfinite : real  ==> bool
public  JavaLangDouble : JavaLangString ==> JavaLangDouble
public  equals : JavaLangObject ==> bool
public  static valueOf : JavaLangString ==> JavaLangDouble
public  compareTo : JavaLangObject ==> int
public  static doubleToLongBits : real  ==> int
public  static longBitsToDouble : int  ==> real
public  static parseDouble : JavaLangString ==> real
public  static doubleToRawLongBits : real  ==> int
\end{verbatim}
\end{small}

\subsubsection{\texttt{JavaLangException}}
\begin{small}
\begin{verbatim}
public  JavaLangException : () ==> JavaLangException
public  JavaLangException : <NIL> ==> JavaLangException
public  JavaLangException : JavaLangString ==> JavaLangException
\end{verbatim}
\end{small}

\subsubsection{\texttt{JavaLangIllegalAccessException}}
\begin{small}
\begin{verbatim}
public  JavaLangIllegalAccessException : 
          () ==> JavaLangIllegalAccessException
public  JavaLangIllegalAccessException : 
          <NIL> ==> JavaLangIllegalAccessException
public  JavaLangIllegalAccessException : 
          JavaLangString ==> JavaLangIllegalAccessException
\end{verbatim}
\end{small}

\subsubsection{\texttt{JavaLangIllegalArgumentException}}
\begin{small}
\begin{verbatim}
public  JavaLangIllegalArgumentException : 
          () ==> JavaLangIllegalArgumentException
public  JavaLangIllegalArgumentException : 
          <NIL> ==> JavaLangIllegalArgumentException
public  JavaLangIllegalArgumentException : 
          JavaLangString ==> JavaLangIllegalArgumentException
\end{verbatim}
\end{small}

\subsubsection{\texttt{JavaLangIllegalStateException}}
\begin{small}
\begin{verbatim}
public  JavaLangIllegalStateException : 
          () ==> JavaLangIllegalStateException
public  JavaLangIllegalStateException : 
          <NIL> ==> JavaLangIllegalStateException
public  JavaLangIllegalStateException : 
          JavaLangString ==> JavaLangIllegalStateException
\end{verbatim}
\end{small}

\subsubsection{\texttt{JavaLangIndexOutOfBoundsException}}
\begin{small}
\begin{verbatim}
public  JavaLangIndexOutOfBoundsException : 
          () ==> JavaLangIndexOutOfBoundsException
public  JavaLangIndexOutOfBoundsException : 
          <NIL> ==> JavaLangIndexOutOfBoundsException
public  JavaLangIndexOutOfBoundsException : 
          JavaLangString ==> JavaLangIndexOutOfBoundsException
\end{verbatim}
\end{small}

\subsubsection{\texttt{JavaLangInstantiationException}}
\begin{small}
\begin{verbatim}
public  JavaLangInstantiationException : 
          () ==> JavaLangInstantiationException
public  JavaLangInstantiationException : 
          <NIL> ==> JavaLangInstantiationException
public  JavaLangInstantiationException : 
          JavaLangString ==> JavaLangInstantiationException
\end{verbatim}
\end{small}

\subsubsection{\texttt{JavaLangInteger}}
\begin{small}
\begin{verbatim}
public  CLASS : JavaLangClass = 
          new JavaLangClass("JavaLangInteger", <CLASS>, false);
public  TYPE : JavaLangClass =
          new JavaLangClass("int", <CLASS>, true);
public  MIN_VALUE : int=-2147483648;
public  digits : map  int  to  char
static  public MAX_VALUE : int:=2147483647;
public  JavaLangInteger : ()  ==> JavaLangInteger
public  JavaLangInteger : int  ==> JavaLangInteger
public  JavaLangInteger : char  ==> JavaLangInteger
public  JavaLangInteger : JavaLangString ==> JavaLangInteger
public  getClass : () ==> JavaLangClass
public  hashCode : () ==> int
public  intValue : () ==> int
public  toString : () ==> JavaLangString
public  byteValue : () ==> int
public  longValue : () ==> int
public  charValue : () ==> char
static  public toString : int  ==> JavaLangString
public  floatValue : () ==> real
public  shortValue : () ==> int
public  doubleValue : () ==> real
public  static toString_ : int  * int  ==> JavaLangString
public  JavaLangInteger : <NIL> ==> JavaLangInteger
public  static toHexString : int  ==> JavaLangString
public  static decode : JavaLangString ==> JavaLangInteger
public  equals : JavaLangObject ==> bool
public  static toOctalString : int  ==> JavaLangString
public  static valueOf : JavaLangString ==> JavaLangInteger
public  parseInt : JavaLangString ==> int
public  compareTo : JavaLangObject ==> int
public  compareToInt : JavaLangInteger ==> int
public  static getInteger : JavaLangString ==> JavaLangInteger
public  parseInt : JavaLangString * int  ==> int
public  static getInteger : JavaLangString * int  ==> JavaLangInteger
public  static getInteger : 
          JavaLangString * JavaLangInteger ==> JavaLangInteger
\end{verbatim}
\end{small}

\subsubsection{\texttt{J2VUTIL}}
\begin{small}
\begin{verbatim}
public  BitOp: int * (<AND> | <OR> | <EXCLOR>) * int ==> int
public  ConcatStr: JavaLangString * JavaLangString ==> JavaLangString
public  toString : 
          int  | real | char | JavaLangObject ==> JavaLangString
public  toChar : int  | real | char ==> char
public  toInt : int  | real | char ==> int
public  toFloat : int  | real | char ==> real
\end{verbatim}
\end{small}

\subsubsection{\texttt{JavaLangNullPointerException}}
\begin{small}
\begin{verbatim}
public  JavaLangNullPointerException : 
          () ==> JavaLangNullPointerException
public  JavaLangNullPointerException : 
          <NIL> ==> JavaLangNullPointerException
public  JavaLangNullPointerException : 
          JavaLangString ==> JavaLangNullPointerException
\end{verbatim}
\end{small}

\subsubsection{\texttt{Nullable}}
\begin{small}
\begin{verbatim}
public  IsNil: () ==> bool
public  isNil: bool := false
\end{verbatim}
\end{small}

\subsubsection{\texttt{JavaLangNumber}}
\begin{small}
\begin{verbatim}
public  intValue : () ==> int
public  byteValue : () ==> int
public  longValue : () ==> int
public  floatValue : () ==> real
public  shortValue : () ==> int
public  JavaLangNumber : <NIL> ==> JavaLangNumber
public  doubleValue : () ==> real
\end{verbatim}
\end{small}

\subsubsection{\texttt{JavaLangNumberFormatException}}
\begin{small}
\begin{verbatim}
public  JavaLangNumberFormatException : 
          () ==> JavaLangNumberFormatException
public  JavaLangNumberFormatException : 
          <NIL> ==> JavaLangNumberFormatException
public  JavaLangNumberFormatException : 
          JavaLangString ==> JavaLangNumberFormatException
\end{verbatim}
\end{small}

\subsubsection{\texttt{JavaLangObject}}
\begin{small}
\begin{verbatim}
public JavaLangObject: <NIL> ==> JavaLangObject
public  JavaLangObject : () ==> JavaLangObject
public  wait : () ==> ()
public  clone : () ==> JavaLangObject
public  wait : int  ==> ()
public  notify : () ==> ()
public  wait : int  * int  ==> ()
protected  finalize : () ==> ()
public  getClass : () ==> JavaLangClass
public  hashCode : () ==> int
public  toString : () ==> JavaLangString
public  notifyAll : () ==> ()
public  equals : JavaLangObject ==> bool
\end{verbatim}
\end{small}

\subsubsection{\texttt{JavaLangRuntimeException}}
\begin{small}
\begin{verbatim}
public  JavaLangRuntimeException : () ==> JavaLangRuntimeException
public  JavaLangRuntimeException : <NIL> ==> JavaLangRuntimeException
public  JavaLangRuntimeException : 
          JavaLangString ==> JavaLangRuntimeException
\end{verbatim}
\end{small}

\subsubsection{\texttt{JavaLangString}}
\begin{small}
\begin{verbatim}
public  CLASS : JavaLangClass =
          new JavaLangClass("JavaLangString", <CLASS>, false);
public  JavaLangString : seq of  char  ==> JavaLangString
public  JavaLangString : JavaLangString ==> JavaLangString
public  getClass : () ==> JavaLangClass
public  toSeqOfChar : () ==> seq of  char
public  JavaLangString: <NIL> ==> JavaLangString
public  trim: () ==> JavaLangString
public  JavaLangString : () ==> JavaLangString
public  intern : () ==> JavaLangString
public  length : () ==> int
public  charAt : int  ==> char
public  getBytes : () ==> map  int  to  int
public  hashCode : () ==> int
public  indexOf : int | char ==> int
public  toString : () ==> JavaLangString
public  indexOf : int  * int  ==> int
public  replace : char  * char  ==> JavaLangString
public  substring : int * int ==> JavaLangString
public  substring : int  ==> JavaLangString
public  JavaLangString : map  int  to  char  ==> JavaLangString
public  JavaLangString : map  int  to  int  ==> JavaLangString
public  toCharArray : () ==> map  int  to  char
public  toLowerCase : () ==> JavaLangString
public  toUpperCase : () ==> JavaLangString
public  lastIndexOf : int  ==> int
public  JavaLangString : 
          map  int  to  int  * int  ==> JavaLangString
public  lastIndexOf : int  * int  ==> int
public  JavaLangString : 
          (map  int  to  char)  * int  * int  ==> JavaLangString
public  JavaLangString : 
          map  int  to  int  * int  * int  ==> JavaLangString
public  concat : JavaLangString ==> JavaLangString
public  concat' : seq of char ==> ()
public  static copyValueOf : map  int  to  char  ==> JavaLangString
public  equals: JavaLangObject ==> bool
public  JavaLangString : 
          map  int  to  int  * int  * int  * int  ==> JavaLangString
public  endsWith : JavaLangString ==> bool
public  getBytes : JavaLangString ==> map  int  to  int
public  compareTo : JavaLangObject ==> int
public  getBytes : int  * int  * map  int  to  int  * int  ==> ()
public  getChars : int  * int  * map  int  to  char  * int  ==> ()
public  indexOf : JavaLangString  ==> int
public  indexOf : JavaLangString * int  ==> int
public  static copyValueOf : 
          map  int  to  char  * int  * int  ==> JavaLangString
public  startsWith : JavaLangString * int  ==> bool
public  startsWith : JavaLangString ==> bool
public  lastIndexOf : JavaLangString ==> int
public  lastIndexOf : JavaLangString * int ==> int
public  JavaLangString : 
          map  int  to  int  * JavaLangString ==> JavaLangString
public  JavaLangString : map  int  to  int  * 
                   int  * int  * JavaLangString ==> JavaLangString
public  equalsIgnoreCase : JavaLangString ==> bool
public  compareToIgnoreCase: JavaLangString ==> int
public  regionMatches : int  * JavaLangString * int  * int  ==> bool
public  regionMatches : 
              bool  * int  * JavaLangString * int  * int  ==> bool
public  static  valueOf : 
              map  int  to  char  * int  * int  ==> JavaLangString
public  static  valueOf : map  int  to  char  ==> JavaLangString
\end{verbatim}
\end{small}

\subsubsection{\texttt{JavaLangStringBuffer}}
\begin{small}
\begin{verbatim}
public  CLASS : JavaLangClass =
          new JavaLangClass("JavaLangString", <CLASS>, false);
public  JavaLangStringBuffer : <NIL> ==> JavaLangStringBuffer
public  getClass : () ==> JavaLangClass
public  toString : () ==> JavaLangString
\end{verbatim}
\end{small}

\subsubsection{\texttt{JavaLangSystem}}
\begin{small}
\begin{verbatim}
private  static  props : JavaUtilProperties := 
           new JavaUtilProperties(<NIL>);
public  static  out : JavaIoPrintStream := 
  new JavaIoPrintStream(new JavaIoFileOutputStream("stdout", true), true);
public  static  err : JavaIoPrintStream := 
  new JavaIoPrintStream(new JavaIoFileOutputStream("stderr", true), true);
public  static  gc : () ==> ()
public  JavaLangSystem : () ==> JavaLangSystem
public  static  exit__ : int  ==> ()
public  static  load : JavaLangString ==> ()
public  static  getProperties : () ==> JavaUtilProperties
public  static  getenv : JavaLangString ==> JavaLangString
public  static  runFinalization : () ==> ()
public  static  currentTimeMillis : () ==> int
public  static  getProperty : JavaLangString ==> JavaLangString
public  static  loadLibrary : JavaLangString ==> ()
public  static  runFinalizersOnExit : bool  ==> ()
public  static  mapLibraryName : JavaLangString ==> JavaLangString
public  static  identityHashCode : JavaLangObject ==> int
public  static  getProperty : 
          JavaLangString * JavaLangString ==> JavaLangString
public  static  setProperty : 
          JavaLangString * JavaLangString ==> JavaLangString
\end{verbatim}
\end{small}

\subsubsection{\texttt{JavaLangThrowable}}
\begin{small}
\begin{verbatim}
public  toString : () ==> JavaLangString
public  JavaLangThrowable : () ==> JavaLangThrowable
public  getMessage : () ==> JavaLangString
public  JavaLangThrowable : <NIL> ==> JavaLangThrowable
public  printStackTrace : () ==> ()
public  fillInStackTrace : () ==> JavaLangThrowable
public  JavaLangThrowable : JavaLangString ==> JavaLangThrowable
public  getLocalizedMessage : () ==> JavaLangString
public  printStackTrace : JavaIoPrintStream ==> ()
public  printStackTrace : JavaIoPrintWriter ==> ()
\end{verbatim}
\end{small}

\subsubsection{\texttt{JavaLangUnsupportedOperationException}}
\begin{small}
\begin{verbatim}
public  JavaLangUnsupportedOperationException : 
          () ==> JavaLangUnsupportedOperationException
public  JavaLangUnsupportedOperationException : 
          <NIL> ==> JavaLangUnsupportedOperationException
public  JavaLangUnsupportedOperationException : 
          JavaLangString ==> JavaLangUnsupportedOperationException
\end{verbatim}
\end{small}

\subsection{\texttt{java.util}}

\subsubsection{\texttt{JavaUtilALItr}}
\begin{small}
\begin{verbatim}
protected  cursor : int := 0;
protected  lastRet : int := -1;
protected  expectedModCount : int;
protected  al : JavaUtilAbstractList;
public  hasNext : () ==> bool
public  next : () ==> JavaLangObject
\end{verbatim}
\end{small}

\subsubsection{\texttt{JavaUtilALListItr}}
\begin{small}
\begin{verbatim}
public  hasPrevious: () ==> bool
public  previous : () ==> JavaLangObject
public  nextIndex : () ==> int
public  previousIndex : () ==> int
public  set__ : JavaLangObject ==> ()
\end{verbatim}
\end{small}

\subsubsection{\texttt{JavaUtilAbstractCollection}}
\begin{small}
\begin{verbatim}
public  clear : () ==> ()
public  isEmpty : () ==> bool
public  toArray : () ==> map  int  to  JavaLangObject
public  toString : () ==> JavaLangString
public  contains : JavaLangObject ==> bool
public  addAll : JavaUtilCollection ==> bool
public  removeAll : JavaUtilCollection ==> bool
public  retainAll : JavaUtilCollection ==> bool
public  JavaUtilAbstractCollection : 
          <NIL> ==> JavaUtilAbstractCollection
\end{verbatim}
\end{small}

\subsubsection{\texttt{JavaUtilAbstractList}}
\begin{small}
\begin{verbatim}
public  modCount : int  := 0
public  get : int  ==> JavaLangObject
public  clear : () ==> ()
public  getClass : () ==> JavaLangClass
public  hashCode : () ==> int
public  iterator : () ==> JavaUtilIterator
public  add : JavaLangObject ==> bool
public  listIterator : () ==> JavaUtilListIterator
public  listIterator : int  ==> JavaUtilListIterator
public  equals : JavaLangObject ==> bool
protected  removeRange : int  * int  ==> ()
public  indexOf : JavaLangObject ==> int
public  set__ : int  * JavaLangObject ==> JavaLangObject
public  JavaUtilAbstractList : <NIL> ==> JavaUtilAbstractList
public  lastIndexOf : JavaLangObject ==> int
protected  cursor : int := 0;
protected  lastRet : int := -1;
protected  expectedModCount : int;
protected  al : JavaUtilAbstractList;
public  hasNext : () ==> bool
public  next : () ==> JavaLangObject
public  hasPrevious : () ==> bool
public  previous : () ==> JavaLangObject
public  nextIndex : () ==> int
public  previousIndex : () ==> int
public  set__ : JavaLangObject ==> ()
\end{verbatim}
\end{small}

\subsubsection{\texttt{JavaUtilAbstractMap}}
\begin{small}
\begin{verbatim}
public  size : () ==> int
public  clear : () ==> ()
public  keySet : () ==> JavaUtilSet
public  isEmpty : () ==> bool
public  entrySet : () ==> JavaUtilSet
public  hashCode : () ==> int
public  toString : () ==> JavaLangString
public  values__ : () ==> JavaUtilCollection
protected  JavaUtilAbstractMap : () ==> JavaUtilAbstractMap
public  get : JavaLangObject ==> JavaLangObject
public  putAll : JavaUtilMap ==> ()
public  equals : JavaLangObject ==> bool
public  remove : JavaLangObject ==> JavaLangObject
public  JavaUtilAbstractMap : <NIL> ==> JavaUtilAbstractMap
public  containsKey : JavaLangObject ==> bool
public  put : JavaLangObject * JavaLangObject ==> JavaLangObject
public  containsValue : JavaLangObject ==> bool
\end{verbatim}
\end{small}

\subsubsection{\texttt{JavaUtilAbstractSet}}
\begin{small}
\begin{verbatim}
public  hashCode : () ==> int
public  JavaUtilAbstractSet : () ==> JavaUtilAbstractSet
public  equals : JavaLangObject ==> bool
public  JavaUtilAbstractSet : <NIL> ==> JavaUtilAbstractSet
public  removeAll : JavaUtilCollection ==> bool
\end{verbatim}
\end{small}

\subsubsection{\texttt{JavaUtilCollection}}
\begin{small}
\begin{verbatim}
public  size : () ==> int
public  clear : () ==> ()
public  isEmpty : () ==> bool
public  hashCode : () ==> int
public  iterator : () ==> JavaUtilIterator
public  equals : JavaLangObject ==> bool
public  contains : JavaLangObject ==> bool
public  addAll : JavaUtilCollection ==> bool
public  removeAll : JavaUtilCollection ==> bool
public  retainAll : JavaUtilCollection ==> bool
\end{verbatim}
\end{small}

\subsubsection{\texttt{JavaUtilConcurrentModificationException}}
\begin{small}
\begin{verbatim}
public  JavaUtilConcurrentModificationException : 
          () ==> JavaUtilConcurrentModificationException
public  JavaUtilConcurrentModificationException : 
          <NIL> ==> JavaUtilConcurrentModificationException
public  JavaUtilConcurrentModificationException : 
          JavaLangString ==> JavaUtilConcurrentModificationException
\end{verbatim}
\end{small}

\subsubsection{\texttt{JavaUtilDate}}
\begin{small}
\begin{verbatim}
public  JavaUtilDate : () ==> JavaUtilDate
public  JavaUtilDate : int  ==> JavaUtilDate
public  clone : () ==> JavaLangObject
public  getDay : () ==> int
public  getDate : () ==> int
public  getTime : () ==> int
public  getYear : () ==> int
public  getHours : () ==> int
public  getMonth : () ==> int
public  hashCode : () ==> int
public  setDate : int  ==> ()
public  setTime : int  ==> ()
public  setYear : int  ==> ()
public  toString : () ==> JavaLangString
public  JavaUtilDate : <NIL> ==> JavaUtilDate
public  JavaUtilDate : int  * int  * int  ==> JavaUtilDate
public  setHours : int  ==> ()
public  setMonth : int  ==> ()
public  getMinutes : () ==> int
public  getSeconds : () ==> int
public  after : JavaUtilDate ==> bool
public  setMinutes : int  ==> ()
public  setSeconds : int  ==> ()
public  toGMTString : () ==> JavaLangString
public  JavaUtilDate : JavaLangString ==> JavaUtilDate
public  before : JavaUtilDate ==> bool
public  JavaUtilDate : int  * int  * int  * int  * int  ==> JavaUtilDate
public  static parse : JavaLangString ==> int
public  static UTC : int  * int  * int  * int  * int  * int  ==> int
public  equals : JavaLangObject ==> bool
public  toLocaleString : () ==> JavaLangString
public  JavaUtilDate : 
          int  * int  * int  * int  * int  * int  ==> JavaUtilDate
public  compareTo : JavaLangObject ==> int
public  getTimezoneOffset : () ==> int
\end{verbatim}
\end{small}

\subsubsection{\texttt{JavaUtilDictionary}}
\begin{small}
\begin{verbatim}
public  isEmpty : () ==> bool
public  JavaUtilDictionary : <NIL> ==> JavaUtilDictionary
\end{verbatim}
\end{small}

\subsubsection{\texttt{JavaUtilEmptyEnumerator}}
\begin{small}
\begin{verbatim}
public  nextElement : () ==> JavaLangObject
public  JavaUtilEmptyEnumerator : () ==> JavaUtilEmptyEnumerator
public  hasMoreElements : () ==> bool
\end{verbatim}
\end{small}

\subsubsection{\texttt{JavaUtilEmptyIterator}}
\begin{small}
\begin{verbatim}
public  next : () ==> JavaLangObject
public  remove : () ==> ()
public  hasNext : () ==> bool
public  JavaUtilEmptyIterator : () ==> JavaUtilEmptyIterator
\end{verbatim}
\end{small}

\subsubsection{\texttt{JavaUtilEmptyStackException}}
\begin{small}
\begin{verbatim}
public  JavaUtilEmptyStackException : 
          () ==> JavaUtilEmptyStackException
public  JavaUtilEmptyStackException : 
          <NIL> ==> JavaUtilEmptyStackException
\end{verbatim}
\end{small}

\subsubsection{\texttt{JavaUtilEntry}}
\begin{small}
\begin{verbatim}
public  getClass : () ==> JavaLangClass
public  hashCode : () ==> int
public  equals : JavaLangObject ==> bool
\end{verbatim}
\end{small}

\subsubsection{\texttt{JavaUtilEnumeration}}
\begin{small}
\begin{verbatim}
public  JavaUtilEnumeration : <NIL> ==> JavaUtilEnumeration
public  nextElement : () ==> JavaLangObject
public  hasMoreElements : () ==> bool
protected  expectedModCount : int;
public  remove : () ==> ()
public  hasNext : () ==> bool
public  nextElement : () ==> JavaLangObject
public  JavaUtilEnumerator : 
          int  * bool * JavaUtilHashtable ==> JavaUtilEnumerator
public  JavaUtilEnumerator : <NIL> ==> JavaUtilEnumerator
public  hasMoreElements : () ==> bool
\end{verbatim}
\end{small}

\subsubsection{\texttt{JavaUtilHTEntry}}
\begin{small}
\begin{verbatim}
public  hash : int ;
public  key : JavaLangObject;
public  value__ : JavaLangObject;
public  next : JavaUtilHTEntry
public  clone : () ==> JavaLangObject
public  getKey : () ==> JavaLangObject
public  getValue : () ==> JavaLangObject
public  hashCode : () ==> int
public  toString : () ==> JavaLangString
public  JavaUtilHTEntry : <NIL> ==> JavaUtilHTEntry
public  equals : JavaLangObject ==> bool
public  setValue : JavaLangObject ==> JavaLangObject
public  JavaUtilHTEntry : int  * JavaLangObject * 
                 JavaLangObject * JavaUtilHTEntry ==> JavaUtilHTEntry
\end{verbatim}
\end{small}

\subsubsection{\texttt{JavaUtilHTKeySet}}
\begin{small}
\begin{verbatim}
public  JavaUtilHTKeySet : JavaUtilHashtable ==> JavaUtilHTKeySet
public  iterator : () ==> JavaUtilIterator
public  size : () ==> int
public  contains : JavaLangObject ==> bool
public  remove : JavaLangObject ==> bool
public  clear : () ==> ()
\end{verbatim}
\end{small}

\subsubsection{\texttt{JavaUtilHashMap}}
\begin{small}
\begin{verbatim}
public  hash : int;
public  key : JavaLangObject;
public  value__: JavaLangObject;
public  next : HMEntry;
public  HMEntry : <NIL> ==> HMEntry
public  clone : () ==> JavaLangObject
public  getKey : () ==> JavaLangObject
public  getValue : () ==> JavaLangObject
public  setValue : JavaLangObject ==> JavaLangObject
public  equals : JavaLangObject ==> bool
public  hashCode : () ==> int
public  toString : () ==> JavaLangString
public  table : map int to HMEntry;
public  count : int :=0;
public  threshold : int;
public  loadFactor_ : real;
public  modCount : int := 0;
public  static  KEYS : int := 0;
public  static  VALUES : int := 1;
public  static  ENTRIES : int := 2;
public  size : () ==> int
public  clear : () ==> ()
public  clone : () ==> JavaLangObject
public  keySet : () ==> JavaUtilSet
public  getHashIterator : int ==> JavaUtilIterator
public  JavaUtilHashMap : () ==> JavaUtilHashMap
public  isEmpty : () ==> bool
public  JavaUtilHashMap : int  ==> JavaUtilHashMap
public  entrySet : () ==> JavaUtilSet
public  values__ : () ==> JavaUtilCollection
public  JavaUtilHashMap : int  * real  ==> JavaUtilHashMap
public  get : JavaLangObject ==> JavaLangObject
public  putAll : JavaUtilMap ==> ()
public  JavaUtilHashMap : JavaUtilMap ==> JavaUtilHashMap
public  JavaUtilHashMap : <NIL> ==> JavaUtilHashMap
public  remove : JavaLangObject ==> JavaLangObject
public  containsKey : JavaLangObject ==> bool
public  put : JavaLangObject * JavaLangObject ==> JavaLangObject
protected  rehash : () ==> ()
public  containsValue : JavaLangObject ==> bool
public  HMKeySet: JavaUtilHashMap ==> HMKeySet
public  iterator : () ==> JavaUtilIterator
public  size : () ==> int
public  contains : JavaLangObject ==> bool
public  remove : JavaLangObject ==> bool
public  clear : () ==> ()
public  HMEntrySet : JavaUtilHashMap ==> HMEntrySet
public  iterator : () ==> JavaUtilIterator
public  size : () ==> int
public  contains : JavaLangObject ==> bool
public  remove : JavaLangObject ==> bool
public  clear : () ==> ()
public  next : () ==> JavaLangObject
public  remove : () ==> ()
public  hasNext : () ==> bool
public  EmptyHashIterator : () ==> EmptyHashIterator
public  HashIterator: int * JavaUtilHashMap ==> HashIterator
public  hasNext : () ==> bool
public  next : () ==> JavaLangObject
public  remove : () ==> ()
\end{verbatim}
\end{small}

\subsubsection{\texttt{JavaUtilHashSet}}
\begin{small}
\begin{verbatim}
public  size : () ==> int
public  clear : () ==> ()
public  clone : () ==> JavaLangObject
public  JavaUtilHashSet : () ==> JavaUtilHashSet
public  isEmpty : () ==> bool
public  JavaUtilHashSet : int  ==> JavaUtilHashSet
public  iterator : () ==> JavaUtilIterator
public  JavaUtilHashSet : int  * real  ==> JavaUtilHashSet
public  add : JavaLangObject ==> bool
public  JavaUtilHashSet : <NIL> ==> JavaUtilHashSet
public  remove : JavaLangObject ==> bool
public  contains : JavaLangObject ==> bool
public  JavaUtilHashSet : JavaUtilCollection ==> JavaUtilHashSet
\end{verbatim}
\end{small}

\subsubsection{\texttt{JavaUtilHashtable}}
\begin{small}
\begin{verbatim}
public  static  KEYS : int := 0;
public  static  VALUES : int := 1;
public  static  ENTRIES : int := 2;
public  table : map  int  to  JavaUtilHTEntry;
public  count : int := 0;
public  modCount : int  := 0;
public  keys : () ==> JavaUtilEnumeration
public  size : () ==> int
public  clear : () ==> ()
public  clone : () ==> JavaLangObject
public  keySet : () ==> JavaUtilSet
protected  rehash : () ==> ()
public  isEmpty : () ==> bool
public  elements : () ==> JavaUtilEnumeration
public  entrySet : () ==> JavaUtilSet
public  getClass : () ==> JavaLangClass
public  hashCode : () ==> int
public  toString : () ==> JavaLangString
public  values__ : () ==> JavaUtilCollection
public  JavaUtilHashtable : () ==> JavaUtilHashtable
public  JavaUtilHashtable : int  ==> JavaUtilHashtable
public  get : JavaLangObject ==> JavaLangObject
public  putAll : JavaUtilMap ==> ()
public  JavaUtilHashtable : int * real  ==> JavaUtilHashtable
public  getIterator : int  ==> JavaUtilIterator
public  JavaUtilHashtable : <NIL> ==> JavaUtilHashtable
public  equals : JavaLangObject ==> bool
public  remove : JavaLangObject ==> JavaLangObject
public  contains : JavaLangObject ==> bool
public  containsKey : JavaLangObject ==> bool
public  put : JavaLangObject * JavaLangObject ==> JavaLangObject
public  containsValue : JavaLangObject ==> bool
\end{verbatim}
\end{small}

\subsubsection{\texttt{JavaUtilIterator}}
\begin{small}
\begin{verbatim}
public  JavaUtilIterator : <NIL> ==> JavaUtilIterator
public  next : () ==> JavaLangObject
public  remove : () ==> ()
public  hasNext : () ==> bool
\end{verbatim}
\end{small}

\subsubsection{\texttt{JavaUtilList}}
\begin{small}
\begin{verbatim}
public  get : int  ==> JavaLangObject
public  size : () ==> int
public  clear : () ==> ()
public  isEmpty : () ==> bool
public  toArray : () ==> map  int  to  JavaLangObject
public  hashCode : () ==> int
public  iterator : () ==> JavaUtilIterator
public  subList : int  * int  ==> JavaUtilList
public  equals : JavaLangObject ==> bool
public  indexOf : JavaLangObject ==> int
public  set__ : int  * JavaLangObject ==> JavaLangObject
public  contains : JavaLangObject ==> bool
public  addAll : JavaUtilCollection ==> bool
public  lastIndexOf : JavaLangObject ==> int
public  toArray : map  int  to  JavaLangObject ==> 
                                map  int  to  JavaLangObject
public  addAll : int  * JavaUtilCollection ==> bool
public  removeAll : JavaUtilCollection ==> bool
public  retainAll : JavaUtilCollection ==> bool
public  containsAll : JavaUtilCollection ==> bool
\end{verbatim}
\end{small}

\subsubsection{\texttt{JavaUtilListIterator}}
\begin{small}
\begin{verbatim}
public  next : () ==> JavaLangObject
public  remove : () ==> ()
public  hasNext : () ==> bool
public  previous : () ==> JavaLangObject
public  nextIndex : () ==> int
public  add : JavaLangObject ==> ()
public  hasPrevious : () ==> bool
public  previousIndex : () ==> int
public  set__ : JavaLangObject ==> ()
\end{verbatim}
\end{small}

\subsubsection{\texttt{JavaUtilLocale}}
\begin{small}
\begin{verbatim}
public  static  ENGLISH : [JavaUtilLocale] := nil;
public  static  FRENCH : [JavaUtilLocale] := nil;
public  static  GERMAN : [JavaUtilLocale] := nil;
public  static  ITALIAN : [JavaUtilLocale] := nil;
public  static  JAPANESE : [JavaUtilLocale] := nil;
public  static  KOREAN : [JavaUtilLocale] := nil;
public  static  CHINESE : [JavaUtilLocale] := nil;
public  static  SIMPLIFIED_CHINESE : [JavaUtilLocale] := nil;
public  static  TRADITIONAL_CHINESE : [JavaUtilLocale] := nil;
public  static  FRANCE : [JavaUtilLocale] := nil;
public  static  GERMANY : [JavaUtilLocale] := nil;
public  static  ITALY : [JavaUtilLocale] := nil;
public  static  JAPAN : [JavaUtilLocale] := nil;
public  static  KOREA : [JavaUtilLocale] := nil;
public  static  CHINA : [JavaUtilLocale] := nil;
public  static  PRC : [JavaUtilLocale] := nil;
public  static  TAIWAN : [JavaUtilLocale] := nil;
public  static  UK : [JavaUtilLocale] := nil;
public  static  US : [JavaUtilLocale] := nil;
public  static  CANADA : [JavaUtilLocale] := nil;
public  static  CANADA_FRENCH : [JavaUtilLocale] := nil;
public  clone : () ==> JavaLangObject
public  hashCode : () ==> int
public  toString : () ==> JavaLangString
public  getCountry : () ==> JavaLangString
public  static  getDefault : () ==> JavaUtilLocale
public  getVariant : () ==> JavaLangString
public  JavaUtilLocale : <NIL> ==> JavaUtilLocale
public  getLanguage : () ==> JavaLangString
public  equals : JavaLangObject ==> bool
public  getDisplayName : () ==> JavaLangString
public  getISO3Country : () ==> JavaLangString
public  getISO3Language : () ==> JavaLangString
public  static  getISOCountries : () ==> map  int  to  JavaLangString
public  static  getISOLanguages : () ==> map  int  to  JavaLangString
public  getDisplayCountry : () ==> JavaLangString
public  getDisplayVariant : () ==> JavaLangString
public  getDisplayLanguage : () ==> JavaLangString
public  static setDefault : JavaUtilLocale ==> ()
public  static getAvailableLocales : 
          () ==> map  int  to  JavaUtilLocale
public  getDisplayName : JavaUtilLocale ==> JavaLangString
public  JavaUtilLocale : 
          JavaLangString * JavaLangString ==> JavaUtilLocale
public  getDisplayCountry : JavaUtilLocale ==> JavaLangString
public  getDisplayVariant : JavaUtilLocale ==> JavaLangString
public  getDisplayLanguage : JavaUtilLocale ==> JavaLangString
public  JavaUtilLocale : JavaLangString * JavaLangString * 
                                    JavaLangString ==> JavaUtilLocale
\end{verbatim}
\end{small}

\subsubsection{\texttt{JavaUtilMap}}
\begin{small}
\begin{verbatim}
public  size : () ==> int
public  clear : () ==> ()
public  keySet : () ==> JavaUtilSet
public  isEmpty : () ==> bool
public  get : JavaLangObject ==> JavaLangObject
public  containsKey : JavaLangObject ==> bool
public  put : JavaLangObject * JavaLangObject ==> JavaLangObject
public  containsValue : JavaLangObject ==> bool
\end{verbatim}
\end{small}

\subsubsection{\texttt{JavaUtilMissingResourceException}}
\begin{small}
\begin{verbatim}
public  getKey : () ==> JavaLangString
public  getClassName : () ==> JavaLangString
public  JavaUtilMissingResourceException : 
          <NIL> ==> JavaUtilMissingResourceException
public  JavaUtilMissingResourceException : 
          JavaLangString * JavaLangString * JavaLangString ==> 
                                JavaUtilMissingResourceException
\end{verbatim}
\end{small}

\subsubsection{\texttt{JavaUtilNoSuchElementException}}
\begin{small}
\begin{verbatim}
public  JavaUtilNoSuchElementException : 
          () ==> JavaUtilNoSuchElementException
public  JavaUtilNoSuchElementException : 
          <NIL> ==> JavaUtilNoSuchElementException
public  JavaUtilNoSuchElementException : 
          JavaLangString ==> JavaUtilNoSuchElementException
\end{verbatim}
\end{small}

\subsubsection{\texttt{JavaUtilObservable}}
\begin{small}
\begin{verbatim}
public  JavaUtilObservable : () ==> JavaUtilObservable
public  hasChanged : () ==> bool
protected  setChanged : () ==> ()
protected  clearChanged : () ==> ()
public  countObservers : () ==> int
public  JavaUtilObservable : <NIL> ==> JavaUtilObservable
public  deleteObservers : () ==> ()
public  notifyObservers : () ==> ()
public  addObserver : JavaUtilObserver ==> ()
public  notifyObservers : JavaLangObject ==> ()
public  deleteObserver : JavaUtilObserver ==> ()
\end{verbatim}
\end{small}

\subsubsection{\texttt{JavaUtilObserver}}
\begin{small}
\begin{verbatim}
public  JavaUtilObserver : <NIL> ==> JavaUtilObserver
public  update : JavaUtilObservable * JavaLangObject ==> ()
\end{verbatim}
\end{small}

\subsubsection{\texttt{JavaUtilProperties}}
\begin{small}
\begin{verbatim}
protected  defaults : JavaUtilProperties;
public  JavaUtilProperties : () ==> JavaUtilProperties
public  propertyNames : () ==> JavaUtilEnumeration
public  JavaUtilProperties : <NIL> ==> JavaUtilProperties
public  list : JavaIoPrintStream ==> ()
public  load : JavaIoInputStream ==> ()
public  getProperty : JavaLangString ==> JavaLangString
public  JavaUtilProperties : JavaUtilProperties ==> JavaUtilProperties
public  save : JavaIoOutputStream * JavaLangString ==> ()
public  getProperty : JavaLangString * JavaLangString ==> JavaLangString
public  setProperty : JavaLangString * JavaLangString ==> JavaLangObject
public  store : JavaIoOutputStream * JavaLangString ==> ()
\end{verbatim}
\end{small}

\subsubsection{\texttt{JavaUtilResourceBundle}}
\begin{small}
\begin{verbatim}
protected  parent : JavaUtilResourceBundle
public  getKeys : () ==> JavaUtilEnumeration
public  getLocale : () ==> JavaUtilLocale
public  JavaUtilResourceBundle : () ==> JavaUtilResourceBundle
public  static getBundle : JavaLangString ==> JavaUtilResourceBundle
public  getObject : JavaLangString ==> JavaLangObject
public  getString : JavaLangString ==> JavaLangString
public  JavaUtilResourceBundle : <NIL> ==> JavaUtilResourceBundle
public  getStringArray : JavaLangString ==> map  int  to  JavaLangString
protected  handleGetObject : JavaLangString ==> JavaLangObject
protected  setParent : JavaUtilResourceBundle ==> ()
public  static  getBundle : JavaLangString * 
                       JavaUtilLocale ==> JavaUtilResourceBundle
\end{verbatim}
\end{small}

\subsubsection{\texttt{JavaUtilSet}}
\begin{small}
\begin{verbatim}
public  clear : () ==> ()
public  isEmpty : () ==> bool
public  toArray : () ==> map  int  to  JavaLangObject
public  hashCode : () ==> int
public  iterator : () ==> JavaUtilIterator
public  add : JavaLangObject ==> bool
public  equals : JavaLangObject ==> bool
public  contains : JavaLangObject ==> bool
public  addAll : JavaUtilCollection ==> bool
public  toArray : map  int  to  JavaLangObject ==> 
                                map  int  to  JavaLangObject
public  removeAll : JavaUtilCollection ==> bool
public  retainAll : JavaUtilCollection ==> bool
public  containsAll : JavaUtilCollection ==> bool
\end{verbatim}
\end{small}

\subsubsection{\texttt{JavaUtilStack}}
\begin{small}
\begin{verbatim}
public  pop : () ==> JavaLangObject
public  peek : () ==> JavaLangObject
public  JavaUtilStack : () ==> JavaUtilStack
public  empty : () ==> bool
public  getClass : () ==> JavaLangClass
public  JavaUtilStack : <NIL> ==> JavaUtilStack
public  push : JavaLangObject ==> JavaLangObject
public  search : JavaLangObject ==> int
\end{verbatim}
\end{small}

\subsubsection{\texttt{JavaUtilStringTokenizer}}
\begin{small}
\begin{verbatim}
public  nextToken : () ==> JavaLangString
public  countTokens : () ==> int
public  nextElement : () ==> JavaLangObject
public  hasMoreTokens : () ==> bool
public  hasMoreElements : () ==> bool
public  nextToken : JavaLangString ==> JavaLangString
public  JavaUtilStringTokenizer : <NIL> ==> JavaUtilStringTokenizer
public  JavaUtilStringTokenizer : 
          JavaLangString ==> JavaUtilStringTokenizer
public  JavaUtilStringTokenizer : J
          avaLangString * JavaLangString ==> JavaUtilStringTokenizer
public  JavaUtilStringTokenizer : JavaLangString * 
              JavaLangString * bool  ==> JavaUtilStringTokenizer
\end{verbatim}
\end{small}

\subsubsection{\texttt{JavaUtilVector}}
\begin{small}
\begin{verbatim}
public  insertElementAt : JavaLangObject * int ==> ()
public  clear : () ==> ()
public  clone : () ==> JavaLangObject
public  contains : JavaLangObject ==> bool
public  containsAll : JavaUtilCollection ==> bool
public  elementAt : int ==> JavaLangObject
public  firstElement : () ==> JavaLangObject
public  get : int ==> JavaLangObject
public  isEmpty : () ==> bool
public  indexOf : JavaLangObject ==> int
public  indexOfFrom : JavaLangObject * int ==> int
public  lastElement : () ==> JavaLangObject
public  remove : int ==> JavaLangObject
public  remove' : JavaLangObject ==> bool
public  removeElementAt : int ==> ()
public  removeElement : JavaLangObject ==> bool
public  removeAll : JavaUtilCollection ==> bool
public  retainAll : JavaUtilCollection ==> bool
public  subList : int * int ==> JavaUtilList
public  elements : () ==> JavaUtilEnumeration
public  VEnumeration : JavaUtilVector ==> VEnumeration
public  hasMoreElements : () ==> bool
public  nextElement : () ==> JavaLangObject
\end{verbatim}
\end{small}

\subsection{\texttt{java.io}}

\subsubsection{\texttt{JavaIoBufferedInputStream}}
\begin{small}
\begin{verbatim}
protected  buf : map  int  to  int ;
protected  count : int ;
protected  pos : int ;
protected  markpos : int  := -1;
protected  marklimit : int
public  read : () ==> int
public  close : () ==> ()
public  mark : int  ==> ()
public  reset : () ==> ()
public  skip__ : int  ==> int
public  getClass : () ==> JavaLangClass
public  available : () ==> int
public  markSupported : () ==> bool
public  read : map  int  to  int  * int  * int  ==> int
public  readMIIIntInt' : map  int  to  int  * 
                     int  * int  ==> int * map  int  to  int
public  JavaIoBufferedInputStream : 
          <NIL> ==> JavaIoBufferedInputStream
public  JavaIoBufferedInputStream : 
          JavaIoInputStream ==> JavaIoBufferedInputStream
public  JavaIoBufferedInputStream : 
          JavaIoInputStream * int  ==> JavaIoBufferedInputStream
\end{verbatim}
\end{small}

\subsubsection{\texttt{JavaIoBufferedOutputStream}}
\begin{small}
\begin{verbatim}
protected  buf : map  int  to  int ;
protected  count : int
public  flush : () ==> ()
public  write : int  ==> ()
public  getClass : () ==> JavaLangClass
public  write2 : map  int  to  int  * int  * int  ==> ()
public  JavaIoBufferedOutputStream : 
          <NIL> ==> JavaIoBufferedOutputStream
public  JavaIoBufferedOutputStream : 
          JavaIoOutputStream ==> JavaIoBufferedOutputStream
public  JavaIoBufferedOutputStream : 
          JavaIoOutputStream * int  ==> JavaIoBufferedOutputStream
\end{verbatim}
\end{small}

\subsubsection{\texttt{JavaIoBufferedReader}}
\begin{small}
\begin{verbatim}
public  read : () ==> int
public  close : () ==> ()
public  mark : int  ==> ()
public  ready : () ==> bool
public  reset : () ==> ()
public  skip__ : int  ==> int
public  readLine : () ==> JavaLangString
public  markSupported : () ==> bool
public  read : map  int  to  char  * int  * int  ==> int
public  read' : map  int  to  char  * 
               int  * int  ==> int * map  int  to  char
public  JavaIoBufferedReader : <NIL> ==> JavaIoBufferedReader
public  JavaIoBufferedReader : JavaIoReader ==> JavaIoBufferedReader
public  JavaIoBufferedReader : 
          JavaIoReader * int  ==> JavaIoBufferedReader
\end{verbatim}
\end{small}

\subsubsection{\texttt{JavaIoBufferedWriter}}
\begin{small}
\begin{verbatim}
public  close : () ==> ()
public  flush : () ==> ()
public  write : int  ==> ()
public  newLine : () ==> ()
public  flushBuffer : () ==> ()
public  write : (map  int  to  char | JavaLangString)  * 
                                                 int  * int  ==> ()
public  writeICMIntInt : map  int  to  char  * int  * int  ==> ()
public  writeStrIntInt : JavaLangString * int  * int  ==> ()
public  JavaIoBufferedWriter : <NIL> ==> JavaIoBufferedWriter
public  JavaIoBufferedWriter : JavaIoWriter ==> JavaIoBufferedWriter
public  JavaIoBufferedWriter : 
          JavaIoWriter * int  ==> JavaIoBufferedWriter
\end{verbatim}
\end{small}

\subsubsection{\texttt{JavaIoByteArrayInputStream}}
\begin{small}
\begin{verbatim}
protected  buf : map  int  to  int ;
protected  pos : int ;
protected  mark_ : int  := 0;
protected  count : int
public  read : () ==> int
public  close : () ==> ()
public  mark : int  ==> ()
public  reset : () ==> ()
public  skip__ : int  ==> int
public  getClass : () ==> JavaLangClass
public  available : () ==> int
public  markSupported : () ==> bool
public  read : map  int  to  int  * int  * int  ==> int
public  readMIIIntInt' : map  int  to  int  * int  * int  ==> 
                                             int * map  int  to  int
public  JavaIoByteArrayInputStream : 
          map  int  to  int  ==> JavaIoByteArrayInputStream
public  JavaIoByteArrayInputStream : <NIL> ==> JavaIoByteArrayInputStream
public  JavaIoByteArrayInputStream : 
          map  int  to  int  * int  * int  ==> JavaIoByteArrayInputStream
\end{verbatim}
\end{small}

\subsubsection{\texttt{JavaIoCharArrayReader}}
\begin{small}
\begin{verbatim}
protected  buf : map  int  to  char ;
protected  pos : int ;
protected  markedPos : int  := 0;
protected  count : int
public  read : () ==> int
public  close : () ==> ()
public  mark : int  ==> ()
public  ready : () ==> bool
public  reset : () ==> ()
public  skip__ : int  ==> int
public  getClass : () ==> JavaLangClass
public  markSupported : () ==> bool
public  read : map  int  to  char  * int  * int  ==> 
                                      int * map  int  to  char
public  JavaIoCharArrayReader : 
          map  int  to  char  ==> JavaIoCharArrayReader
public  JavaIoCharArrayReader : <NIL> ==> JavaIoCharArrayReader
public  JavaIoCharArrayReader : 
          map  int  to  char  * int  * int  ==> JavaIoCharArrayReader
\end{verbatim}
\end{small}

\subsubsection{\texttt{JavaIoFile}}
\begin{small}
\begin{verbatim}
public  static  separatorChar : char  := fs.getSeparator();
public  static  separator : JavaLangString := 
            new JavaLangString([fs.getSeparator()]);
public  static  pathSeparatorChar : char  := 
                              fs.getPathSeparator();
public  static  pathSeparator : JavaLangString := 
             new JavaLangString([fs.getPathSeparator()]);
public  list : () ==> map  int  to  JavaLangString
public  mkdir : () ==> bool
public  toURL : () ==> JavaNetURL
public  delete : () ==> bool
public  isFile : () ==> bool
public  length : () ==> int
public  mkdirs : () ==> bool
public  canRead : () ==> bool
public  getName : () ==> JavaLangString
public  getPath : () ==> JavaLangString
public  canWrite : () ==> bool
public  exists__ : () ==> bool
public  getClass : () ==> JavaLangClass
public  hashCode : () ==> int
public  isHidden : () ==> bool
public  toString : () ==> JavaLangString
public  JavaIoFile : <NIL> ==> JavaIoFile
public  getParent : () ==> JavaLangString
public  listFiles : () ==> map  int  to  JavaIoFile
public  static listRoots : () ==> map  int  to  JavaIoFile
public  isAbsolute : () ==> bool
public  isDirectory : () ==> bool
public  setReadOnly : () ==> bool
public  JavaIoFile : JavaLangString ==> JavaIoFile
public  deleteOnExit : () ==> ()
public  lastModified : () ==> int
public  createNewFile : () ==> bool
public  getParentFile : () ==> JavaIoFile
public  JavaIoFile : JavaLangString * int  ==> JavaIoFile
public  JavaIoFile : JavaIoFile * JavaLangString ==> JavaIoFile
public  equals : JavaLangObject ==> bool
public  renameTo : JavaIoFile ==> bool
public  getAbsoluteFile : () ==> JavaIoFile
public  getAbsolutePath : () ==> JavaLangString
public  getCanonicalFile : () ==> JavaIoFile
public  getCanonicalPath : () ==> JavaLangString
public  setLastModified : int  ==> bool
public  compareTo : JavaLangObject ==> int
public  JavaIoFile : JavaLangString * JavaLangString ==> JavaIoFile
public  static createTempFile : 
          JavaLangString * JavaLangString ==> JavaIoFile
public  static createTempFile : 
          JavaLangString * JavaLangString * JavaIoFile ==> JavaIoFile
\end{verbatim}
\end{small}

\subsubsection{\texttt{JavaIoFileDescriptor}}
\begin{small}
\begin{verbatim}
public  valid : () ==> bool
public  sync__ : () ==> ()
public  JavaIoFileDescriptor : () ==> JavaIoFileDescriptor
public  JavaIoFileDescriptor : <NIL> ==> JavaIoFileDescriptor
public  getClass : () ==> JavaLangClass
\end{verbatim}
\end{small}

\subsubsection{\texttt{JavaIoFileInputStream}}
\begin{small}
\begin{verbatim}
public  read : () ==> int
public  close : () ==> ()
public  getFD : () ==> JavaIoFileDescriptor
public  skip__ : int  ==> int
protected  finalize : () ==> ()
public  available : () ==> int
public  readMIIIntInt' : map  int  to  int  * int  * int  ==> 
                                                int * map  int  to  int
public  JavaIoFileInputStream : JavaIoFile ==> JavaIoFileInputStream
public  JavaIoFileInputStream : JavaLangString ==> JavaIoFileInputStream
public  JavaIoFileInputStream : seq of char ==> JavaIoFileInputStream
public  JavaIoFileInputStream : 
          JavaIoFileDescriptor ==> JavaIoFileInputStream
\end{verbatim}
\end{small}

\subsubsection{\texttt{JavaIoFileNotFoundException}}
\begin{small}
\begin{verbatim}
public  JavaIoFileNotFoundException : () ==> JavaIoFileNotFoundException
public  JavaIoFileNotFoundException : 
          <NIL> ==> JavaIoFileNotFoundException
public  JavaIoFileNotFoundException : 
          JavaLangString ==> JavaIoFileNotFoundException
\end{verbatim}
\end{small}

\subsubsection{\texttt{JavaIoFileOutputStream}}
\begin{small}
\begin{verbatim}
public  close : () ==> ()
public  getFD : () ==> JavaIoFileDescriptor
public  write : int  ==> ()
protected  finalize : () ==> ()
public  JavaIoFileOutputStream : JavaIoFile ==> JavaIoFileOutputStream
public  JavaIoFileOutputStream : JavaLangString ==> JavaIoFileOutputStream
public  JavaIoFileOutputStream : 
          JavaLangString * bool  ==> JavaIoFileOutputStream
public  JavaIoFileOutputStream : 
          seq of char * bool  ==> JavaIoFileOutputStream
public  JavaIoFileOutputStream : 
          JavaIoFileDescriptor ==> JavaIoFileOutputStream
\end{verbatim}
\end{small}

\subsubsection{\texttt{JavaIoFileReader}}
\begin{small}
\begin{verbatim}
public  JavaIoFileReader : <NIL> ==> JavaIoFileReader
public  JavaIoFileReader : JavaIoFile ==> JavaIoFileReader
public  JavaIoFileReader : JavaLangString ==> JavaIoFileReader
public  JavaIoFileReader : JavaIoFileDescriptor ==> JavaIoFileReader
\end{verbatim}
\end{small}

\subsubsection{\texttt{JavaIoFileSystem}}
\begin{small}
\begin{verbatim}
public  static  BA_EXISTS : int :=1;
public  static  BA_REGULAR : int :=2;
public  static  BA_DIRECTORY : int :=4;
public  static  BA_HIDDEN : int :=8
public  listRoots : () ==> map  int  to  JavaIoFile
public  list : JavaIoFile ==> map  int  to  JavaLangString
public  delete : JavaIoFile ==> bool
public  getSeparator : () ==> char
public  static  getFileSystem : () ==> JavaIoFileSystem
public  resolve : JavaIoFile ==> JavaLangString
public  hashCode : JavaIoFile ==> int
public  getLength : JavaIoFile ==> int
public  getDefaultParent : () ==> JavaLangString
public  getPathSeparator : () ==> char
public  isAbsolute : JavaIoFile ==> bool
public  normalize : JavaLangString ==> JavaLangString
public  setReadOnly : JavaIoFile ==> bool
public  deleteOnExit : JavaIoFile ==> bool
public  checkAccess : JavaIoFile * bool  ==> bool
public  rename : JavaIoFile * JavaIoFile ==> bool
public  canonicalize : JavaLangString ==> JavaLangString
public  compare : JavaIoFile * JavaIoFile ==> int
public  prefixLength : JavaLangString ==> int
public  createDirectory : JavaIoFile ==> bool
public  resolve : JavaLangString * JavaLangString ==> JavaLangString
public  getLastModifiedTime : JavaIoFile ==> int
public  getBooleanAttributes : JavaIoFile ==> int
public  setLastModifiedTime : JavaIoFile * int  ==> bool
public  createFileExclusively : JavaLangString ==> bool
public  J2VFileSystem : <NIL> ==> J2VFileSystem
public  listRoots : () ==> map  int  to  JavaIoFile
public  list : JavaIoFile ==> map  int  to  JavaLangString
public  delete : JavaIoFile ==> bool
public  getSeparator : () ==> char
public  static  getFileSystem : () ==> JavaIoFileSystem
public  resolve : JavaIoFile ==> JavaLangString
public  hashCode : JavaIoFile ==> int
public  getLength : JavaIoFile ==> int
public  getDefaultParent : () ==> JavaLangString
public  getPathSeparator : () ==> char
public  isAbsolute : JavaIoFile ==> bool
public  normalize : JavaLangString ==> JavaLangString
public  setReadOnly : JavaIoFile ==> bool
public  deleteOnExit : JavaIoFile ==> bool
public  checkAccess : JavaIoFile * bool  ==> bool
public  rename : JavaIoFile * JavaIoFile ==> bool
public  canonicalize : JavaLangString ==> JavaLangString
public  compare : JavaIoFile * JavaIoFile ==> int
public  prefixLength : JavaLangString ==> int
public  createDirectory : JavaIoFile ==> bool
public  resolve : JavaLangString * JavaLangString ==> JavaLangString
public  getLastModifiedTime : JavaIoFile ==> int
public  getBooleanAttributes : JavaIoFile ==> int
public  setLastModifiedTime : JavaIoFile * int  ==> bool
public  createFileExclusively : JavaLangString ==> bool
\end{verbatim}
\end{small}

\subsubsection{\texttt{JavaIoFileWriter}}
\begin{small}
\begin{verbatim}
public  JavaIoFileWriter : <NIL> ==> JavaIoFileWriter
public  JavaIoFileWriter : JavaLangString ==> JavaIoFileWriter
public  JavaIoFileWriter : JavaLangString * bool  ==> JavaIoFileWriter
public  JavaIoFileWriter : JavaIoFileDescriptor ==> JavaIoFileWriter
public  JavaIoFileWriter : JavaIoFile ==> JavaIoFileWriter
\end{verbatim}
\end{small}

\subsubsection{\texttt{JavaIoFilterInputStream}}
\begin{small}
\begin{verbatim}
protected  in__ : JavaIoInputStream
public  read : () ==> int
public  close : () ==> ()
public  mark : int  ==> ()
public  reset : () ==> ()
public  skip__ : int  ==> int
public  getClass : () ==> JavaLangClass
public  available : () ==> int
public  read : map  int  to  int  ==> int
public  markSupported : () ==> bool
public  read : map  int  to  int  * int  * int  ==> int
public  JavaIoFilterInputStream : <NIL> ==> JavaIoFilterInputStream
protected  JavaIoFilterInputStream : 
             JavaIoInputStream ==> JavaIoFilterInputStream
\end{verbatim}
\end{small}

\subsubsection{\texttt{JavaIoFilterOutputStream}}
\begin{small}
\begin{verbatim}
protected  out : JavaIoOutputStream
public  close : () ==> ()
public  flush : () ==> ()
public  write : int  ==> ()
public  write1 : map  int  to  int  ==> ()
public  write2 : map  int  to  int  * int  * int  ==> ()
public  JavaIoFilterOutputStream : 
          JavaIoOutputStream ==> JavaIoFilterOutputStream
\end{verbatim}
\end{small}

\subsubsection{\texttt{JavaIoIOException}}
\begin{small}
\begin{verbatim}
public  JavaIoIOException : () ==> JavaIoIOException
public  JavaIoIOException : <NIL> ==> JavaIoIOException
public  JavaIoIOException : JavaLangString ==> JavaIoIOException
\end{verbatim}
\end{small}

\subsubsection{\texttt{JavaIoInputStream}}
\begin{small}
\begin{verbatim}
public  read : () ==> int
public  close : () ==> ()
public  mark : int  ==> ()
public  reset : () ==> ()
public  skip__ : int  ==> int
public  getClass : () ==> JavaLangClass
public  available : () ==> int
public  read : map  int  to  int  ==> int
public  markSupported : () ==> bool
public  readMIIIntInt : map  int  to  int  * int  * int  ==> int
public  readMIIIntInt' : map  int  to  int  * 
                  int  * int  ==> int * map  int  to  int
public  JavaIoInputStream : <NIL> ==> JavaIoInputStream
\end{verbatim}
\end{small}

\subsubsection{\texttt{JavaIoInputStreamReader}}
\begin{small}
\begin{verbatim}
public  read : () ==> int
public  close : () ==> ()
public  ready : () ==> bool
public  getClass : () ==> JavaLangClass
public  getEncoding : () ==> JavaLangString
public  readMICIntInt : map  int  to  char  * int  * int  ==> int
public  readMICIntInt' : map  int  to  char  * 
                            int  * int  ==> int * map  int  to  char
public  JavaIoInputStreamReader : <NIL> ==> JavaIoInputStreamReader
public  JavaIoInputStreamReader : 
          JavaIoInputStream ==> JavaIoInputStreamReader
public  JavaIoInputStreamReader : JavaIoInputStream * 
                          JavaLangString ==> JavaIoInputStreamReader
public  ByteToCharConverter : <NIL> ==> ByteToCharConverter
\end{verbatim}
\end{small}

\subsubsection{\texttt{JavaIoOutputStream}}
\begin{small}
\begin{verbatim}
public  close : () ==> ()
public  flush : () ==> ()
public  write : int  ==> ()
public  write1 : map  int  to  int  ==> ()
public  write2 : map  int  to  int  * int  * int  ==> ()
\end{verbatim}
\end{small}

\subsubsection{\texttt{JavaIoOutputStreamWriter}}
\begin{small}
\begin{verbatim}
public  close : () ==> ()
public  flush : () ==> ()
public  write : int  ==> ()
public  flushBuffer : () ==> ()
public  getEncoding : () ==> JavaLangString
public  writeICMIntInt : map  int  to  char  * int  * int  ==> ()
public  writeStrIntInt : JavaLangString * int  * int  ==> ()
public  JavaIoOutputStreamWriter : <NIL> ==> JavaIoOutputStreamWriter
public  JavaIoOutputStreamWriter : 
          JavaIoOutputStream ==> JavaIoOutputStreamWriter
public  JavaIoOutputStreamWriter : 
          JavaIoOutputStream * JavaLangString ==> JavaIoOutputStreamWriter
public  CharToByteConverter : <NIL> ==> CharToByteConverter
\end{verbatim}
\end{small}

\subsubsection{\texttt{JavaIoPrintStream}}
\begin{small}
\begin{verbatim}
public  close : () ==> ()
public  flush : () ==> ()
public  print : bool | char | int | real | 
                        (map  int  to  char ) | JavaLangObject ==> ()
public  write : int  ==> ()
public  println : () ==> ()
public  println : bool | char | int | real | 
                        (map  int  to  char ) | JavaLangObject ==> ()
protected  setError : () ==> ()
public  checkError : () ==> bool
public  write : map  int  to  int  * int  * int  ==> ()
public  JavaIoPrintStream : 
          JavaIoOutputStream * bool  ==> JavaIoPrintStream
\end{verbatim}
\end{small}

\subsubsection{\texttt{JavaIoPrintWriter}}
\begin{small}
\begin{verbatim}
protected  out : JavaIoWriter;
public  close : () ==> ()
public  flush : () ==> ()
public  write : int | map  int  to  char ==> ()
public  println : () ==> ()
public  println : bool | char | int | real | 
          (map  int  to  char ) | JavaLangObject  ==> ()
public  print : bool | char | int | real | 
          (map  int  to  char ) | JavaLangObject ==> ()
protected  setError : () ==> ()
public  checkError : () ==> bool
public  write : JavaLangString ==> ()
public  write : map  int  to  char  * int  * int  ==> ()
public  JavaIoPrintWriter : <NIL> ==> JavaIoPrintWriter
public  write : JavaLangString * int  * int  ==> ()
public  JavaIoPrintWriter : JavaIoWriter ==> JavaIoPrintWriter
public  JavaIoPrintWriter : JavaIoWriter * bool  ==> JavaIoPrintWriter
public  JavaIoPrintWriter : JavaIoOutputStream ==> JavaIoPrintWriter
public  JavaIoPrintWriter : 
          JavaIoOutputStream * bool  ==> JavaIoPrintWriter
\end{verbatim}
\end{small}

\subsubsection{\texttt{JavaIoReader}}
\begin{small}
\begin{verbatim}
public  read : () ==> int
public  close : () ==> ()
public  mark : int  ==> ()
public  ready : () ==> bool
public  reset : () ==> ()
public  JavaIoReader : () ==> JavaIoReader
public  skip__ : int  ==> int
public  getClass : () ==> JavaLangClass
public  readMIC : map  int  to  char  ==> int
public  JavaIoReader : <NIL> ==> JavaIoReader
public  markSupported : () ==> bool
public  readMICIntInt : map  int  to  char  * int  * int  ==> int
public  readMICIntInt' : map  int  to  char  * 
                          int  * int  ==> int * map  int  to  char
protected  JavaIoReader : JavaLangObject ==> JavaIoReader
\end{verbatim}
\end{small}

\subsubsection{\texttt{JavaIoStreamTokenizer}}
\begin{small}
\begin{verbatim}
public  ttype : int  := TT_NOTHING;
public  static  TT_EOF : int  := -1;
public  static  TT_EOL : int  := J2VUTIL`toInt('\n');
public  static  TT_NUMBER : int  := -2;
public  static  TT_WORD : int  := -3;
public  sval : JavaLangString;
public  nval : real
public  lineno : () ==> int
public  pushBack : () ==> ()
public  toString : () ==> JavaLangString
public  nextToken : () ==> int
public  quoteChar : int  ==> ()
public  resetSyntax : () ==> ()
public  commentChar : int  ==> ()
public  parseNumbers : () ==> ()
public  wordChars : int  * int  ==> ()
public  ordinaryChar : int  ==> ()
public  lowerCaseMode : bool  ==> ()
public  ordinaryChars : int  * int  ==> ()
public  eolIsSignificant : bool  ==> ()
public  slashStarComments : bool  ==> ()
public  whitespaceChars : int  * int  ==> ()
public  slashSlashComments : bool  ==> ()
public  JavaIoStreamTokenizer : <NIL> ==> JavaIoStreamTokenizer
public  JavaIoStreamTokenizer : JavaIoReader ==> JavaIoStreamTokenizer
public  JavaIoStreamTokenizer : 
          JavaIoInputStream ==> JavaIoStreamTokenizer
\end{verbatim}
\end{small}

\subsubsection{\texttt{JavaIoStringWriter}}
\begin{small}
\begin{verbatim}
public  close : () ==> ()
public  flush : () ==> ()
public  write : int  ==> ()
public  toString : () ==> JavaLangString
public  getBuffer : () ==> JavaLangStringBuffer
public  JavaIoStringWriter : () ==> JavaIoStringWriter
public  JavaIoStringWriter : int  ==> JavaIoStringWriter
public  write : JavaLangString ==> ()
public  write : map  int  to  char  * int  * int  ==> ()
public  JavaIoStringWriter : <NIL> ==> JavaIoStringWriter
public  write : JavaLangString * int  * int  ==> ()
\end{verbatim}
\end{small}

\subsubsection{\texttt{JavaIoUnsupportedEncodingException}}
\begin{small}
\begin{verbatim}
public  JavaIoUnsupportedEncodingException : 
          () ==> JavaIoUnsupportedEncodingException
public  JavaIoUnsupportedEncodingException : 
          <NIL> ==> JavaIoUnsupportedEncodingException
public  JavaIoUnsupportedEncodingException : 
          JavaLangString ==> JavaIoUnsupportedEncodingException
\end{verbatim}
\end{small}

\subsubsection{\texttt{JavaIoWriter}}
\begin{small}
\begin{verbatim}
public  JavaIoWriter : <NIL> ==> JavaIoWriter
public  close : () ==> ()
public  flush : () ==> ()
protected  JavaIoWriter : () ==> JavaIoWriter
public  write : int | map  int  to  char ==> ()
public  writeInt : int  ==> ()
public  getClass : () ==> JavaLangClass
public  writeMIC : map  int  to  char  ==> ()
public  writeICMIntInt : map  int  to  char  * int  * int  ==> ()
public  writeStr : JavaLangString ==> ()
protected  JavaIoWriter : JavaLangObject ==> JavaIoWriter
public  writeStrIntInt : JavaLangString * int  * int  ==> ()
\end{verbatim}
\end{small}

\subsection{\texttt{java.net}}

\subsubsection{\texttt{JavaNetURL}}
\begin{small}
\begin{verbatim}
public JavaNetURL : <NIL> ==> JavaNetURL
\end{verbatim}
\end{small}

\subsection{\texttt{java.text}}

\subsubsection{\texttt{JavaTextDateFormat}}
\begin{small}
\begin{verbatim}
public  getClass : () ==> JavaLangClass
protected  JavaTextDateFormat : () ==> JavaTextDateFormat
public  format : JavaUtilDate ==> JavaLangString
public  JavaTextDateFormat : <NIL> ==> JavaTextDateFormat
\end{verbatim}
\end{small}

\subsubsection{\texttt{JavaTextDecimalFormat}}
\begin{small}
\begin{verbatim}
public  static  currentSerialVersion : int := 2;
public  static  DOUBLE_INTEGER_DIGITS : int := 309;
public  static  DOUBLE_FRACTION_DIGITS : int := 340;
public  clone : () ==> JavaLangObject
public  getClass : () ==> JavaLangClass
public  hashCode : () ==> int
public  toPattern : () ==> JavaLangString
public  JavaTextDecimalFormat : () ==> JavaTextDecimalFormat
public  getMultiplier : () ==> int
public  equals : JavaLangObject ==> bool
public  setMultiplier : int  ==> ()
public  getGroupingSize : () ==> int
public  setGroupingSize : int  ==> ()
public  getNegativePrefix : () ==> JavaLangString
public  getNegativeSuffix : () ==> JavaLangString
public  getPositivePrefix : () ==> JavaLangString
public  getPositiveSuffix : () ==> JavaLangString
public  JavaTextDecimalFormat : <NIL> ==> JavaTextDecimalFormat
public  toLocalizedPattern : () ==> JavaLangString
public  applyPattern : JavaLangString ==> ()
public  JavaTextDecimalFormat : 
          JavaLangString ==> JavaTextDecimalFormat
public  getDecimalFormatSymbols : 
          () ==> JavaTextDecimalFormatSymbols
public  setMaximumIntegerDigits : int  ==> ()
public  setMinimumIntegerDigits : int  ==> ()
public  setMaximumFractionDigits : int  ==> ()
public  setMinimumFractionDigits : int  ==> ()
public  setNegativePrefix : JavaLangString ==> ()
public  setNegativeSuffix : JavaLangString ==> ()
public  setPositivePrefix : JavaLangString ==> ()
public  setPositiveSuffix : JavaLangString ==> ()
public  applyLocalizedPattern : JavaLangString ==> ()
public  isDecimalSeparatorAlwaysShown : () ==> bool
public  parse : JavaLangString * JavaTextParsePosition ==> JavaLangNumber
public  setDecimalSeparatorAlwaysShown : bool  ==> ()
public  format'' : real  * JavaLangStringBuffer * 
                           JavaTextFieldPosition ==> JavaLangStringBuffer
public  JavaTextDecimalFormat : JavaLangString * 
                   JavaTextDecimalFormatSymbols ==> JavaTextDecimalFormat
public  setDecimalFormatSymbols : JavaTextDecimalFormatSymbols ==> ()
\end{verbatim}
\end{small}

\subsubsection{\texttt{JavaTextDecimalFormatSymbols}}
\begin{small}
\begin{verbatim}
public  clone : () ==> JavaLangObject
public  getNaN : () ==> JavaLangString
public  getClass : () ==> JavaLangClass
public  getDigit : () ==> char
public  hashCode : () ==> int
public  setDigit : char  ==> ()
public  getPerMill : () ==> char
public  getPercent : () ==> char
public  getInfinity : () ==> JavaLangString
public  setPerMill : char  ==> ()
public  setPercent : char  ==> ()
public  getMinusSign : () ==> char
public  getZeroDigit : () ==> char
public  setMinusSign : char  ==> ()
public  setZeroDigit : char  ==> ()
public  equals : JavaLangObject ==> bool
public  setNaN : JavaLangString ==> ()
public  getCurrencySymbol : () ==> JavaLangString
public  getDecimalSeparator : () ==> char
public  getPatternSeparator : () ==> char
public  setInfinity : JavaLangString ==> ()
public  JavaTextDecimalFormatSymbols : () ==> JavaTextDecimalFormatSymbols
public  getGroupingSeparator : () ==> char
public  setDecimalSeparator : char  ==> ()
public  setPatternSeparator : char  ==> ()
public  setGroupingSeparator : char  ==> ()
public  JavaTextDecimalFormatSymbols : 
          <NIL> ==> JavaTextDecimalFormatSymbols
public  setCurrencySymbol : JavaLangString ==> ()
public  getMonetaryDecimalSeparator : () ==> char
public  JavaTextDecimalFormatSymbols : 
          JavaUtilLocale ==> JavaTextDecimalFormatSymbols
public  setMonetaryDecimalSeparator : char  ==> ()
public  getInternationalCurrencySymbol : () ==> JavaLangString
public  setInternationalCurrencySymbol : JavaLangString ==> ()
public  getClass : () ==> JavaLangClass
\end{verbatim}
\end{small}

\subsubsection{\texttt{JavaTextFieldPosition}}
\begin{small}
\begin{verbatim}
public  getField : () ==> int
public  hashCode : () ==> int
public  toString : () ==> JavaLangString
public  getEndIndex : () ==> int
public  setEndIndex : int  ==> ()
public  getBeginIndex : () ==> int
public  JavaTextFieldPosition : int  ==> JavaTextFieldPosition
public  equals : JavaLangObject ==> bool
public  setBeginIndex : int  ==> ()
public  JavaTextFieldPosition : <NIL> ==> JavaTextFieldPosition
\end{verbatim}
\end{small}

\subsubsection{\texttt{JavaTextFormat}}
\begin{small}
\begin{verbatim}
public  clone : () ==> JavaLangObject
public  getClass : () ==> JavaLangClass
public  JavaTextFormat : <NIL> ==> JavaTextFormat
public  formatObject : JavaLangObject ==> JavaLangString
public  parseObject : JavaLangString ==> JavaLangObject
public  parseObject : 
          JavaLangString * JavaTextParsePosition ==> JavaLangObject
public  format' : JavaLangObject * JavaLangStringBuffer * 
                     JavaTextFieldPosition ==> JavaLangStringBuffer
\end{verbatim}
\end{small}

\subsubsection{\texttt{JavaTextMessageFormat}}
\begin{small}
\begin{verbatim}
public  clone : () ==> JavaLangObject
public  getClass : () ==> JavaLangClass
public  hashCode : () ==> int
public  getLocale : () ==> JavaUtilLocale
public  toPattern : () ==> JavaLangString
public  getFormats : () ==> map  int  to  JavaTextFormat
public  parse : JavaLangString ==> map  int  to  JavaLangObject
public  equals : JavaLangObject ==> bool
public  setLocale : JavaUtilLocale ==> ()
public  JavaTextMessageFormat : <NIL> ==> JavaTextMessageFormat
public  setFormat : int  * JavaTextFormat ==> ()
public  applyPattern : JavaLangString ==> ()
public  JavaTextMessageFormat : JavaLangString ==> JavaTextMessageFormat
public  setFormats : map  int  to  JavaTextFormat ==> ()
public  static  format : 
        JavaLangString * map  int  to  JavaLangObject ==> JavaLangString
public  parse : JavaLangString * JavaTextParsePosition ==> 
                                             map  int  to  JavaLangObject
public  JavaTextMessageFormat : JavaLangString * JavaUtilLocale ==> 
                                                    JavaTextMessageFormat
public  parseObject : JavaLangString * JavaTextParsePosition ==> 
                                                           JavaLangObject
public  format' : JavaLangObject * JavaLangStringBuffer * 
                           JavaTextFieldPosition ==> JavaLangStringBuffer
public  formatObjects : map  int  to  JavaLangObject * 
    JavaLangStringBuffer * JavaTextFieldPosition ==> JavaLangStringBuffer
\end{verbatim}
\end{small}

\subsubsection{\texttt{JavaTextNumberFormat}}
\begin{small}
\begin{verbatim}
public  static  INTEGER_FIELD : int  := 0;
public  static  FRACTION_FIELD : int  := 1;
public  static  cachedLocaleData : 
          JavaUtilHashtable := new  JavaUtilHashtable(3);
public  static  NUMBERSTYLE : int  := 0;
public  static  CURRENCYSTYLE : int  := 1;
public  static  PERCENTSTYLE : int  := 2;
public  static  SCIENTIFICSTYLE : int  := 3;
public  clone : () ==> JavaLangObject
public  format : real  ==> JavaLangString
public  getClass : () ==> JavaLangClass
public  hashCode : () ==> int
public  static  getInstance : () ==> JavaTextNumberFormat
public  parse : JavaLangString ==> JavaLangNumber
public  equals : JavaLangObject ==> bool
public  isGroupingUsed : () ==> bool
public  setGroupingUsed : bool  ==> ()
public  JavaTextNumberFormat : <NIL> ==> JavaTextNumberFormat
public  static getNumberInstance : () ==> JavaTextNumberFormat
public  static getPercentInstance : () ==> JavaTextNumberFormat
public  isParseIntegerOnly : () ==> bool
public  static getAvailableLocales : () ==> map  int  to  JavaUtilLocale
public  static getCurrencyInstance : () ==> JavaTextNumberFormat
public  static getInstance : JavaUtilLocale ==> JavaTextNumberFormat
public  setParseIntegerOnly : bool  ==> ()
public  static getInstance : 
          JavaUtilLocale * int  ==> JavaTextNumberFormat
public  getMaximumIntegerDigits : () ==> int
public  getMinimumIntegerDigits : () ==> int
public  getMaximumFractionDigits : () ==> int
public  getMinimumFractionDigits : () ==> int
public  setMaximumIntegerDigits : int  ==> ()
public  setMinimumIntegerDigits : int  ==> ()
public  static  getNumberInstance : 
          JavaUtilLocale ==> JavaTextNumberFormat
public  setMaximumFractionDigits : int  ==> ()
public  setMinimumFractionDigits : int  ==> ()
public  static  getPercentInstance : 
          JavaUtilLocale ==> JavaTextNumberFormat
public  static  getCurrencyInstance : 
          JavaUtilLocale ==> JavaTextNumberFormat
public  parse : 
          JavaLangString * JavaTextParsePosition ==> JavaLangNumber
public  parseObject : 
          JavaLangString * JavaTextParsePosition ==> JavaLangObject
public  format'' : real  * JavaLangStringBuffer * 
                      JavaTextFieldPosition ==> JavaLangStringBuffer
public  format' : JavaLangObject * JavaLangStringBuffer * 
                      JavaTextFieldPosition ==> JavaLangStringBuffer
\end{verbatim}
\end{small}

\subsubsection{\texttt{JavaTextParseException}}
\begin{small}
\begin{verbatim}
public  getClass : () ==> JavaLangClass
public  getErrorOffset : () ==> int
public  JavaTextParseException : <NIL> ==> JavaTextParseException
public  JavaTextParseException : 
          JavaLangString * int  ==> JavaTextParseException
\end{verbatim}
\end{small}

\subsubsection{\texttt{JavaTextParsePosition}}
\begin{small}
\begin{verbatim}
public index : int  := 0;
public errorIndex : int  := -1
public  getClass : () ==> JavaLangClass
public  getIndex : () ==> int
public  hashCode : () ==> int
public  toString : () ==> JavaLangString
public  setIndex : int  ==> ()
public  getErrorIndex : () ==> int
public  JavaTextParsePosition : int  ==> JavaTextParsePosition
public  equals : JavaLangObject ==> bool
public  setErrorIndex : int  ==> ()
public  JavaTextParsePosition : <NIL> ==> JavaTextParsePosition
\end{verbatim}
\end{small}

\subsubsection{\texttt{JavaTextSimpleDateFormat}}
\begin{small}
\begin{verbatim}
public  getClass : () ==> JavaLangClass
public  JavaTextSimpleDateFormat : <NIL> ==> JavaTextSimpleDateFormat
public  JavaTextSimpleDateFormat : 
          JavaLangString ==> JavaTextSimpleDateFormat
\end{verbatim}
\end{small}

\subsection{\texttt{java.sql}}

\subsubsection{\texttt{JavaSqlConnection}}
\begin{small}
\begin{verbatim}
public  static  TRANSACTION_NONE : int :=0;
public  static  TRANSACTION_READ_UNCOMMITTED : int :=0;
public  static  TRANSACTION_READ_COMMITTED : int :=0;
public  static  TRANSACTION_REPEATABLE_READ : int :=0;
public  static  TRANSACTION_SERIALIZABLE : int :=0
public  JavaSqlConnection : <NIL> ==> JavaSqlConnection
public  close : () ==> ()
public  commit : () ==> ()
public  getClass : () ==> JavaLangClass
public  isClosed : () ==> bool
public  rollback : () ==> ()
public  getCatalog : () ==> JavaLangString
public  getTypeMap : () ==> JavaUtilMap
public  isReadOnly : () ==> bool
public  setReadOnly : bool  ==> ()
public  clearWarnings : () ==> ()
public  getAutoCommit : () ==> bool
public  setAutoCommit : bool  ==> ()
public  createStatement : () ==> JavaSqlStatement
public  nativeSQL : JavaLangString ==> JavaLangString
public  createStatement : int  * int  ==> JavaSqlStatement
public  setCatalog : JavaLangString ==> ()
public  getTransactionIsolation : () ==> int
public  setTransactionIsolation : int  ==> ()
\end{verbatim}
\end{small}

\subsubsection{\texttt{JavaSqlDriverManager}}
\begin{small}
\begin{verbatim}
public  getClass : () ==> JavaLangClass
public  static  getConnection : JavaLangString ==> JavaSqlConnection
public  static  getConnection : JavaLangString * JavaLangString * 
                                   JavaLangString ==> JavaSqlConnection
\end{verbatim}
\end{small}

\subsubsection{\texttt{JavaSqlResultSet}}
\begin{small}
\begin{verbatim}
public  static  FETCH_FORWARD : int := 1000;
public  static  FETCH_REVERSE : int := 1001;
public  static  FETCH_UNKNOWN : int := 1002;
public  static  TYPE_FORWARD_ONLY : int := 1003;
public  static  TYPE_SCROLL_INSENSITIVE : int := 1004;
public  static  TYPE_SCROLL_SENSITIVE : int := 1005;
public  static  CONCUR_READ_ONLY : int := 1007;
public  static  CONCUR_UPDATABLE : int := 1008
public  last : () ==> bool
public  next : () ==> bool
public  close : () ==> ()
public  first : () ==> bool
public  getRow : () ==> int
public  isLast : () ==> bool
public  getInt : int  ==> int
public  getType : () ==> int
public  isFirst : () ==> bool
public  wasNull : () ==> bool
public  getByte : int  ==> int
public  getClass : () ==> JavaLangClass
public  getLong : int  ==> int
public  previous : () ==> bool
public  absolute : int  ==> bool
public  afterLast : () ==> ()
public  deleteRow : () ==> ()
public  getBytes : int  ==> map  int  to  int
public  getFloat : int  ==> real
public  getShort : int  ==> int
public  insertRow : () ==> ()
public  relative : int  ==> bool
public  updateRow : () ==> ()
public  getDouble : int  ==> real
public  getString : int  ==> JavaLangString
public  refreshRow : () ==> ()
public  rowDeleted : () ==> bool
public  rowUpdated : () ==> bool
public  beforeFirst : () ==> ()
public  getBoolean : int  ==> bool
public  isAfterLast : () ==> bool
public  rowInserted : () ==> bool
public  updateNull : int  ==> ()
public  getFetchSize : () ==> int
public  getStatement : () ==> JavaSqlStatement
public  updateInt : int  * int  ==> ()
public  clearWarnings : () ==> ()
public  getCursorName : () ==> JavaLangString
public  isBeforeFirst : () ==> bool
public  setFetchSize : int  ==> ()
public  updateByte : int  * int  ==> ()
public  updateLong : int  * int  ==> ()
public  getConcurrency : () ==> int
public  getInt : JavaLangString ==> int
public  updateFloat : int  * real  ==> ()
public  updateShort : int  * int  ==> ()
public  getByte : JavaLangString ==> int
public  getLong : JavaLangString ==> int
public  moveToInsertRow : () ==> ()
public  updateDouble : int  * real  ==> ()
public  cancelRowUpdates : () ==> ()
public  getBytes : JavaLangString ==> map  int  to  int
public  getFloat : JavaLangString ==> real
public  getShort : JavaLangString ==> int
public  moveToCurrentRow : () ==> ()
public  updateBoolean : int  * bool  ==> ()
public  getDouble : JavaLangString ==> real
public  getFetchDirection : () ==> int
public  getObject : JavaLangString ==> JavaLangObject
public  getString : JavaLangString ==> JavaLangString
public  findColumn : JavaLangString ==> int
public  getBoolean : JavaLangString ==> bool
public  setFetchDirection : int  ==> ()
public  updateBytes : int  * map  int  to  int  ==> ()
public  updateNull : JavaLangString ==> ()
public  updateInt : JavaLangString * int  ==> ()
public  updateByte : JavaLangString * int  ==> ()
public  updateLong : JavaLangString * int  ==> ()
public  updateFloat : JavaLangString * real  ==> ()
public  updateShort : JavaLangString * int  ==> ()
public  updateDouble : JavaLangString * real  ==> ()
public  updateObject : int  * JavaLangObject ==> ()
public  updateString : int  * JavaLangString ==> ()
public  updateBoolean : JavaLangString * bool  ==> ()
public  updateBytes : JavaLangString * map  int  to  int  ==> ()
public  updateString : JavaLangString * JavaLangString ==> ()
\end{verbatim}
\end{small}

\subsubsection{\texttt{JavaSqlSQLException}}
\begin{small}
\begin{verbatim}
public  getClass : () ==> JavaLangClass
public  getSQLState : () ==> JavaLangString
public  JavaSqlSQLException : () ==> JavaSqlSQLException
public  getErrorCode : () ==> int
public  getNextException : () ==> JavaSqlSQLException
public  JavaSqlSQLException : <NIL> ==> JavaSqlSQLException
public  JavaSqlSQLException : JavaLangString ==> JavaSqlSQLException
public  JavaSqlSQLException : 
          JavaLangString * JavaLangString ==> JavaSqlSQLException
public  setNextException : JavaSqlSQLException ==> ()
public  JavaSqlSQLException : JavaLangString * 
                        JavaLangString * int  ==> JavaSqlSQLException
\end{verbatim}
\end{small}

\subsubsection{\texttt{SqlConnection}}
\begin{small}
\begin{verbatim}
public  static  TRANSACTION_NONE : int := 0;
public  static  TRANSACTION_READ_UNCOMMITTED : int := 0;
public  static  TRANSACTION_READ_COMMITTED : int := 0;
public  static  TRANSACTION_REPEATABLE_READ : int := 0;
public  static  TRANSACTION_SERIALIZABLE : int := 0
public  close : () ==> ()
public  commit : () ==> ()
public  getClass : () ==> JavaLangClass
public  isClosed : () ==> bool
public  rollback : () ==> ()
public  getCatalog : () ==> JavaLangString
public  getTypeMap : () ==> JavaUtilMap
public  isReadOnly : () ==> bool
public  setReadOnly : bool  ==> ()
public  clearWarnings : () ==> ()
public  getAutoCommit : () ==> bool
public  setAutoCommit : bool  ==> ()
public  createStatement : () ==> JavaSqlStatement
public  nativeSQL : JavaLangString ==> JavaLangString
public  createStatement : int  * int  ==> JavaSqlStatement
public  setCatalog : JavaLangString ==> ()
public  getTransactionIsolation : () ==> int
public  setTransactionIsolation : int  ==> ()
\end{verbatim}
\end{small}

\subsubsection{\texttt{SqlResultSet}}
\begin{small}
\begin{verbatim}
public  getColumns : () ==> map seq of char to int * int * int * int
public  last : () ==> bool
public  next : () ==> bool
public  close : () ==> ()
public  first : () ==> bool
public  getRow : () ==> int
public  isLast : () ==> bool
public  getInt : int  ==> int
public  getType : () ==> int
public  isFirst : () ==> bool
public  wasNull : () ==> bool
public  getByte : int  ==> int
public  getClass : () ==> JavaLangClass
public  getLong : int  ==> int
public  previous : () ==> bool
public  absolute : int  ==> bool
public  afterLast : () ==> ()
public  deleteRow : () ==> ()
public  getBytes : int  ==> map  int  to  int
public  getFloat : int  ==> real
public  getShort : int  ==> int
public  insertRow : () ==> ()
public  relative : int  ==> bool
public  updateRow : () ==> ()
public  getDouble : int  ==> real
public  getString : int  ==> JavaLangString
public  refreshRow : () ==> ()
public  rowDeleted : () ==> bool
public  rowUpdated : () ==> bool
public  beforeFirst : () ==> ()
public  getBoolean : int  ==> bool
public  isAfterLast : () ==> bool
public  rowInserted : () ==> bool
public  updateNull : int  ==> ()
public  getFetchSize : () ==> int
public  getStatement : () ==> JavaSqlStatement
public  updateInt : int  * int  ==> ()
public  clearWarnings : () ==> ()
public  getCursorName : () ==> JavaLangString
public  isBeforeFirst : () ==> bool
public  setFetchSize : int  ==> ()
public  updateByte : int  * int  ==> ()
public  updateLong : int  * int  ==> ()
public  getConcurrency : () ==> int
public  getInt : JavaLangString ==> int
public  updateFloat : int  * real  ==> ()
public  updateShort : int  * int  ==> ()
public  getByte : JavaLangString ==> int
public  getLong : JavaLangString ==> int
public  moveToInsertRow : () ==> ()
public  updateDouble : int  * real  ==> ()
public  cancelRowUpdates : () ==> ()
public  getBytes : JavaLangString ==> map  int  to  int
public  getFloat : JavaLangString ==> real
public  getShort : JavaLangString ==> int
public  moveToCurrentRow : () ==> ()
public  updateBoolean : int  * bool  ==> ()
public  getDouble : JavaLangString ==> real
public  getFetchDirection : () ==> int
public  getObject : JavaLangString ==> JavaLangObject
public  getString : JavaLangString ==> JavaLangString
public  findColumn : JavaLangString ==> int
public  getBoolean : JavaLangString ==> bool
public  setFetchDirection : int  ==> ()
public  updateBytes : int  * map  int  to  int  ==> ()
public  updateNull : JavaLangString ==> ()
public  updateInt : JavaLangString * int  ==> ()
public  updateByte : JavaLangString * int  ==> ()
public  updateLong : JavaLangString * int  ==> ()
public  updateFloat : JavaLangString * real  ==> ()
public  updateShort : JavaLangString * int  ==> ()
public  updateDouble : JavaLangString * real  ==> ()
public  updateObject : int  * JavaLangObject ==> ()
public  updateString : int  * JavaLangString ==> ()
public  updateBoolean : JavaLangString * bool  ==> ()
public  updateBytes : JavaLangString * map  int  to  int  ==> ()
public  updateString : JavaLangString * JavaLangString ==> ()
\end{verbatim}
\end{small}

\subsubsection{\texttt{SqlStatement}}
\begin{small}
\begin{verbatim}
public  close : () ==> ()
public  getClass : () ==> JavaLangClass
public  executeQuery : JavaLangString ==> JavaSqlResultSet
public  executeUpdate : JavaLangString ==> int
\end{verbatim}
\end{small}

\subsubsection{\texttt{JavaSqlStatement}}
\begin{small}
\begin{verbatim}
public close : () ==> ()
public  cancel : () ==> ()
public  getClass : () ==> JavaLangClass
public  clearBatch : () ==> ()
public  getMaxRows : () ==> int
public  setMaxRows : int  ==> ()
public  executeBatch : () ==> map  int  to  int
public  getFetchSize : () ==> int
public  getResultSet : () ==> JavaSqlResultSet
public  clearWarnings : () ==> ()
public  getConnection : () ==> JavaSqlConnection
public  setFetchSize : int  ==> ()
public  getMoreResults : () ==> bool
public  getUpdateCount : () ==> int
public  execute : JavaLangString ==> bool
public  getMaxFieldSize : () ==> int
public  getQueryTimeout : () ==> int
public  addBatch : JavaLangString ==> ()
public  getResultSetType : () ==> int
public  setMaxFieldSize : int  ==> ()
public  setQueryTimeout : int  ==> ()
public  getFetchDirection : () ==> int
public  setFetchDirection : int  ==> ()
public  executeQuery : JavaLangString ==> JavaSqlResultSet
public  setEscapeProcessing : bool  ==> ()
public  executeUpdate : JavaLangString ==> int
public  setCursorName : JavaLangString ==> ()
public  getResultSetConcurrency : () ==> int
\end{verbatim}
\end{small}

\appendix

\newpage
\bibliographystyle{iptes}
\bibliography{ifad}

\newpage
\addcontentsline{toc}{section}{Index}
\printindex
\end{document}
