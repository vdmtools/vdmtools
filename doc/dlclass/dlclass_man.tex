%%
%% Toolbox DL Classes Manual
%% $Id: dlclass_man.tex,v 1.9 2006/04/19 10:24:40 vdmtools Exp $
%% 

%%%%%%%%%%%%%%%%%%%%%%%%%%%%%%%%%%%%%%%%
% PDF compatibility code. 
\makeatletter
\newif\ifpdflatex@
\ifx\pdftexversion\@undefined
\pdflatex@false
%\message{Not using pdf}
\else
\pdflatex@true
%\message{Using pdf}
\fi

\newcommand{\latexpdf}[2]{
  \ifpdflatex@ #1
  \else #2
  \fi
}

\newcommand{\latexorpdf}[2]{
  \ifpdflatex@ #2
  \else #1
  \fi
}

\makeatother

#ifdef A4Format
\newcommand{\pformat}{a4paper}
#endif A4Format
#ifdef LetterFormat
\newcommand{\pformat}{letterpaper}
#endif LetterFormat

%%%%%%%%%%%%%%%%%%%%%%%%%%%%%%%%%%%%%%%%

\latexorpdf{
\documentclass[\pformat,12pt]{article}
}{
% pdftex option is used by graphic[sx],hyperref,toolbox.sty
\documentclass[\pformat,pdftex,12pt]{article}
}

\usepackage{toolbox}
\usepackage{vdmsl-2e}
\usepackage{makeidx}
\usepackage{alltt}
\usepackage{verbatim}
\usepackage{moreverb}

% Ueki change start
\AtBeginDvi{\special{pdf:tounicode 90ms-RKSJ-UCS2}}
\usepackage[dvipdfm,bookmarks=true,bookmarksnumbered=true,colorlinks,plainpages=true]{hyperref}
\usepackage{cite}

\def\seename{$\Rightarrow$}
% Ueki change end

% Ueki delete start
%\latexorpdf{
%\usepackage[plainpages=true,colorlinks,linkcolor=black,citecolor=black,pagecolor=black, urlcolor=black]{hyperref}
%}{
%\usepackage[plainpages=true,colorlinks]{hyperref}
%}
% Ueki delete end

\newcommand{\meti}[1]{\item[#1]\mbox{}\\}
\newcommand{\Lit}[1]{`#1\Quote}
\newcommand{\Rule}[2]{
  \begin{quote}\begin{tabbing}
    #1\index{#1}\ \ \= = \ \ \= #2  ; %    Adds production rule to index
  \end{tabbing}\end{quote}
  }

\newcommand{\SeqPt}[1]{\{\ #1\ \}}
\newcommand{\lfeed}{\\ \> \>}
\newcommand{\OptPt}[1]{[\ #1\ ]}
\newcommand{\dsepl}{\ $|$\ }
\newcommand{\dsep}{\\ \> $|$ \>}
\newcommand{\Lop}[1]{\Lit{\kw{#1}}}
\newcommand{\Sig}[1]{\Lit{{\tt #1}}}
\newcommand{\blankline}{\vspace{\baselineskip}}
\newcommand{\Brack}[1]{(\ #1\ )}
\newcommand{\nmk}{\footnotemark}
\newcommand{\ntext}[1]{\footnotetext{{\bf Note: } #1}}


%\usepackage[dvips]{color}
\usepackage{longtable}
\definecolor{covered}{rgb}{0,0,0}     %black
\definecolor{not_covered}{gray}{0.5} %gray

\parindent0mm

\newlength{\keywwidth}

\newcommand{\xfigpicture}[4]{
\begin{figure}[hbt]
\setlength{\unitlength}{1mm}
\begin{center}
\mbox{
\begin{picture}(#1,#2)
\put(0,0){\special{psfile=#3 hscale=70 vscale=55}}
\end{picture} }
\end{center}
\caption{#4}
\end{figure}
}


\newcommand{\vdmslpp}{VDM-SL}
\newcommand{\Toolbox}{Toolbox}
\newcommand{\aaa}{\tt }
\newcommand{\cmd}{\tt }
\newcommand{\id}[1]{%
\settowidth{\keywwidth}{\tt #1}%
\protect\makebox[\keywwidth][l]{{\it #1}}}
\nolinenumbering


\newcommand{\vdmcpplib}{\textit{VDM C++ Library}}

\begin{document}
\vdmtoolsmanualcsk{The Dynamic Link Facility for VDM++} 
       {v9.0.6}
       {2016}
       {VDM++}
       {1.0}


\section{Introduction}

This document is an extension to the {\it User Manual for the
 VDM++ Toolbox} \cite{UserManPP-CSK}. Knowledge
about C++ \cite{Stroustrup91} and The {\vdmcpplib}
\cite{LibMan-CSK} is also assumed. Throughout the document we use the term
``DL Classes''~(``dynamic link classes'') to mean VDM++ classes for
which external code is to be executed. 

You do not need to read this entire manual to get started using the
Dynamic Link feature. Start by reading Section~\ref{sec:overview}
to get an overall description of this
feature. Section~\ref{sec:dlexample} shows an example of using the
feature; all files from this example are also present in the
appendices. Section~\ref{sec:codegen} demonstrates how code is
generated from DL Classes by the C++ Code Generator.
Section~\ref{sec:refguide} describes the individual components which are
involved when using the Dynamic Link facility. 

Appendix~\ref{sec:sysreq} provides detailed information about system
requirements.




\section{Overview}\label{sec:overview}

In this section an overview of the use of DL classes is presented. The
idea here is to give an informal idea of how to use the facility. More
detailed information is presented in the Reference Guide in
Section~\ref{sec:refguide}. 

A diagram showing the approach is given in Figure~\ref{idea}. 
\begin{figure}
\begin{center}
\includegraphics{overview.mps}
\caption{Combination of code and VDM++ specification\label{idea}}
\end{center}
\end{figure}
As can be seen from this diagram, a number of different components
need to be provided to enable the facility:
\begin{itemize}
\item For each C++ class which is to be accessed at the VDM++ level, a
DL Class must be written at the VDM++ level.
\item An interface layer needs to be written in C++. This translates
method calls at the VDM++ level into calls to the external code and
translates the results into {\vdmcpplib} types. This should be compiled as a
shared library (on Windows, a \texttt{.dll}).
\end{itemize}
Each of these components is described in more detail below.


\subsection{Specifying a DL Class in VDM++}

To specify a class as being a DL class in VDM++, the keyword
\textsf{dlclass} should be used instead of \textsf{class}. The DL
Class must contain a directive indicating the location of the
shared library implementing the DL class. This is specified using the
\textsf{uselib} directive. Thus a simple DL class might be

\begin{alltt}
\textsf{dlclass} EmptyDLClass

\textsf{uselib} "EmptyDLClass.so"\footnote{Note that shared libraries
  have file type \texttt{.so} on Unix whereas on Windows they are of
  type \texttt{.dll}} 

\textsf{end} EmptyDLClass
\end{alltt}

Of course, this is not a very interesting class since it contains no
functionality! To expose methods in the external code, methods with
body \textsf{is not yet specified} should be defined at the VDM++
level. For example:

\begin{alltt}
\textsf{dlclass} SimpleDLClass

\textsf{uselib} "SimpleDLClass.so"

\textsf{operations}

\textsf{public} CallExternal : nat ==> nat
CallExternal (n) == \textsf{is not yet specified};

\textsf{end} SimpleDLClass
\end{alltt}
Calls to \texttt{SimpleDLClass`CallExternal} will then, via the shared
library, be directed to the external code corresponding to
\texttt{SimpleDLClass}. 

\subsection{Interfacing with a DL Class}

The user must provide an interface between the VDM++ DL Class and the
external code to which the DL Class corresponds. This interface is
referred to as the interface layer and falls into three parts:
\begin{enumerate}
\item Functions for creating and deleting external objects must be
provided. These are called respectively \texttt{DlClass\_new} and
\texttt{DlClass\_delete}; 
\item A function for redirecting method calls must be defined, which
  is called \texttt{DlClass\_call};
\item Each external class must be wrapped as a subclass of the class
\texttt{DlClass}. That is, every external class must implement the
\texttt{DlClass} interface.
\end{enumerate}
The prototypes for these functions and classes are defined in the file
\texttt{dlclass.h} which is supplied with the Toolbox
distribution. They can also be found in Appendix~\ref{app:dlclass_h}.  



\subsection{Creating a Shared Library}

Having created the interface layer, a shared library must be
constructed. This is done by compiling the interface layer and the
external code into a single shared library. On Unix using GNU g++, this
is achieved using the flags \texttt{-shared -fpic}. On Windows using
Microsoft Visual C++ it is achieved using the link flags
\texttt{/dll /incremental:no}. In both cases the VDM C++ library must
be linked in since the interface layer must use the types defined in
this library. See the example Makefiles that are provided in
Appendices~\ref{app:unixmake} and~\ref{app:winmake}. 

\section{Example}\label{sec:dlexample}

In this section we present portions of a small example of how to
write a DL Class and interface layer code. The example is based on an
external library -- ``MAPM'' -- for arbitrary precision mathematical
calculations. This library is freely available for download from 
\texttt{http://www.tc.umn.edu/$\sim$ringx004/mapm-main.html}.

Since the VDM++ interpreter uses finite precision arithmetic, it is
important that infinite precision mathematical objects are never
handled directly by the Toolbox since by doing so the precision of
the mathematical values will be compromised. 

\subsection{The VDM++ Model}

We define a class \texttt{BigInt} to represent arbitrary precision
integers. This defines operations to create \texttt{BigInt}s, to
convert a texttt{BigInt} to a string representation, and to perform
addition on \texttt{BigInt}s. (Note that the \texttt{SetVal} operation is
only used to create an initial value.) The class is connected to an
interface layer called \texttt{bigint\_dl.so}. 

\begin{alltt}
\textsf{dlclass} BigInt

\textsf{uselib} "bigint_dl.so"

\textsf{operations}

\textsf{public} Create : \textsf{nat} ==> BigInt
Create(n) ==
( SetVal(n);
  \textsf{return} self
);

\textsf{public} toString : () ==> \textsf{seq} \textsf{of} \textsf{char}
toString() ==
  \textsf{is} \textsf{not} \textsf{yet} \textsf{specified};

\textsf{public} plus : BigInt ==> BigInt
plus(i) ==
  \textsf{is} \textsf{not} \textsf{yet} \textsf{specified};

\textsf{protected} SetVal : \textsf{nat} ==> ()
SetVal(n) ==
  \textsf{is} \textsf{not} \textsf{yet} \textsf{specified};

\textsf{end} BigInt
\end{alltt}

We can then use \texttt{BigInt} in other classes in our
specification. For example, the class \texttt{BankAccount} uses 
\texttt{BigInt} as the type of one of its instance variables. As shown
by the \texttt{Init} and \texttt{Deposit} operations, manipulation of
this object is performed via operations 
rather than directly.

\begin{alltt}
\textsf{class} BankAccount

\textsf{instance} \textsf{variables}
  name : \textsf{seq} \textsf{of} \textsf{char};
  number : \textsf{nat};
  balance : BigInt

\textsf{operations}

\textsf{public} Init : \textsf{seq} \textsf{of} \textsf{char} * \textsf{nat} ==> ()
Init(nname, nnum) ==
( name := nname;
  number := nnum;
  balance := new BigInt().Create(nnum);
);

\textsf{public} Deposit : BigInt ==> ()
Deposit(bi) ==
  balance := balance.plus(bi);

\textsf{public} GetBalance : () ==> BigInt
GetBalance() ==
  \textsf{return} balance

\textsf{end} BankAccount
\end{alltt}


\subsection{Interface Layer}

The interface layer maps calls at the VDM++ level to C++ function
calls. The complete files are included in
Appendix~\ref{app:interfacelayer}. Here we describe selected excerpts.

For each VDM++ \texttt{BigInt} object, there is a corresponding C++ \texttt{BigIntDL}
object. This is linked to the external MAPM library via a member
variable: 
\begin{verbatim}
class BigIntDL : public DlClass
{

  MAPM val;
 
 public:
 ...
};
\end{verbatim}



To see how communication between the VDM++ level and the C++ level is
mediated, consider the class function \texttt{toString} which returns
a string representation of a \texttt{BigInt}.

\begin{alltt}
Sequence BigIntDL::toString()
\{
\#ifdef DEBUG
  cout << "BigIntDL::toString" << endl;
\#endif //DEBUG
  char res[100];
  val.toIntegerString(res);
  return Sequence(string(res));
\}
\end{alltt}
This function uses the library's own \texttt{toIntegerString}
function. Note that since the value is to be returned to the VDM++
Toolbox, a {\vdmcpplib} value must be returned by the function. 

More interesting is the case where VDM++ objects are passed as
parameters and/or returned as result. In this case, the {\vdmcpplib}
class \texttt{DLObject} should be used. For example, consider the
operation of addition:
\begin{alltt}
DLObject BigIntDL::plus (Sequence &p)
\{
\#ifdef DEBUG
  cout << "BigIntDL::plus" << endl;
\#endif //DEBUG
  DLObject result("BigInt", new BigIntDL);
  DLObject arg (p.Hd());
  BigIntDL *resPtr = (BigIntDL*) result.GetPtr(),
           *argPtr = (BigIntDL*) arg.GetPtr(); 
  resPtr->setVal( val + argPtr->getVal());
  return result;
\}
\end{alltt}
Here, a new \texttt{DLObject} is created, as an instance of \texttt{BigInt}, to store
the result of the operation. The argument object is extracted from the
parameter list the pointers to the \texttt{BigIntDL} objects are extracted
from the \texttt{DLObjects}, and the result is computed. Finally the
result object is returned. 

Objects created and returned in this way are subject to the VDM++
Toolbox's normal object maintenance. Thus they will persist until no
references exist to them at the VDM++ level, at which point
\texttt{DlClass\_delete} will be called on such objects.



\section{Using DL Classes in combination with the C++ Code Generator}\label{sec:codegen}

DL Classes are treated by the code generator in such a manner as to
allow virtually seamless use of generated code with the interface
layer and external code. Of course, user implementations for VDM++
operations specified as \textsf{is not yet specified} can be provided
in the usual way, thus allowing direct communication with the external
code rather than via the interface layer. In this section we
illustrate both of these approaches.

\subsection{Using Generated Code}

The generated code uses the same calling structure as used by DL
Classes. To allow this, each class has a public member which is a
pointer to a DL Class:
\begin{verbatim}
class vdm_BigInt : public virtual CGBase {
...
public:
  DlClass * BigInt_dlClassPtr;
}
\end{verbatim}
C++ function calls corresponding to operation or function calls to DL
Classes are then code generated using this pointer. For example,
consider the operation \texttt{toString} from
Section~\ref{sec:dlexample}. This would be code generated as a call to
\texttt{DlClass\_call} using \texttt{BigInt\_dlClassPtr}.
\begin{alltt}
\#ifndef DEF_BigInt_toString
 
type_cL vdm_BigInt::vdm_toString () \{  Sequence parmSeq_2;
  int success_3;
 
  return DlClass_call(BigInt_dlClassPtr, "toString", parmSeq_2, success_3);
 
\}
\#endif   
\end{alltt}
Thus calls within the generated code to
\texttt{vdm\_BigInt::vdm\_toString} will be seamlessly handled by the
interface layer.

For VDM++ functions or operations that pass or receive objects, the
situation is slightly more complicated. This is because the VDM++ Code
Generator uses \texttt{ObjectRef} from the {\vdmcpplib} to represent
objects passed between C++ functions. Therefore the code in the
interface layer needs to be able to deal with this. For example, we 
might modify \texttt{BigIntDL::plus} as follows:
\begin{alltt}
\#ifdef CG
ObjectRef BigIntDL::plus (const Sequence &p)
\#else
DLObject BigIntDL::plus (const Sequence &p)
\#endif //CG
\{
\#ifdef DEBUG
  cout << "BigIntDL::plus" << endl;
\#endif //DEBUG
  // Extract arguments
  BigIntDL *argPtr = GetDLPtr(p.Hd());

  // Set up result object
\#ifdef CG
  ObjectRef result (new vdm_BigInt);
\#else
  DLObject result("BigInt", new BigIntDL);
\#endif 

  BigIntDL *resPtr = GetDLPtr(result);

  // Perform manipulation on pointers, as needed for function
  resPtr->setVal( val + argPtr->getVal());

  return result;
\}
\end{alltt}
This definition exploits the function \texttt{BigIntDL::GetDLPtr}:
\begin{alltt}
\#ifdef CG
BigIntDL *BigIntDL::GetDLPtr(const ObjectRef& obj)
\#else
BigIntDL *BigIntDL::GetDLPtr(const DLObject& obj)
\#endif //CG
\{
\#ifdef CG
  vdm_BigInt *objRefPtr = ObjGet_vdm_BigInt(obj);
  BigIntDL *objPtr = (BigIntDL*) objRefPtr->BigInt_dlClassPtr;
\#else
  BigIntDL *objPtr = (BigIntDL*) obj.GetPtr(); 
\#endif
  return objPtr;
\}
\end{alltt}

\subsection{Supplying a User Implementation}

%\hyperlink{toastref}{toast}
Rather than communicate with the external code via the interface
layer, the generated code can be made to call the external code
directly. This is achieved using the standard mechanism for
substituting code in generated code, as described in \cite{CGManPP-CSK}. 

For instance, suppose we wish to call the addition
operation  directly. For this, we need to supply an implementation of
\texttt{vdm\_BigInt::vdm\_plus}. In the file
\texttt{BigInt\_userimpl.cc} the function would be defined:

\verbatiminput{dl-example/BigInt_userimpl.cc}

To ensure that the correct version of the function is compiled, the
file \texttt{BigInt\_userdef.h} needs to have the following contents: 

\begin{alltt}
\#define DEF_BigInt_USERIMPL
\#define DEF_BigInt_plus
\end{alltt}

When the files are compiled, the execution uses the user supplied
function instead of being directed to the interface layer.



\section{Reference Guide}\label{sec:refguide}

\subsection{DL Class Definitions}
A DL Class is specified using the syntax

\Rule{dlclass}{%
  \Lop{dlclass}, identifier, \OptPt{inheritance clause} \lfeed
   uselib clause, \lfeed
   \OptPt{class body}, \lfeed
  \Lop{end} identifier
}
\Rule{uselib clause}{%
  \Lop{uselib}, text literal
}
Note that the syntactic classes not defined here are defined in
\cite{LangManPP-CSK}. 
A DL Class is treated by the Toolbox just as any other class, with the
exception that any call to a function or operation with body
\textsf{is not yet specified} is redirected to the external code
specified in the uselib path. 



\subsection{The \texttt{uselib} Path Environment}
\label{sec:uselibpath}

The name of the library is given with the {\tt uselib} option in the
specification. This location can be given in several ways:
\begin{itemize}
\item A full path name for the library (e.g.\ 
  \texttt{/home/foo/libs/libmath.so})
\item Without path but with the environment variable {\tt
    VDM\_DYNLIB} set to a list of directories in which to search for
  the library. For example, the environment
  variable could look like this: \path+/home/foo/libs:/usr/lib:.+ (Note
  that if you set this variable you need a `.' in the list as in this
  example if you want the search to include the current directory.)
\item Without path and without the {\tt VDM\_DYNLIB} environment
  variable set. This means that the library is supposed to be located
  in the current directory.
\end{itemize}

\subsection{The Function \texttt{DlClass\_call}}
This function is used to redirect method calls at the VDM++ level to
appropriate method calls at the external code level. It takes as
parameters: 
\begin{description}
\item[\texttt{DlClass* c}] A pointer to the object on which the call
is being made;
\item[\texttt{const char* name}] The name of the method being called;
\item[\texttt{const Sequence\& params}] The parameters to the method call,
passed as {\vdmcpplib} values;
\item[\texttt{int\& success}] A reference parameter to a success flag.
Setting this flag to \texttt{0} indicates failure; setting it to
\texttt{1} indicates success.
\end{description}
The function returns the result of the method call in the form of a
{\vdmcpplib} \texttt{Generic} object.

It is the user's responsibility to provide an implementation of this
function. 
\subsection{The Function \texttt{DlClass\_delete}}
This function is called by the Toolbox whenever an instance of a DL
Class is to be destroyed. The Toolbox interpreter uses a garbage
collection scheme based on reference counting, so this function is called
whenever the reference count for an instance of a DL Class reaches 0. 
The function is also called when objects are destroyed as a result of
a \textsf{dlclose} call (see Section~\ref{sec:dlclose}).

It is the user's responsibility to provide an implementation of this
function. 

\subsection{The Function \texttt{DlClass\_new}}
This function is called by the Toolbox whenever an instance of a DL
Class is created. It takes as parameter a string representing the name
of the class being instantiated and returns a pointer to an instance
of \texttt{DlClass}. 

It is the user's responsibility to provide an
implementation of this function.

\subsection{The Class \texttt{DlClass}}
The class \texttt{DlClass} is an abstract class specifying a prototype
for one method: \texttt{DlMethodCall} which is used by a specific
class to translate method calls at the VDM++ level into method calls
on the external code. Note that indirection via \texttt{DlClass\_call}
is necessary because the interface called by the Toolbox interpreter
must be C code.

It is the user's responsibility to provide an implementation of a
subclass of this class for each external class which is being shadowed
at the VDM++ level.

\subsection{The DLObject Class}
The \texttt{DLObject} is an extension to the {\vdmcpplib} which allows
passing 
of objects between the Toolbox interpreter and the external code. Each
\texttt{DLObject} corresponds to an instance of DL Class and contains
\texttt{DlClass} pointer. The class has the following methods:
\begin{description}
\item[\texttt{DLObject(const string \& name, DlClass *object)}]
Constructor that creates a \texttt{DLObject} of the DL Class
called \texttt{name} which is initialized with a pointer to a
\texttt{DlClass} given by \texttt{object}.
\item[\texttt{DLObject(const DLObject \& dlobj)}]
Constructor that creates a new \texttt{DLObject} from an existing
one.
\item[\texttt{DLObject(const Generic \& dlobj)}]
Constructor used for narrowing a \texttt{Generic}.
\item[\texttt{DLObject \& operator=(const DLObject  \& dlobj)}]
Gives this object the value of \texttt{dlobj}.
\item[\texttt{DLObject \& operator=(const Generic \& gen)}]
Gives this object the value of \texttt{gen}.
\item[\texttt{string GetName() const}]
Returns the name of the VDM++ DL Class that this object corresponds to.
\item[\texttt{DlClass * GetPtr() const}]
Returns the pointer to the external instance of \texttt{DlClass}. 
\item[\texttt{DLObject::~DLObject()}]
Destructor.
\end{description}


\subsection{Opening Libraries}

Whenever the Toolbox command \textsf{init} is used all shared
libraries corresponding to DL Classes are opened. These remain open
until either a \textsf{dlclose} command is issued (see
Section~\ref{sec:dlclose}) or \textsf{init} is called again. In the
latter case the currently open libraries are all closed and then
reopened as part of the initialization.

\subsection{Closing Libraries}\label{sec:dlclose}

On windows, dlls can only be overwritten when they are not loaded by another
application. For this purpose you can use the \textsf{dlclose} command
in the interpreter window (or command prompt for the command line
version of the Toolbox) to close the loaded dlls.

\textsf{dlclose} deletes the external C++ objects then closes the
dlls. The VDM++ objects are not deleted in the Toolbox but they are
marked as having no external C++ objects associated. If you then try
to execute specifications that would invoke external methods on
existing dlclass objects you will get a run time error. Non-dlclass
method calls can still be done after a \textsf{dlclose} command, though.

Most likely you will have to use \textsf{init} after a
\textsf{dlclose} and after you have copied the new dlls.



\subsection{Creating a Shared Library}\label{sec:create}

The shared library must be compiled using the proper compiler as
required in Section \ref{sec:sysreq}.  In order to create an
executable shared library the interface layer
must fulfill the following requirements:
\begin{itemize}
\item The prototype C functions defined in \texttt{dlclass.h} must be
implemented in the interface layer;
\item Each class in the external code for which there is a
corresponding DL Class should be wrapped as a subclass of
\texttt{DlClass} and an implementation of the class method
\texttt{DlMethodCall} must be provided.
\item The {\tt dlclass.h} header file, which is supplied as part of the
distribution, must be included in the file(s) with the interface layer
code since  it contains prototypes of the type-specific functions of
the \vdmcpplib. 
\end{itemize}

See the example makefiles in Appendices~\ref{app:unixmake}
and~\ref{app:winmake} for the actual compiler flags to use.

A shared library should have the file extension {\tt ".so"} on Unix
and \texttt{.dll} on Windows. 

\newpage
\bibliographystyle{iptes}
\bibliography{ifad}
\newpage
\appendix

\section{System Requirements}
\label{sec:sysreq}

To use the Dynamic Link feature of the {\it VDM++ Toolbox} you
need a Toolbox license which includes the Dynamic Link feature
(a line with the {\tt vppdl} feature must be present in the
license file). The {\it VDM  C++ Library} and a compiler for creating
an executable shared library from the integrated code are also required.

This feature runs on the following combinations:
\begin{itemize}
\item Microsoft Windows 2000/XP/Vista and Microsoft Visual C++ 2005 SP1
\item Mac OS X 10.4, 10.5
\item Linux Kernel 2.4, 2.6 and GNU gcc 3, 4
\item Solaris 10
\end{itemize}

Note that shared libraries are \textit{extremely} compiler
sensitive. Deviation from the above combinations will most likely lead
to runtime errors when using DL Classes.

The compiler is required in order to create an executable shared
library of the integrated code. 
 
For installation of the Toolbox itself see Section 2 in \cite{UserManPP-CSK}
and for installing the {\it VDM C++ Library} see Section 2 in \cite{CGManPP-CSK}.

\newpage
\section{dlclass.h}\label{app:dlclass_h}

\verbatiminput{dl-example/dlclass.h}

\newpage
\section{Example Files}\label{app:interfacelayer}

\subsection{VDM++ Layer}

\verbatiminput{dl-example/bigint.vpp}

\subsection{bigint\_dl.h}

\verbatiminput{dl-example/bigint_dl.h}

\subsection{bigint\_dl.cc}

\verbatiminput{dl-example/bigint_dl.cc}

\newpage
\section{Unix Makefile}\label{app:unixmake}

\verbatimtabinput{dl-example/Makefile}

\newpage
\section{Windows Makefile}\label{app:winmake}

\verbatimtabinput{dl-example/Makefile.win32}





\end{document}

